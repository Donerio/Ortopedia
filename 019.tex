\documentclass[]{article}
\usepackage{lmodern}
\usepackage{amssymb,amsmath}
\usepackage{ifxetex,ifluatex}
\usepackage{fixltx2e} % provides \textsubscript
\ifnum 0\ifxetex 1\fi\ifluatex 1\fi=0 % if pdftex
  \usepackage[T1]{fontenc}
  \usepackage[utf8]{inputenc}
\else % if luatex or xelatex
  \ifxetex
    \usepackage{mathspec}
  \else
    \usepackage{fontspec}
  \fi
  \defaultfontfeatures{Ligatures=TeX,Scale=MatchLowercase}
\fi
% use upquote if available, for straight quotes in verbatim environments
\IfFileExists{upquote.sty}{\usepackage{upquote}}{}
% use microtype if available
\IfFileExists{microtype.sty}{%
\usepackage{microtype}
\UseMicrotypeSet[protrusion]{basicmath} % disable protrusion for tt fonts
}{}
\usepackage[unicode=true]{hyperref}
\hypersetup{
            pdfborder={0 0 0},
            breaklinks=true}
\urlstyle{same}  % don't use monospace font for urls
\usepackage{graphicx,grffile}
\makeatletter
\def\maxwidth{\ifdim\Gin@nat@width>\linewidth\linewidth\else\Gin@nat@width\fi}
\def\maxheight{\ifdim\Gin@nat@height>\textheight\textheight\else\Gin@nat@height\fi}
\makeatother
% Scale images if necessary, so that they will not overflow the page
% margins by default, and it is still possible to overwrite the defaults
% using explicit options in \includegraphics[width, height, ...]{}
\setkeys{Gin}{width=\maxwidth,height=\maxheight,keepaspectratio}
\IfFileExists{parskip.sty}{%
\usepackage{parskip}
}{% else
\setlength{\parindent}{0pt}
\setlength{\parskip}{6pt plus 2pt minus 1pt}
}
\setlength{\emergencystretch}{3em}  % prevent overfull lines
\providecommand{\tightlist}{%
  \setlength{\itemsep}{0pt}\setlength{\parskip}{0pt}}
\setcounter{secnumdepth}{0}
% Redefines (sub)paragraphs to behave more like sections
\ifx\paragraph\undefined\else
\let\oldparagraph\paragraph
\renewcommand{\paragraph}[1]{\oldparagraph{#1}\mbox{}}
\fi
\ifx\subparagraph\undefined\else
\let\oldsubparagraph\subparagraph
\renewcommand{\subparagraph}[1]{\oldsubparagraph{#1}\mbox{}}
\fi

% set default figure placement to htbp
\makeatletter
\def\fps@figure{htbp}
\makeatother


\date{}

\begin{document}

Coxalgie

Alla base di un dolore all'anca possono esserci molteplici cause
patologiche, di cui la più comune è l'\textbf{artros}i dell'anca. Ma
spesso si ha la tendenza di sottovalutare tutte le altre cause di
coxalgia e magari si fa erroneamente diagnosi di Coxartrosi perché non
si sa da cos'altro possa derivare il dolore.

Per evitare questi errori ovviamente non si può prescindere dal fare una
buona \emph{anamnesi} ed un buon \emph{esame obiettivo}. Solo in un
secondo momento, saranno importanti le tecniche di imaging che devono
infatti essere indirizzate dalle informazioni raccolte durante
l'anamnesi e l'esame obiettivo.

{[}Caso clinico 1: si consideri un paziente giovane (15-20 anni) che non
ha altre patologie e che arriva all'attenzione del medico perché tutte
le notti non riesce a dormire a causa di un dolore all'anca. In questo
caso, nonostante le esigue informazioni, possiamo già essere indirizzati
verso una neoplasia benigna: l'\textbf{osteoma osteoide}. Quest'ultima
infatti si caratterizza proprio per questo dolore tipicamente notturno,
è una neoplasia che non dà alcun tipo di metastasi ed è molto fastidiosa
perché, finché non viene rimosso il tessuto tumorale, provoca un forte
dolore gravativo ed urente nella sede di localizzazione.

Per confermare la diagnosi in questo caso si utilizza un criterio ex
adiuvantibus: la somministrazione di \emph{aspirina}. Il dolore dovuto a
questo tumore infatti recede dopo somministrazione di acido
acetilsalicilico ma non dopo somministrazione di altri FANS.

Se il criterio ex adiuvantibus conferma la nostra diagnosi, (e cioè se
il dolore recede dopo somministrazione di aspirina) andremo ad usare una
TC per valutare dimensioni e localizzazione precisa della neoplasia
benigna.{]}

{[}Caso clinico 2: paziente di sesso maschile di 35-40 anni che
riferisce un dolore sacro-iliaco. Riferisce inoltre di avere la
psoriasi. Anche in questo caso è un dolore che si accentua di notte per
poi affievolirsi durante il giorno. In questo caso però saremo orientati
verso \textbf{un'artrite psoriasica sieronegativa}.{]}

{[}Caso clinico 3: paziente di 12 anni abbastanza corpulento con dolore
all'anca. Molto difficile che abbia un'artrite sieronegativa. Piuttosto,
è più facile che abbia una \textbf{epifisiolisi}, una patologia dovuta a
un difetto delle cartilagini di coniugazione, a causa del quale la testa
del femore tende a scivolare verso il basso provocando dolore.{]}

Da tutti questi casi clinici si capisce che l'anamnesi è assolutamente
fondamentale per fare una buona diagnosi.

Uguale importanza riveste l'esame obiettivo. Con quest'ultimo si
valutano le condizioni generali del paziente e, più in particolare, è
possibile valutare:

\begin{enumerate}
\def\labelenumi{\arabic{enumi}.}
\item
  un eventuale disturbo della deambulazione.
\end{enumerate}

Le \textbf{zoppie} sono di tre tipi:

\begin{itemize}
\item
  Zoppie \emph{di fuga}: il paziente "scappa" dall'appoggio in cui sente
  dolore e quindi fa il passo velocemente;
\item
  Zoppie \emph{da dismetria}: una gamba è più corta dell'altra (ad
  esempio a causa di una precedente frattura);
\item
  Zoppie \emph{da insufficienze glutee}: il gluteo non riesce a
  stabilizzare il bacino quando si fa un passo, per cui si ha la
  cosiddetta andatura anserina, "a papera", con il bacino che cade
  dall'altro lato quando si fa un passo.
\end{itemize}

Ci sono anche altre cause di zoppie, ma queste tre sono le più comuni.

\begin{enumerate}
\def\labelenumi{\arabic{enumi}.}
\item
  l'ampiezza di movimento dell'anca, che se ridotta può orientare verso
  la diagnosi di coxartrosi.
\end{enumerate}

{[}Caso clinico: paziente di sesso femminile, che pratica ginnastica
artistica, ha dolore all'anca trocanterica. Durante la deambulazione si
sente il classico "toc, cloc" dovuto al fatto che l'anca scatta avanti
ed indietro. Siamo in questo caso nell'ambito delle anche a scatto
trocanteriche o delle tendinoborsiti glutee.{]}

\emph{Possibili cause di coxalgia}

Spesso, soprattutto in ambito fisiatrico, viene abusata la diagnosi di
\textbf{sindrome del piriforme}. Quest'ultima è una patologia in cui il
piriforme presenta un'\emph{ipertrofia} (primitiva o per cause
secondarie, come ad esempio il piriforme di un atleta) e va a
\emph{comprimere il nervo sciatico} che passa tra il piriforme stesso ed
il gemello superiore.

Questo causa un dolore che parte dalla zona sacro-iliaca e si irradia
posteriormente nella gamba fino al ginocchio, più raramente fino al
piede.

E' una patologia che ovviamente esiste, ma non è così comune. Per cui
non bisogna abusarne in sede di diagnosi.

Esiste anche una patologia legata al legamento rotondo. Quest'ultimo ha
una funzione trofica e stabilizzatrice sulla testa del femore in età
infantile (queste funzioni vengono perse nel corso della vita). Quando
questo legamento si lacera, si può sviluppare una sensazione dolorosa
all'anca. E' quindi una \textbf{coxalgia da legamento rotondo}.

Nella \textbf{coxa vara} l'angolo formato dagli assi della testa e del
corpo del femore è troppo piccolo, quasi un angolo retto. In questi
soggetti le forze di flessione che si scaricano sulla testa del femore
sono molto elevate e quindi aumenta enormemente il rischio di frattura
della testa del femore.

La TC è un esame molto utile, ma a volte può risultare superfluo perché
in alcuni casi la diagnosi è già ben evidente alla radiografia.

Questo esame va infatti utilizzato solo quando si hanno forti dubbi sul
referto di una radiografia (esempio: sospetto di presenza di metastasi
ossea).

\emph{Tumore di Ewing}

\includegraphics[width=5.06250in,height=3.10417in]{media/image1.jpeg}

Il tumore di Ewing è una neoplasia molto maligna con picco di incidenza
tra i 10-19 anni ed i 20-29 anni. La possibilità di diagnosi di questo
tumore va tenuta sempre in elevata considerazione in soggetti giovani,
proprio perché questo tumore è molto aggressivo e può portare a morte in
pochissimo tempo.

Coxartrosi

La coxatrosi è la più comune causa di coxalgia. Distinguiamo:

\begin{itemize}
\item
  coxartrosi \emph{primitive}, in cui non ci sono patologie che possono
  essere considerate causa dell'insorgenza della coxartrosi;
\item
  coxartrosi \emph{secondarie}, in cui è presente una patologia che è
  responsabile dell'avvento della coxartrosi.
\end{itemize}

\begin{quote}
Tra le patologie che causano le coxartrosi secondarie si trovano:
\end{quote}

\begin{itemize}
\item
  tutte le malattie infiammatorie dismetaboliche: artrite reumatoide,
  artrite psoriasica, spondiloartrite anchilopoietica, lupus eritematoso
  sistemico, etc. ;
\item
  malattie infiammatorie croniche intestinali: morbo di Crohn,
  retto-colite ulcerosa;
\item
  necrosi della testa del femore;
\item
  impingement femoro-acetabolare;
\item
  algoneurodistrofia di Sudeck.
\end{itemize}

Le ultime tre cause di artrosi secondaria meritano una maggiore
attenzione.

\emph{Necrosi della testa del femore}

E' una patologia che si verifica quando l'afflusso di sangue alla testa
del femore è compromesso. Anche in questo caso abbiamo moltissime cause:

\begin{itemize}
\item
  idiopatiche;
\item
  frattura del collo del femore, in questo caso vengono tranciati i vasi
  arteriosi che si occupano della vascolarizzazione della testa del
  femore;
\item
  post-traumatica, dovuta a traumi (come ad esempio un intervento
  chirurgico all'anca);
\item
  etilica, dovuta all'eccessivo consumo di bevande alcoliche;
\item
  post-cortisonica;
\item
  altre cause più rare.
\end{itemize}

\emph{Impingement Femoro-Acetabolare}

\includegraphics[width=4.73958in,height=2.59375in]{media/image2.jpeg}

L'impingement femoro-acetabolare è una patologia articolare che, nei
soggetti affetti, aumenta enormemente il rischio di sviluppare artrosi
in assenza di fattori di rischio (familiarità, obesità, lussazioni e
sub-lussazioni dell'anca, fratture etc.).

Questo è dovuto al fatto che in questi soggetti sono presenti delle
\textbf{anomalie di conformazione} della testa del femore o
dell'acetabolo che non permettono un normale rapporto di adesione tra
femore ed acetabolo.

Si conoscono tre tipi di impingement femoro-acetabolare:

\begin{enumerate}
\def\labelenumi{\arabic{enumi}.}
\item
  Impingement di \textbf{tipo Cam}: è presente una convessità ossea che
  sporge dalla testa del femore e, nei movimenti di flesso-estensione e
  di rotazione dall'anca, va a scontrarsi patologicamente con
  l'acetabolo e con la cartilagine articolare;
\item
  Impingement di \textbf{tipo Pincer}: è presente un acetabolo che
  avvolge troppo la testa del femore (normalmente dovrebbe coprire il
  50\% della testa del femore) ed anche in questo caso ci sarà un
  continuo e patologico "scontro" tra testa del femore ed acetabolo;
\item
  Impingement di \textbf{tipo misto}: è la somma degli altri due
  impingement ed ovviamente in questo caso la patologia è ancora più
  grave.
\end{enumerate}

\emph{Algoneurodistrofia di Sudeck}

\includegraphics[width=5.29167in,height=2.51042in]{media/image3.jpeg}

E' un \textbf{cortocircuito} a livello \textbf{del sistema simpatico}
che causa un circolo vizioso tra vasodilatazione e dolore.

Questo porta a rarefazione ossea fino a quando la demineralizzazione
dell'osso è tale da configurare la cosiddetta \emph{atrofia vitrea}.

L'atrofia vitrea si può notare all'imaging: la zona della testa del
femore è radiopaca (bianca), esattamente come lo è l'urina al centro
dell'immagine. Questo perché c'è un voluminosissimo edema interstiziale
nella zona interessata.

Anatomia e Fisiologia dell'Anca

L'anca è una struttura ossea che è data dall'incontro della
\textbf{testa del femore} con \textbf{l'acetabolo}. La ricettività di
quest'ultimo nei confronti della testa del femore è aumentata dal
\emph{cercine fibro-cartilagineo}. C'è poi \emph{sistema capsulare} che
è un sistema di rinforzo dell'articolazione.

L'angolo di inclinazione tra l'asse della testa del femore e l'asse del
corpo del femore è normalmente di 125\(\). Inoltre esiste un angolo di
declinazione di circa 25\(\) tra l'asse bicondilare e l'asse
cervico-cefalico del femore.

Alterazioni di questi angoli possono riflettersi sulla biomeccanica
dell'anca e provocare artrosi.

\includegraphics[width=4.68750in,height=2.60417in]{media/image4.jpeg}

\begin{quote}
L'immagine sopra mostra il cosiddetto \textbf{schema di Powell}, molto
importante nell'impianto della protesi d'anca. Questo schema dice che,
considerando una condizione di appoggio monopodalico, le forze che si
scaricano sull'anca (vettore R nell'immagine) sono la sommatoria della
forza di gravità (e quindi del peso corporeo, vettore K) e della forza
generata dalla stabilizzazione del bacino da parte dei muscoli (vettore
M).

La forza risultante R si scarica sulla testa del femore e, a seconda di
come siano posizionate tra loro la testa del femore e l'acetabolo, R può
variare e può quindi generare sollecitazioni verso l'una o l'altra parte
della testa del femore.
\end{quote}

\begin{itemize}
\item
  Ad esempio in una coxa valga (angolo di inclinazione tra testa del
  femore e corpo del femore maggiore di 125\(\)) ci sarà una patologica
  sollecitazione della parte superiore della testa del femore.
\item
  Al contrario una coxa vara (angolo di inclinazione minore di 125\(\))
  solleciterà maggiormente la parte inferiore della testa del femore.
\end{itemize}

\begin{quote}
\includegraphics[width=4.38542in,height=2.44792in]{media/image5.jpeg}
\end{quote}

Nell'immagine sopra si può vedere l'irrorazione della testa del femore,
che è molto importante tenere a mente nei casi di osteonecrosi da
frattura del collo del femore.

L'\textbf{irrorazione} prevede un \emph{anello alla base del collo} del
femore da cui si diramano dei vasi che si fanno \emph{intracapsulari}
decorrendo dal collo alla testa (che sono proprio quelli che vengono
lesionati in caso in frattura del collo del femore).

\includegraphics[width=4.39583in,height=2.34167in]{media/image6.jpeg}Fisiopatologia
della Coxartrosi

Nell'immagine vediamo una testa del femore. In condizioni fisiologiche
dovremmo trovarla ricoperta di cartilagine, lucida e lucente.
Nell'immagine invece si vede bene la totale assenza di cartilagine.

L'\textbf{artrosi} è una patologia degenerativa.

E' diversa dall'\emph{artrite}, ma ci possono essere dei casi in cui le
due patologie \emph{si sovrappongono}. In questo caso le artrosi hanno
un aspetto di tipo "sinoviale".

Gli esami per confermare la diagnosi di artrite reumatoide saranno
probabilmente negativi, però se andiamo ad incidere la capsula vediamo
fuoriuscire del liquido che è dovuto ad una sinovite villosa.

\begin{itemize}
\item
  Nelle \textbf{coxartrosi primitive} non ci sono delle patologie
  scatenanti la coxartrosi stessa. In questo caso la testa del femore
  sarà uniformemente consumata ed ancora sferica, e dopo incisione non
  si trovano necrosi né liquido o altre caratteristiche che possono
  indirizzare verso una patologia che può avere scatenato l'artrosi.
\item
  Tra le \textbf{coxartrosi secondarie} più comuni si riconoscono:
\end{itemize}

\begin{itemize}
\item
  coxartrosi secondarie post-traumatiche (soprattutto da fratture);
\item
  coxartrosi secondarie a displasie congenite dell'anca (coxa vara e
  coxa valga);
\item
  coxartrosi secondarie ad impingement femoro-acetabolare;
\item
  coxartrosi secondarie ad artriti (infiammatorie, settiche etc..);
\item
  coxartrosi secondarie a malattie dell'infanzia come il morbo di
  Perthes e l'epifisiolisi.
\end{itemize}

Il \textbf{morbo di Perthes} è una patologia tipica dei bambini che si
presentano con coxalgia verso i 6-7 anni. La causa della patologia
risiede nel \emph{ridotto afflusso di~sangue alla testa del femore}, la
quale può andare incontro, prima, ad~osteonecrosi e, successivamente,
a~frattura.

Questa sofferenza trofica porta all'appiattimento della testa del femore
che può portare verso la coxartrosi.

L'\textbf{epifisiolisi} è anch'essa una patologia del bambino ma in
questo caso è dovuta a un \emph{difetto delle cartilagini di
coniugazione}, a causa del quale la testa del femore tende a scivolare
verso il basso provocando dolore.

L'artrosi degenerativa dell'anca provoca una limitazione del movimento
femorale. Ne consegue un importante impatto sulla funzione del cingolo
pelvico e sul rachide lombare, poiché entrambi tentano di compensare la
perdita do movimento a livello dell'anca.

\textbf{\emph{Valutazione}}

All'\textbf{ispezione} è evidente, soprattutto nelle forme più avanzate,
un atteggiamento in adduzione, flessione e rotazione esterna, più
raramente interna.

Alla \textbf{palpazione} possono riscontrarsi punti dolorosi
dell'articolazione.

Alla \textbf{valutazione dei singoli movimenti} si ha una riduzione
della flessione e, in misura maggiore e più precocemente,
dell'abduzione, dell'intra ed extrarotazione e dell'estensione (il
paziente fa fatica ad infilare le calze, a calzare le scarpe, a scendere
o salire le scale). Negli stadi iniziali solo attività di maggior
impegno articolare, come quelle sportive, determinano un aggravamento
della situazione. Instaurata la malattia si avranno i \textbf{risultati
a livello radiologico}: riduzione dell'interlinea articolare; gli
osteofiti costituiscono un segno radiologico precoce e possono rilevarsi
fenomeni di osteosclerosi e rarefazione a stampo (geodi).

L'\textbf{evoluzione} è lenta ed inesorabile e conduce ad un
aggravamento progressivo con limitazione dei movimenti fino
all'anchilosi. La degenerazione artrosica può avere molteplici cause, di
particolare importanza è l'aspetto biomeccanico: una disfunzione dei
sistemi articolare e neuromuscolare può dar luogo ad una cattiva
distribuzione dei carichi, favorendo lo sviluppo di coxartrosi.

Diagnosi di Coxartrosi

E' basata innanzitutto sulla presenza di \textbf{dolore all'anca}
(coxalgia) e sulla presenza di \textbf{zoppie}.

I sintomi sono il \textbf{dolore alla marcia}, avvertito all'inguine ed
alla parte anteriore della coscia, il quale compare anche dopo essere
rimasti seduti a lungo, mentre scompare in posizione orizzontale. Tipica
è anche la \textbf{progressiva limitazione funzionale} prima dei
movimenti di \emph{intrarotazione} (difficoltà nell'uscire dalla vasca o
nel salire su una bici), poi di quelli di \emph{abduzione} e quindi di
\emph{adduzione}. Si accompagna ad una \emph{scoliosi lombare insorta
per compensazione} di una ``atteggiamento viziato'' in adduzione,
flessione ed extrarotazione della coscia. L'Rx del bacino rivela un
restringimento della rima articolare, con la direzione della
dislocazione delle testa del femore indica inoltre se il danno
cartilagineo è uniforme o meno. La diagnosi differenziale va posta con
l'artrite dell'anca, in cui si ha un dolore di tipo flogistico, con
aumento degli indici di flogosi ed erosioni, e le neoplasie dell'osso,
in cui si ha un dolore simile ma che non recede a riposo.

Generalmente è abbastanza agevole la conferma diagnostica tramite la
radiografia, ma in alcuni casi, se sono presenti dei dubbi, si può
ricorrere alla RMN o alla TC.

Una ricostruzione tridimensionale può essere richiesta in preparazione
ad un intervento chirurgico.

Trattamento della Coxartrosi

\begin{quote}
\textbf{\emph{Obiettivi}}

• Riduzione del dolore e della reattività articolare

• Aumento dell'ampiezza articolare

• Miglioramento della stabilità articolare

\textbf{\emph{Principali strategie di intervento terapeutico}}

− Tecniche articolari: trazione, traslazione, roll, swing, movimenti
osteocinematici

− Tecniche legamentose/tendinee: mobilizzazioni, trazioni

− Tecniche muscolari: stretching, contrazioni isotoniche ed isometriche

− Tecniche neuro dinamiche: sollevamento dell'arto inferiore esteso,
propriocettiva

− Combinazione di 2 o più tecniche

− Massoterapia

− Termoterapia, elettroterapia, ultrasuonoterapia, laserterapia, ortesi

\textbf{\emph{RIABILITAZIONE PRECHIRURGICA DELL'ANCA }}

L'artroprotesi totale di anca prevede un percorso riabilitativo ben
definito finalizzato a restituire la funzionalità dell'articolazione e
possibilmente a migliorare quella precedente all'intervento.

Studi recenti hanno evidenziato come un trattamento pre-chirurgico dia
le migliori garanzie post-intervento dal punto di vista della
performance muscolare e motoria; ciò soprattutto in vista della ripresa
dell'attività lavorativa e sociale del soggetto.

L'obiettivo del trattamento pre-chirurgico è far sì che il paziente si
abitui sin da subito a svolgere esercizi per il recupero del
tonotrofismo muscolare; in questo modo si presenterà all'intervento con
una buona condizione muscolare e ciò accelererà i tempi di recupero.

Inoltre la riabilitazione pre-chirurgica educa il paziente all'utilizzo
degli arti brachiali e alla gestione del carico.

Nella fase pre-operatoria il fisiatra deve

• valutare la sintomatologia dolorosa

• valutare la capacità di deambulare con o senza ausili

• valutare se dolore e impedimento possono causare disabilità e
peggiorare gli atti della vita quotidiana. Esiste a questo scopo la
Scala di valutazione WOMAC, che appunto valuta la sintomatologia
dolorosa, l'arco di movimento dell'articolazione e la capacità del
soggetto di attendere agli atti della vita quotidiana. Essa inoltre
permette di monitorare l'esito dell'intervento chirurgico e di quello
riabilitativo.

Riassumendo, nella fase pre-operatoria il paziente va istruito su:
esercizi preintervento, consigli sulle corrette posture da assumere nel
periodo di ricovero, uso di ausili e calzature adeguate per il percorso
riabilitativo previsto. Inoltre, in base alle condizioni cliniche
preintervento del paziente si deciderà la sede del trattamento
riabilitativo. Se il soggetto è giovane e compliante e necessita di un
rapido recupero va diretto verso un percorso ambulatoriale; un altro
soggetto anziano e con comorbilità va invece riabilitato nelle strutture
apposite o a casa.

\textbf{\emph{RIABILITAZIONE POSTCHIRURGICA DELL'ANCA }}

Il trattamento post-intervento deve

- prevenire i danni secondari all'immobilità

- agire sul dolore

- recuperare il tonotrofismo muscolare

- recuperare il corretto schema del passo, perché questi pazienti già in
partenza ce l'hanno alterato.

Il soggetto presenta infatti un' ipotonotrofia muscolare importante,
soprattutto riguardante i muscoli glutei e ischio-crurali, quadricipite,
tensore della fascia lata, adduttori e abduttori di anca che
stabilizzano l'articolazione.

Il programma riabilitativo è alquanto denso e preciso, si divide in:

\textbf{\emph{FASE DI MASSIMA PROTEZIONE (4-7 GIORNI)}}

In questa fase è necessario:

- prevenire la dislocazione o la sublussazione, le complicanze
polmonari;

- raggiungere l'indipendenza nei trasferimenti prima della dimissione;

- mantenere la forza degli arti superiori e dell'arto inferiore sano con
esercizi attivi;

- mantenere la mobilità dell'arto operato con esercizi di mobilizzazione
passiva, in flessione, abduzione, rotazioni non controindicate dal tipo
di accesso chirurgico;

- prevenire una contrattura in flessione ponendo l'arto sano in massima
flessione e lasciando rilassato l'arto operato, allungando i flessori
dell'anca.

\textbf{1° giorno postintervento}

Posizionare il paziente seduto sul letto, e consigliare di eseguire
movimenti di flesso-estensione attiva degli adduttori e abduttori
dell'arto controlaterale, in modo da stimolare nei giorni successivi gli
stessi esercizi sull'arto operato. Secondo la scuola americana, il
paziente va messo in carico dopo 24 h dall'intervento, questo grazie
agli effetti della terapia antalgica infusiva; in Italia si è più cauti,
ed il paziente viene verticalizzato in terza giornata. Questo perchè in
primis vanno monitorate le complicanze vascolari dell'intervento: è vero
che il paziente assume anticoagulanti e subisce un input-system
finalizzato alla ginnastica vascolare, ma gli imprevisti non vanno
sottovalutati. Il paziente poi ha un catetere per il drenaggio
dell'articolazione, e questo gli impedisce di deambulare (NB nelle
protesi d'anca non si usa praticamente più,mentre si usa per quelle di
ginocchio).

\textbf{2° giorno}

• Il paziente si siede sul letto in posizione adeguata con le ginocchia
leggermente flesse. Se il paziente è in buone condizioni si può sedere
in poltrona, meglio se gestito con le gambe fuori dal letto. Infatti il
mantenimento del controllo del tronco aiuterà la successiva
verticalizzazione del paziente in terza giornata.

• Rimozione dei drenaggi al fine di poter deambulare.

- \textbf{3° giorno}

Verticalizziamo il paziente e lo facciamo camminare con carico sfiorante
o progressivo, usando un deambulatore con appoggio ascellare.

\textbf{4° giorno}

Si prosegue il programma con deambulatore con appoggio ascellare.

\textbf{5° giorno}

Se tutto funziona, il paziente usa due bastoni canadesi ad appoggio
antibrachiale con un ritmo a 3 tempi. •

\textbf{6° giorno}

Prosegue l'allenamento coi bastoni antibrachiali.

\textbf{7° giorno}

Il paziente viene dimesso. Il periodo di ricovero è quindi di 6-8
giorni. Una volta dimessi, i pazienti possono essere inviati a domicilio
o in una struttura pubblica o privata per il programma riabilitativo;
qui svolgeranno un trattamento estensivo di 1 ora al giorno, oppure
intensivo di 2 ore al giorno con personale dedicato (cosa non possibile
in un reparto di ortopedia). Il setting riabilitativo dipende dalle
condizioni del paziente e dalla compliance dei familiari, oltre che
dall'assenza di barriere architettoniche a domicilio. A questo scopo il
servizio territoriale invia un terapista a domicilio che, in 5 sedute,
istruirà il paziente e i familiari su come usare il deambulatore e i
bastoni canadesi antibrachiali. Spesso infatti il paziente esce
dall'ospedale usando ancora il deambulatore ad appoggio ascellare, e la
figura del terapista a domicilio diventa quindi essenziale. Riassumendo,
il trattamento post-intervento riabilita la muscolatura glutea, gli
adduttori di coscia e il quadricipite. Per fare ciò si deve recuperare
l'arco di movimento e istruire il paziente sull'uso dei bastoni
canadesi, che non andrebbero abbandonati prima di 1mese-1 mese e mezzo.
E' utile poi fare il controllo della protesi dopo 30 giorni
dall'intervento e somministrare al paziente la scala di valutazione
WOMAC per avere un numero che quantifichi il recupero. NB In caso di
Problemi sul trasferimento del paziente nel centro riabilitativo
(mancanza del posto), il trattamento deve proseguire nel reparto di
ortopedia dove questi è stato operato.

\textbf{\emph{FASE DI MODERATA E MINIMA PROTEZIONE (8 giorni/6-12
settimane) }}

È necessario: - recuperare la forza, la resistenza ed il movimento
articolare dell'arto operato con:

• Mobilizzazione passiva seguendo la corretta artrocinematica dell'anca

• Flesso estensione di anca e ginocchio strisciando il tallone sul
lettino

• Flessione dell'anca a ginocchio esteso, assistita poi attiva, infine
contro resistenza

• Abduzione dell'anca sul piano del letto, poi con resistenza elastica,
infine contro gravità in decubito laterale controlaterale all'arto
operato

• Esercizio del ponte, prima con entrambe le gambe, poi con l'arto
operato flesso e l'altro esteso e sollevato dal lettino

• Esercizi di flessione e abduzione attiva dell'arto operato in
posizione eretta - qualora sia concesso il carico monopodalico, si
possono eseguire mini-squat, passi laterali e affondi - migliorare
l'equilibrio mediante esercizi su superfici instabili (tavole
propriocettive), incrementare il carico durante la deambulazione e
correggere eventuali compensi; - migliorare la resistenza
cardiorespiratoria con attività aerobiche, - preparare il paziente al
ritorno alle normali attività con esercizi funzionali come salire e
scendere le scale, percorrere a piedi distanze crescenti.
\end{quote}

\begin{itemize}
\item
  In presenza di una coxartrosi secondaria, ove possibile si cerca di
  eliminare la causa.
\end{itemize}

Ad esempio se il paziente ha un impingement femoro-acetabolare si può
cercare di ridurlo con un intervento di artroscopia.

\begin{itemize}
\item
  Negli altri casi l'alternativa è l'\textbf{osteotomia} eseguita
  sull'acetabolo, sulla testa del femore o su entrambi.
\item
  Quando però l'artrosi è troppo avanzata, non si può fare a meno di
  fare un intervento che preveda l'impianto di una \textbf{protesi
  d'anca}.
\end{itemize}

La protesi ha il compito di ripristinare la corretta geometria dell'anca
per ripristinare la funzione e per ridurre l'usura che aveva causato la
coxartrosi.

Normalmente nell'artrosi d'anca non si possono fare delle endoprotesi
parziali, ma la protesi è sempre e comunque una \emph{protesi totale}.
Questa prevede la sostituzione della testa del femore e dell'acetabolo.

Le \emph{endoprotesi parziali} possono essere invece una valida
possibilità in un paziente anziano che abbia subito una frattura del
collo del femore. In quel caso si può sostituire soltanto il femore
lasciando stare l'acetabolo.

Il 94\% delle protesi d'anca impiantate in Emilia Romagna sono ancora
sopravviventi dopo 10 anni.

Nell'impianto di una protesi è di fondamentale importanza lo
\textbf{schema di Powell}. Bisogna infatti usare una protesi che, una
volta impiantata, generi un sistema il più possibile conforme allo
schema di Powell. Se questo non succede ovviamente la protesi può non
avere il successo terapeutico sperato.

Per questo motivo, non esiste una protesi che vada bene per qualsiasi
paziente. Esistono infatti moltissimi tipi di protesi con
caratteristiche diverse ed è di fondamentale importanza la scelta di una
protesi che sia adeguata per un determinato paziente.

Nella scelta della protesi vanno considerati parametri come

\begin{itemize}
\item
  l'età
\item
  il sesso
\item
  la morfologia ossea
\item
  la qualità ossea.
\end{itemize}

Inoltre, la ricostruzione tridimensionale dell'anca del paziente può
essere di grande aiuto.

\begin{itemize}
\item
  La \textbf{chirurgia mini-invasiva} è una tecnica con la quale si
  cerca di danneggiare al minimo i tessuti sani (in particolare il
  tessuto muscolare) per far sì che possano essere ben ricostruiti.
\end{itemize}

E' una metodica utile ma purtroppo non può essere utilizzata in tutti i
casi.

\end{document}
