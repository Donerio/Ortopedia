\documentclass[]{article}
\usepackage{lmodern}
\usepackage{amssymb,amsmath}
\usepackage{ifxetex,ifluatex}
\usepackage{fixltx2e} % provides \textsubscript
\ifnum 0\ifxetex 1\fi\ifluatex 1\fi=0 % if pdftex
  \usepackage[T1]{fontenc}
  \usepackage[utf8]{inputenc}
\else % if luatex or xelatex
  \ifxetex
    \usepackage{mathspec}
  \else
    \usepackage{fontspec}
  \fi
  \defaultfontfeatures{Ligatures=TeX,Scale=MatchLowercase}
\fi
% use upquote if available, for straight quotes in verbatim environments
\IfFileExists{upquote.sty}{\usepackage{upquote}}{}
% use microtype if available
\IfFileExists{microtype.sty}{%
\usepackage{microtype}
\UseMicrotypeSet[protrusion]{basicmath} % disable protrusion for tt fonts
}{}
\usepackage[unicode=true]{hyperref}
\hypersetup{
            pdfborder={0 0 0},
            breaklinks=true}
\urlstyle{same}  % don't use monospace font for urls
\usepackage{graphicx,grffile}
\makeatletter
\def\maxwidth{\ifdim\Gin@nat@width>\linewidth\linewidth\else\Gin@nat@width\fi}
\def\maxheight{\ifdim\Gin@nat@height>\textheight\textheight\else\Gin@nat@height\fi}
\makeatother
% Scale images if necessary, so that they will not overflow the page
% margins by default, and it is still possible to overwrite the defaults
% using explicit options in \includegraphics[width, height, ...]{}
\setkeys{Gin}{width=\maxwidth,height=\maxheight,keepaspectratio}
\IfFileExists{parskip.sty}{%
\usepackage{parskip}
}{% else
\setlength{\parindent}{0pt}
\setlength{\parskip}{6pt plus 2pt minus 1pt}
}
\setlength{\emergencystretch}{3em}  % prevent overfull lines
\providecommand{\tightlist}{%
  \setlength{\itemsep}{0pt}\setlength{\parskip}{0pt}}
\setcounter{secnumdepth}{0}
% Redefines (sub)paragraphs to behave more like sections
\ifx\paragraph\undefined\else
\let\oldparagraph\paragraph
\renewcommand{\paragraph}[1]{\oldparagraph{#1}\mbox{}}
\fi
\ifx\subparagraph\undefined\else
\let\oldsubparagraph\subparagraph
\renewcommand{\subparagraph}[1]{\oldsubparagraph{#1}\mbox{}}
\fi

% set default figure placement to htbp
\makeatletter
\def\fps@figure{htbp}
\makeatother


\date{}

\begin{document}

\emph{Piede cavo}

\emph{Per quanto riguarda il piede cavo siamo nella deformità opposta
rispetto al piede piatto. Morfologicamente avremo un
\textbf{\emph{aumento della volta plantare, una rotazione interna del
tallone con una rotazione di tutto l'arto inferiore all'esterno}},
mentre nel piede piatto c'è una rotazione all'interno. Il piede cavo sta
tra il piede normale e la massima espressione di eccesso di pronazione,
ovvero il piede piatto.}

\emph{Definizione}

Deformità caratterizzata dal \textbf{punto di vista morfologico} dalla
presenza di un \emph{aumento dell'altezza della volta plantare, una
rotazione interna del tallone associata a deformità più o meno
importante del primo dito e di tutte le dita,} in particolare
\textbf{griffe delle dita} dovute ad uno squilibrio muscolare: il 90\%
dei piedi cavi dipende proprio da questo. Da un \textbf{punto di vista
funzionale} si intende per piede cavo una \emph{deformità caratterizzata
da una prevalenza o eccesso di supinazione}.

In questa definizione rientrano numerosi quadri clinici con diversi
aspetti eziopatologici, anatomopatologici e terapeutici.

\emph{Eziologia}

Dal punto di vista eziologico si può distinguere il piede cavo in tre
forme principali:

\begin{quote}
1. Piede Cavo Congenito, fortunatamente di riscontro raro, che si può
osservare già dalla nascita e che si manifesta con un marcato equinismo
asimmetrico dei metatarsi senza un interessamento del retropiede. Per
quanto riguarda l'origine di questa forma, alcuni autori ipotizzano che
la deformità possa essere dovuta a cause genetiche, ma senza un preciso
modello di ereditarietà, mentre secondo altri avrebbe un ruolo
predominante una qualche alterazione meccanica dovuta a mal posizioni
intrauterine o a squilibri muscolari.

2. Piede Cavo Secondario (o Acquisito), che è la forma di piede cavo più
comune, potendo a sua volta essere distinto in piede cavo neurogeno,
miopatico, post-traumatico o degenerativo a seconda della patologia alla
base, e tra queste il più comune è il piede cavo neurogeno, dovuto ad
esempio a poliomielite, spina bifida, malattia di Friederich, paralisi
spastiche, neuropatie sensitivo-motorie e malattia di
Charcot-Maria-Tooth.
\end{quote}

3. Piede Cavo Essenziale (o Idiopatico), in cui non è possibile
identificare una causa specifica alla base. Un tempo queste forme
rappresentavano l'80\% delle forme di piede cavo, mentre oggi, grazie
alla moderne tecniche diagnostiche, la percentuale è scesa al solo 10\%.

\begin{itemize}
\item
  \emph{\textbf{Malattie neurologiche}} (man mano che la neurologia si è
  evoluta in termini di diagnosi le percentuali di piede cavo idiopatico
  si sono notevolmente ridotte). Malattie come l'atassia
  spinocerebellare, la malattia di Charcot Marie Tooth, la poliomielite,
  la paralisi spastica infantile possono comunque dare una deformità in
  cavismo sempre mediata da uno squilibrio muscolare dei muscoli della
  gamba sull'azione sul piede.
\item
  \emph{\textbf{Malattie muscolari}} (miopatie, distrofia muscolare di
  Duchenne)
\item
  \emph{\textbf{Malattie vascolari}}: ad esempio \textbf{retrazione
  ischemica di Volkmann} che c'è nella mano e anche nel piede. Si ha una
  sproporzione tra il tempo di ischemia e quella di vascolarizzazione
  che crea la possibilità che i muscoli vadano in necrosi e che ci sia
  quindi una retrazione di alcuni muscoli e questo crea il piede cavo.
\item
  \emph{\textbf{Fratture }}
\item
  \emph{\textbf{Malattie reumatiche:}} l'artrite reumatoide dà delle
  deformità più o meno tipiche prevalentemente legate al piede piatto
  perché crolla tutto, il ginocchio e il piede vanno in valgo. Ci sono
  però alcune forme di reumatismi infiammatori cronici che non sono
  artrite reumatoide, dove la componente di disallineamento è minore,
  come la sindrome di Reiter, reumatismo psoriasico. In queste forme la
  tendenza non è quella a crollare, ma ad avere delle retrazioni e delle
  ``mummificazioni''. Alle volte può svilupparsi in associazione ad
  artrite reumatoide in quanto nulla vieta che uno abbia un piede cavo
  da prima e che a 30 anni sviluppi artrite reumatoide.
\end{itemize}

\includegraphics[width=2.79167in,height=2.16250in]{media/image1.jpg}

Nel piede cavo da poliomielite c'è una assoluta mancanza del tricipite e
il calcagno, non essendo più sostenuto dal tono muscolare del tricipite,
va giù determinando un cavismo oltre ad una paralisi dei muscoli
anteriori della gamba.

Qui si può osservare l'esito di una fasciotomia. Se la frattura è data
da un trauma che ha solo leso l'osso, può non succedere niente, ma se si
tratta di un trauma importante per cui c'è stata una trazione dei
muscoli e un notevole aumento di volume, si crea la cosiddetta
\textbf{sindrome compartimentale}. I muscoli sono contenuti in
compartimenti inestensibili quindi se questi gonfiano dentro queste
fasce, la pressione compartimentale aumenta e le arterie si chiudono con
conseguente gangrena. Per ridurre la pressione compartimentale può
essere messa in atto una fasciotomia. Quando questo non avviene i
muscoli vanno in sofferenza, c'è una trazione di alcuni muscoli rispetto
ad altri e ciò crea uno squilibrio muscolare che porta al piede cavo.
\includegraphics[width=2.72500in,height=1.96389in]{media/image2.jpg}

\includegraphics[width=3.56042in,height=2.73611in]{media/image3.jpg}

\includegraphics[width=2.93333in,height=2.20000in]{media/image4.jpg}

\emph{(Dalla foto si può notare che mentre nel piede piatto c'è un
astragalo che ruota all'interno e va giù perché il calcagno va
all'esterno, nel piede cavo è il contrario, l'astragalo è più su ed
extra ruotato insieme a tutto l'arto inferiore, rispetto al piede.)}

(foto)

\emph{Patogenesi}

Osservando lo schema a destra, vediamo che le barrette sono le ossa, i
pallini le articolazioni mobili, le frecce i muscoli da immaginare come
degli elastici. Una articolazione possiamo accomunarla alla metatarso
falangea dell'alluce, una alla mediotarsica e l'altra all'articolazione
della caviglia. Il piede riesce a mantenere la sua forma, quella
normale, perché tutti i muscoli si ritrovano in equilibrio. Nel piede
cavo ci possono essere diverse progressioni come ad esempio, \emph{un
tibiale anteriore che funziona meno non trattiene il primo metatarsale
che va in basso dando una forma di piede cavo antero-interno} perché gli
altri metatarsi, diversamente dal primo, non sono andati giù. Abbiamo
quindi uno squilibrio di tutte le altre articolazioni con comparsa della
\textbf{griffe del primo dito}. Poi c'è una \emph{progressione
disto-prossimale}. Ci sono alcune patologie che colpiscono i muscoli
lombricali e interossei del piede con squilibrio disto-prossimale.
Esiste anche una possibilità di progressione combinata prossimale e
distale della deformità.
\includegraphics[width=3.20833in,height=2.42569in]{media/image5.jpg}

\emph{Evoluzione di un piede cavo idiopatico }

\includegraphics[width=3.05000in,height=2.24931in]{media/image6.jpg}\includegraphics[width=3.07014in,height=2.25069in]{media/image7.jpg}

\includegraphics[width=3.12014in,height=2.37222in]{media/image8.jpg}\includegraphics[width=3.23056in,height=2.36806in]{media/image9.jpg}

\emph{Clinica}

Il quadro clinico nel bambino e nell'adulto è differente. \textbf{Nel
bambino} siccome le articolazioni sono mobili, \emph{in scarico si vede
un aumento della volta plantare che scompare sotto carico}. Con il
passare del tempo le articolazioni diventano rigide e la deformità si
struttura. Il piede si presenta tozzo, comincia a comparire un
\textbf{callo} sotto la testa del primo metatarsale, la \textbf{griffe
del primo dito} e una \textbf{callosità dorsale da sfregamento della
calzatura della prima metatarso falangea}. Questa rigidità del primo
metatarsale determina quello che è \emph{l'effetto tripode}. Uno dei
segni del piede cavo è infatti l'\emph{instabilità}. I pazienti infatti,
vanno facilmente incontro a distorsione del retropiede perché hanno un
assetto con appoggio prevalentemente esterno e quindi difficilmente
riescono a far fronte ad una supinazione addizionale come, ad esempio,
quando si corre o si mette il piede in una buca. A forza di avere questo
effetto tripode, compare una callosità sul quinto metatarsale con
comparsa di un arco metatarsale trasverso che normalmente non è
presente, in quanto i metatarsali devono avere tutti uno stesso
appoggio. \emph{Successivamente} si ha la caduta di tutti i metatarsali
con \textbf{callosità su tutto il tallone anteriore}. La caduta di tutto
l'avampiede crea uno squilibrio a livello dei muscoli interossei,
lombricali ed estensori con comparsa di una estensione forzata della
falange basale e una flessione forzata delle falangi distali con
formazione della griffe. Compare anche metatarsalgia che viene
ulteriormente peggiorata dalla griffe delle dita. Se si verifica
un'evoluzione segue questo decorso, ma non è detto che ciò avvenga.

Clinicamente si possono distinguere:

- Piede Cavo Anteriore, che si caratterizza per una deformità
dell'avampiede, mentre il retropiede mantiene la posizione normale, così
che si viene a formare un eccessivo slivellamento in basso
dell'avampiede rispetto al retropiede che supera il valore fisiologico
di 1 cm. Questa forma può essere simmetrica (tutti i metatarsi si
equinizzano allo stesso modo) o asimmetrica (l'equinismo ha un'entità
differente per ciascun metatarso).

- Piede Cavo Posteriore, come si osserva nelle forme paralitiche che
interessano il tricipite surale. Questa paralisi determina infatti una
talizzazione del calcagno, che nelle forme più marcate giunge anche a
vertalizzarsi.

- Piede Cavo Misto, caratterizzato da cavismo associato anteriore e
posteriore.

- Piede Cavo Antero-Interno, che si caratterizza invece per la caduta in
equinismo del 1° metatarsale, mentre gli altri rimangono in posizione,
con conseguente pronazione dell'avampiede e dal varismo del retro piede.

\emph{Sintomatologia}

\begin{itemize}
\item
  Anomalie della marcia
\item
  Frequenti cadute
\item
  Alterazioni del rapporto con la calzatura
\item
  Deformità
\item
  Dolore
\end{itemize}

Nel \emph{piede cavo idiopatico} abbiamo una prevalente caduta
dell'avampiede rispetto al retropiede, in quello della poliomielite una
caduta del retropiede.

\includegraphics[width=3.91042in,height=2.85069in]{media/image10.jpg}

Il paziente presenta un'alterazione della marcia normale (arrivo con il
tallone -- appoggio il tallone anteriore -- trasferimento di carico dal
retropiede all'avampiede -- spinta dell'avampiede fino all'alluce ). Nel
piede cavo c'è una \textbf{\emph{marcia invertita}} a causa della caduta
dell'avampiede rispetto al retro piede. Prima tocca l'avampiede, poi il
retropiede e poi una fase di spinta anche diminuita, se la deformità è
grave. Il piede cavo è un piede ``pronto'', infatti è molto comune negli
atleti. Se c'è un piede cavo neurologico, l'alterazione dinamica è
primitiva e l'alterazione morfologica è secondaria; mentre, se è dovuto
ad una frattura con causa osteo-articolare, primitiva è l'alterazione
morfologica e secondaria quella dinamica.

\includegraphics[width=4.37014in,height=3.12222in]{media/image11.jpg}

Utilizzando un rialzino di alcuni centimetri sotto il tallone, si può
riportare la caviglia in posizione normale ovvero ad angolo retto e con
il tallone orizzontale. Essendoci un cavismo prevalentemente anteriore
\textbf{l'avampiede è slivellato rispetto al retro piede} quindi, per
avere una posizione neutra della caviglia ho bisogno di un tacco.
Togliendo il rialzo, abbiamo una deformità rigida e per poter poggiare
il tallone, \emph{la caviglia va in flessione dorsale con conseguente
dolorabilità nella parte anteriore della caviglia}. Questo perché
l'astragalo davanti è più largo rispetto alla coda quindi ogni volta che
il piede va in flessione dorsale la pinza tibio-peroneale si apre,
costringendo la parte anteriore dell'astragalo continuamente all'interno
della pinza e costringendo, inoltre, a trazione continua i legamenti
della caviglia. Quindi senza tacco tutto il sistema calcaneo -- achilleo
-- plantare è in trazione continua con conseguente fastidio. Infatti, si
hanno tendinopatie d'Achille, fascite plantare.

\emph{Diagnosi}

\begin{itemize}
\item
  Esame generale
\item
  Esame della marcia
\item
  Esame del piede
\item
  Esame della calzatura
\end{itemize}

Non bisogna mai fermarsi alla semplice osservazione. Fare diagnosi vuol
dire \emph{\emph{tipizzare il piede in funzione del paziente}}. Bisogna
analizzare tutte le variabili, scovare se esiste una malattia di fondo
(il più delle volte sono neurologiche). E' importante fare un buon
\textbf{esame neurologico}, questo serve per vedere innanzitutto il tipo
di malattia ma, soprattutto, per sapere la possibile evoluzione.
Infatti, se il paziente presenta una malattia stabilizzata dal punto di
vista dello squilibrio muscolare, si può pensare ad un intervento, cosa
che risulta inutile in un paziente in cui la malattia non si è ancora
manifestata completamente. In pazienti con piede cavo bisogna sempre
chiedere una \textbf{radiografia del rachide lombo-sacrale}, infatti il
più delle volte si ritrova una schisi vertebrale oppure una qualche
\emph{malformazione del passaggio lombo-sacrale} (clinicamente si può
osservare un ciuffo di peli a livello lombare basso).

~

\emph{\textbf{Valutazione ortopedica (esame del piede)}}:

\begin{itemize}
\item
  \emph{Riducibilità del varismo del tallone}: mentre il piede piatto ha
  un tallone ruotato all' esterno, nel cavo è ruotato all' interno
  spesso a causa di ~una caduta del primo metatarsale per l' effetto
  tripode. Devo capire quindi, se questa rotazione del calcagno è dovuta
  a questo oppure è una componente della deformità.
\item
  \emph{Riducibilità della deformità}
\item
  \emph{Valutazione motilità articolare e lo stato delle articolazioni}:
  se le articolazioni sono rigide e artrosiche non conviene salvarle in
  quanto, il più delle volte, dolenti; al contrario, se esiste ancora
  una mobilità, bisognerà pensare ad interventi che interessino o le
  parti molli o che non tocchino le articolazioni evitando di ridurre
  quella che è la mobilità del piede.
\item
  \emph{Riducibilità della deformità delle dita}
\end{itemize}

\emph{\textbf{Giusto modo di visitare un piede}}: si mette la
tibio-tarsica in posizione neutra, si corregge il varismo del retropiede
con cui viene fuori tutta la caduta del primo metatarsale. Quando i
pazienti tirano su il piede la griffe del primo dito peggiora cosa che
non avviene in un piede normale. Si possono apprezzare le varie
callosità e anche l' evidenza dei tendini sottocutanei.

\emph{Come faccio a capire se il calcagno è varo perché è caduto il
primo metatarsale o perché c'è una deformità alla base?}

Esiste il test di \textbf{COLEMAN-ANDREASI} che mi permette di orientare
la terapia. Se il paziente presenta un varismo strutturato che dipende
dal calcagno, sarà necessario raddrizzare il calcagno. Se dipende dal
primo metatarsale invece, si dovrà agire su quest' ultimo. \emph{Per
effettuare questo test si prende un gradino, il paziente posiziona l'
avampiede nel vuoto escludendolo dal carico e poggia il retropiede sul
\href{http://gradino.in/}{gradino. In} questa posizione il varismo si
riduce qualora dipenda dal primo metatarsale con effetto tripode.}

\emph{\textbf{ESAME RADIOGRAFICO}} (principali proiezioni):

\begin{itemize}
\item
  TIBIO-TARSICA antero-posteriore
\item
  PIEDE DORSO-PLANTARE
\item
  PIEDE LATERALE
\end{itemize}

\emph{\textbf{Radiografia tipica}}: notiamo \emph{astragalo orizzontale,
calcagno impennato, dorso del piede aumentato, griffe delle dita}.
L'astragalo è ruotato verso l'esterno, il calcagno verso l'interno. Con
la proiezione dorso-plantare ~si può osservare il cavismo associato a
adduzione dell' avampiede e affastellamento dei metatarsali. In
proiezione laterale pura si osserva il perone molto indietro rispetto
alla tibia. Lo squilibrio tra avampiede e retropiede con conseguente
flessione dorsale della tibio- tarsica, a lungo andare determina un
impingement (reazione osteofitosica). La rotazione esterna
dell'astragalo si porta dietro tutta la gamba e quindi porta anche ad
una rotazione esterna del ginocchio (strabismo divergente delle rotule)
e dell'anca.

Sulla radiografia possono essere fatte anche delle misurazione che
comunque lasciano il tempo che trovano. Si può osservare che, anziché
avere i normali 120 gradi tra asse della tibia e asse dell' astragalo,
si arriva quasi a 90.

Per quanto riguarda la diagnostica per immagini, è molto importante a
fini diagnostici un Rx che consenta di valutare l'entità dell'angolo di
Costa-Bertani, che viene calcolato in proiezione latero-laterale
tracciando prima una linea che parte dal punto più basso della
tuberosità posteriore del calcagno ed arriva al punto più basso della
testa dell'astragalo, mentre una seconda linea va dalla testa
dell'astragalo sino al punto più basso del sesamoide interno
dell'alluce; in condizioni normali l'angolo formato da queste due linee
si aggira tra i 125° ed i 130°, mentre nel piede cavo l'angolo è
notevolmente ridotto.

TAC e RMN sono poco utili a meno che non si vogliano valutare delle
fusioni o mancate segmentazioni.

\emph{\textbf{Esami complementari}}

\includegraphics[width=3.15000in,height=2.36875in]{media/image12.jpg}\includegraphics[width=3.22014in,height=2.37431in]{media/image13.jpg}

\emph{Trattamento}

\begin{itemize}
\item
  RIABILITATIVO
\item
  ORTESICO~
\item
  CHIRURGICO
\end{itemize}

\textbf{Nel bambino}: nella maggior parte dei casi non ho ancora una
rigidità, esiste quindi una certa riducibilità.

\textbf{-Trattamento riabilitativo}:

\begin{itemize}
\item
  stretching delle parti molli (allungamento dell' aponeurosi plantare
  per far si che la sua retrazione non aumenti la volta plantare)
\item
  rieducazione alla marcia
\end{itemize}

\textbf{-Trattamento ortesico}:

\begin{itemize}
\item
  compensare il piede
\item
  riequilibrare il carico sotto le teste metatarsali con una barra
\item
  favorire la fase di spinta che viene persa con la griffe delle dita
\item
  ridurre il dolore
\end{itemize}

C'è la possibilità di un plantare che non ha la stessa funzione svolta
nel piede piatto dove si spera in una correzione. Nel piede cavo non ~ho
la correzione della deformità. Utile anche l' utilizzo di un rialzo al
tacco per riequilibrare la situazione.

\textbf{-Trattamento chirurgico}:

Nel bambino sono interventi che non toccano le articolazioni e che
interessano il meno possibile l' osso per evitare disparità durante l'
accrescimento.

\begin{itemize}
\item
  release delle parti molli
\item
  trasposizioni tendinee
\item
  tecniche sull' osso
\item
  tecniche combinate
\end{itemize}

L\emph{e tecniche chirurgiche ci interessano poco}: si può effettuare
una \textbf{\emph{fasciotomia plantare}} con possibilità di tagliare l'
aponeurosi plantare che può essere fatta anche per via percutanea (è un
intervento molto pericoloso perchè qui vicino passa l' arteria plantare
mediale), \textbf{\emph{allungamento del tendine d' Achille}} come nel
piede piatto; se c'è una deformità riducibile delle dita si può
intervenire con la \textbf{\emph{sezione dei flessori delle dita o degli
estensori}}; se c'è una caduta prevalente del primo metatarsale si può
fare una \textbf{\emph{osteotomia del primo metatarsale}} in modo da
sollevarlo; se c'è una griffe del primo dito, l'estensore dell' alluce
agisce in favore di deformità quindi bisogna portarlo ad agire in favore
di correzione. Viene quindi tagliato, suturato al pedidio e l' estensore
viene trapiantato a livello del collo del primo metatarsale in modo da
tirarlo su; se c'è una deformità strutturata del retropiede bisogna fare
una \textbf{\emph{osteotomia valgizzante}}, si prende la tuberosità
posteriore e si sposta lateralmente in modo da riequilibrare il carico
posteriore.

-\textbf{Nell'adulto}:~

-\textbf{trattamento ortesico}:

\begin{itemize}
\item
  ridurre il dolore
\item
  compensare il piede (plantari)
\item
  stabilizzare il piede
\item
  ridurre l' effetto tripode
\end{itemize}

\textbf{-trattamento chirurgico}:~

\begin{itemize}
\item
  correggere la deformità
\item
  limitare il dolore
\item
  mantenere la motilità
\item
  migliorare l' utilizzo del movimento della caviglia che è sempre in
  posizione di fine corsa.
\end{itemize}

\textbf{FATTORI DI TIPIZZAZIONE}:

\begin{itemize}
\item
  eziologia
\item
  età~
\item
  sesso~
\item
  grado di cavismo
\item
  apice della deformità
\item
  motilità
\item
  presenza di artrosi
\end{itemize}

\end{document}
