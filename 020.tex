\section{Epifisiolisi}

\subsection{Definizione}
In conseguenza di un cedimento della zona di
coniugazione tra epifisi prossimale e collo femorale, l'epifisi si distacca e scivola in basso e indietro mettendosi in rapporto con il lato mediale del collo femorale. L'anca subisce un varismo e questa condizione predispone ad un'artosi secondaria precoce.

\subsection{Epidemiologia e patogenesi}

Incidenza 2/10 \^{}5, maschi:femmine= 2,5 : 1, età maschi 10-14 anni, età femmine 10-16 anni, bilaterale nel 25-50\% dei casi

\begin{itemize}
\item
  \textbf{ipotesi meccanica}: l'obesità o il sovrappeso (spesso presenti nei soggetti affetti), un'eccessiva richiesta funzionale, un ginocchio valgo, un piede piatto sono tutte condizioni che portano all'applicazione di forze a livello del collo femorale che ne inducono un gioco di leve in grado di dissecare l'epifisi in accrescimento
\item
  \textbf{ipotesi distrofico-metabolica}: in soggetti che hanno una lassità legamentosa costitutiva o genetica presentano epifisiolisi in conseguenza dei maggiori movimenti permessi all'articolazione lassa dell'anca in accrescimento.
\end{itemize}

\subsection{Clinica}

\begin{itemize}
\item
  \textbf{Quadro di epifisiolisi cronica (85\% dei casi)}: le alterazioni si instaurano gradualmente nel corso di settimane durante le quali il giovane paziente accusa dolore localizzato all'inguine o alla coscia/ginocchio mediale che recede con il riposo, c'è limitazione alla intrarotazione e all'abduzione che nelle fasi più avanzate saranno compromesse completamente. Ne risulta una tipica andatura con zoppia. Alla radiografia si nota un aumento dello spessore della linea di congiunzione cervico- epifisaria i cui margini sono irregolari e la struttura dell'osso sottostante presenta un'alternarsi di zone opache e trasparenti indicative del fenomeno ad andamento cronico (sono espressione di precedenti episodi di micro epifisiolisi).Negli stadi più avanzati il collo femorale è scoperto nella sua porzione supero-mediale mentre l'epifisi prossimale è a contatto con la zona inferiore e mediale del collo.Il trattamento può essere conservativo nelle fasi iniziali e consiste nell'applicazione di viti e chiodi transtrocanteriche che, dopo aver raggiunto una riduzione dell'epifisi sul collo femorale, fissano le strutture diastatizzate e ne permettono una stabilità morfologica e funzionale (vedi immagine).
\item
  \textbf{Quadro di epifisolisi acuta o epifisiolisi acuta su cronica(15\% dei casi)}: la presetnazione è uguale alle fratture cervico-epifisiaria e comprende un dolore acuto all'inguine o alla coscia mediale associato ad una grave impotenza funzionale e alla presentazione dell'arto come accorciato addotto e extraruotato. Il trattamento deve essere precoce e si compone ancora una volta di riduzione e fissazione dell'epifisi al collo del femore con viti e chiodi.
\end{itemize}

\begin{figure}[!ht]
\centering
\includegraphics[width=0.4\textwidth]{020/image1.jpeg}
\end{figure}

\section{Osteocondrosi}

\subsection{Definizione}
è un insieme di patologie caratterizzato dalla contemporanea presenza di necrosi ischemica dell'osso, di una degenerazione della cartilagine limitrofa che esitano in microfratture ripetute che possono localizzarsi nelle seguenti strutture:

\begin{itemize}
\item
  nucleo epifisiario: epifisi prossimale del femore (malattia di Perthes), epifisi metatarsale (generalmente testa del secondo)
\item
  osso breve : scafoide tarsico;
\item
  superficie convessa di una articolazione cartilaginea ( osteocondrite dissecante): condilo femorale mediale, testa del femore, condilo omerale, troclea astragalica.
\end{itemize}

\subsection{Patogenesi}

Generalmente colpisce soggetti giovani in accrescimento, come l'epifisiolisi. La zona anatomica interessata dalla necrsosi ischemica è una zona la cui vascolarizzazione è esposta a delle limitazioni anatomiche in quanto l'osso in accrescimento è circondato da cartilagine jalina perforata da microvasi. Qualunque alterazione della cartilagine (a base genetica, familiare o aquisita da traumi ecc) può portare alla obliterazione di questi vasi e questo innesca un circolo vizioso vascolare-meccanico che mantiene il danno cartilagineo e osseo e lo amplifica fino a coinvolgere l'intera zona ed esita in un distacco
patologico.

\subsection{Patologie}

\subsubsection{Malattia di Perthes} 
incidenza 1,5/10\^{}5 aa, maschi.femmine=
5:1, età 3-14, 90\% monolaterale.

Come ogni affezione dolorosa di questa zona il dolore è riferito all'inguine o alla faccia interna della coscia, c'è limitazione funzionale in particolare dei movimenti abduttori e rotativi, la marcia presenta zoppia.

Radiograficamente si distinguono 4 gradi di gravità che vedono un progressivo peggioramento del capo articolare interessato il quale inizialmente presenta una semplice irregolarità del contorno e della trabecolatura ossea , quando inizia a comparire la necrosi sia ha un aumento dell'opacità e uno schiacciamento del profilo del nucleo
epifisario, ancora dopo insorgono fenomeni di rimaneggiamento osseo e di riparazione che portano ad un profilo totalmente irregolare di entrambi i capi articolari, configurandosi dunque una coxartrosi precoce.

Terapia: nelle fasi iniziali ci si può permettere l'applicazione di misure conservative quali apparecchi gessati o tutori utilizzati per lunghi periodi che, associati a un controllo del peso corporeo, stabilizzano la situazione e ne prevengono un' ulteriore involuzione.
Nei gradi più avanzati il trattamento è chirurgico e si eseguono interventi volti a migliorare la congruenza fra acetabolo e testa del femore o a mettere in scarico i punti di contatto articolari con un'osteotomia e una rotazione di 180\textsuperscript{o} del collo femorale con l'obiettivo di limitare la necrosi epifisiaria.

\subsubsection{Apofisiolisi di Osgood-Schalatter} 
(vale anche per apofisi calcaneale e apofisi della tuberosità ischiatica) : si tratta di un distacco e una degenerazione della cartilagine nel punto di inserzione
dei tendini del muscolo quadricipite femorale nella tibia. Durante l'accrescimento infatti coesistono due situazioni disarmoniche: le masse muscolari si accrescono velocemente e acquisiscono sempre più potenza mentre la cartilagine su cui esercitano trazioni i loro tendini
d'inserzione è ancora in via di sviluppo e consolidazione, sostazialmente incapace di sopportare una trazione violenta. Questi distacchi parcellari sono classificabili come fratture da fatica e guariscono con il riposo. La forte sintomatologia dolorosa locale,
esacerbata dall'azione del muscolo e migliorata dal riposo, trae beneficio da tecniche fisiche locamente applicate, come vedremo nelle lezioni di fisiatria.

\subsubsection{Osteocondrosi di metatarso intermedio e di scafoide tarsico}
sono cause di dolore al piede in età giovanile, specialmente nei maschi, specialmente in concomitanza di attività fisica intensa.

Le radiografie hanno caratteristiche anatomo-patologiche simili a quelle identificate per la malattia di Perthes e il trattamento d'elezione è lo scarico funzionale della regione. Solo in rari casi dove l'anatomia delle ossa e delle articolazioni è molto compromessa e la sintomatologia dolorosa non regredisce e dunque si accede alla chirurgia che riassesta l'assetto articolare con osteotomie e/o artrodesi.

\subsubsection{Osteocondrosi dissecante}
Come prima accennato sono patologie delle superfici convesse delle articolazioni dotate di cartilagine
jalina e sottoposte a traumi tangenziali ripetuti (attività sportive, cadute e tutte le altre attività potenzialmente traumatizzanti svolte dai bambini).

Il segmento interessato dalla patologia ischemico-necrotizzante fa si che porzioni di superficie articolare si distacchino e migrino all'interno della cavità articolare, nel tempo si accrescono perchè nutriti per diffusione dal liquido sinoviale e danno segno di se con
dolore e tumefazione articolare. La terapia dipende dalla gravità del quadro clinico e, quando necessaria, la chirurgia è impiegata per fissare l'elemento vagante con viti, dopo aver cruentato il letto accogliente sul lato articolare per fare in modo che si crei un callo osseo in grado di ristabilire la continuità con il segmento stesso.

\section{Metatarsalgie}

METATARSALGIA (qualunque dolore che interessa il metatarso, cioè la parte anteriore del piede, sia plantare che dorsale).

Seconda causa di sofferenza (dopo mal di testa) del corpo umano.

Metatarso è una parte del piede costituita dalle ossa metatarsali (1\textsuperscript{o},
2\textsuperscript{o}, 3\textsuperscript{o}, 4\textsuperscript{o}, 5\textsuperscript{o} osso metatarsale).

\emph{Classificazione:}
\begin{itemize}
\item \textbf{METATARSALGIE NON MECCANICHE}
\begin{itemize}
\item
  osteocondrosi
\item
  artrosi
\item
  infezioni
\item
  neoplasie
\item
  malattie vascolari
\item
  malattie neurologiche (sindrome tunnel tarsale)
\item
  malattie metaboliche (gotta, diabete)
\item
  malattie dermatologiche (micosi, psoriasi)
\end{itemize}
\item \textbf{METATARSALGIE MECCANICHE} (alterazioni della struttura)
\end{itemize}

\subsection{Metatarsalgie non meccaniche}

\subsubsection{Esiti di osteocondrosi}

Esempio necrosi seconda testa metatarsale da bambini, solitamente in fase acuta non si fa diagnosi, il dolore si autolimita, ma la testa invece di essere rotonda è schiacciata.

Nell'adulto tipico dolore nella fase di spinta (nella flessione dorsale del piede).

Terapia: rimodellare la testa metatarsale, artroplastica (si tolgono gli osteofiti e si cerca di dare un minimo di rotondità alla testa); solitamente c'è un frammentino mobile a livello dorsale, che è quello che crea dolore durante la flessione dorsale.

\subsubsection{Artrite reumatoide (o psoriasica) in fase acuta}

Artrite reumatoide colpisce la membrana sinoviale, per cui interessa soprattutto piede e mano che sono le strutture maggiormente sinovializzate (non è infrequente che un'artrite reumatoide inizi con una meta tarsalgia).

Quando il paziente è in piedi e le dita sono allargate e soprattutto non toccano, senza deformità: iperplasia della membrana sinoviale, no perché c'è liquido.

\subsubsection{Infezioni}

Es causate da ferite da punta, o comunque infezioni croniche con il tipico granuloma.

Trattamento: fere in modo che tutto il materiale necrotico e il pus esca e soprattutto che la ferita non si chiuda in un tempo troppo breve (per cui drenaggi) in modo che la ferita si spurghi e che gli antibiotici possano agire.

\subsubsection{Neoplasie}

Es fibrosarcoma (tumore maligno)

Neuroma di Morton (tumore benigno): tumore (non si è capito se è un fibroma, un neuro fibroma,..) che si sviluppa a livello plantare tra la terza e quarta testa del nervo digitale comune, prima che si biforchi nei due digitali propri. Il nervo digitale comune in questo modo viene in contatto con il legamento intermetatarsale (tra le due teste) e sfregando si crea un tipica sintomatologia che va dal dolore, al bruciore, alla sensazione di scossa elettrica che interessa i margini
contrapposti del terzo e quarto dito.

Mulder test: si sente un click tattile a volte anche uditivo quando manualmente stringiamo le teste metatarsali, perche il neuroma, come un piccolo pallino schizza fuori e in più si riproduce la tipica sintomatologia.

Anatomia patologica del neuroma è varia, forse è dovuto a un problema meccanico, cioè una condizione di sovraccarico sui raggi esterni, come in caso di alluce valgo (per insufficienza del 1\textsuperscript{o}raggio c'è uno spostamento del carico sui raggi esterni) e alluce rigido.

Non c'è modo di fare diagnosi (no ecografia, no risonanza)

La sintomatologia è indipendente dalla grandezza del neuroma.

Trattamento chirurgico: accesso dal di sopra (no dalla pianta), si isolano il nervo digitale comune e nervi digitali propri, e si toglie il neuroma, cioè si tolgono i nervi per cui si passa da una condizione di
ipersensibilità a una di ipo-anestesia dopo l'intervento (è stato dimostrato poi che a distanza di anni la sensibilità torna perché le terminazioni nervose tornano a innervare le zone).

\subsubsection{Sindrome del tunnel tarsale}

Nervo tibiale posteriore e arteria tibiale posteriore passano dietro al malleolo interno, nel canale del tarso (simile al canale del carpo). Per cui se c'è una compressione del canale (esempio per una cisti),
compressione del nervo e quindi meta tarsalgia. Di fatto la sindrome del tunnel tarsale non è mai ascrivibile, come anatomia patologica, alla compressione del mediano al carpo (che si verifica perche le guaine dei tendini si gonfiano, per una cisti, per esiti di frattura). In realtà nella sindrome del tunnel tarsale non c'è una compressione, non c'è una sproporzione tra contenente e contenuto come per il tunnel carpale, ma è una sindrome da stiramento (come quando si ha un valgo di retro-piede,
eccesso di pronazione, quindi rotazione esterna del calcagno, tutte le strutture interne vengono sottoposte ad una trazione eccessiva). Per cui non bisogna trattare la sindrome del tunnel tarsale come quella del tunnel carpale, perche avremmo un risultato negativo considerando che non bisogna decomprimere ma raddrizzare il retro-piede.

Si può avere anche un intrappolamento, contusione, stiramento dei nervi anteriori, es. nervo peroneo superficiale (solo sensitivo); in questo caso si valuta il segno di Tinel. Intrappolamento si può verificare in
persone che dimagriscono rapidamente: il nervo all'uscita della fascia rimane appunto intrappolato.

\subsubsection{Patologie sovrasegmentate}

Ernia del disco con interessamento ad esempio solo del nervo sciatico popliteo interno, per cui meta tarsalgia. Oppure lo sciatico popliteo esterno, che passa a livello del collo prossimale del perone: una qualunque patologia della sindesmosi (articolazione) tibia-perone prossimale, come una cisti, può andare a comprimere il nervo per cui
dolore alla dorsale del piede.

\subsubsection{Ischemia}

Ad esempio nei diabetici, nei fumatori forti, può provocare la classica sindrome del dito blu, il sangue arriva poco e c'è stasi per cui deossigenazione. Arteriografia.

\subsubsection{Ipotrofia del pannicolo adiposo plantare}
Meta tarsalgia ``meccanica'', anche se non è sostenuta da cause meccaniche, riduzione dello spessore del pannicolo adiposo. Pelle lassa.

\subsubsection{Cicatrici plantari}
Meta tarsalgia non meccanica: ad esempio se qualche chirurgo ha tolto il neuroma di Morton per via plantare, si crea una cicatrice nella zona di carico con ipertrofia.

\subsubsection{Cheratoma plantare}
Tumore benigno, si crea nelle zone di sovraccarico; ipercheratosi, callo con metaplasia. Asportazione chirurgica.

\subsubsection{Verruche plantari e ragadi plantari}
(spaccamento della pelle).


\subsection{Metatarsalgie meccaniche}

Determinate da una alterazione della meccanica dell'avampiede per:
\begin{itemize}
\item pressione più elevata a livello di una testa metatarsale
\item pressione normale applicata per un periodo di tempo maggiore (come nell'eccesso di pronazione / supinazione)
\end{itemize}

classificazione:

\begin{itemize}
\item
  sovraccarico globale dell'avampiede
\item
  distribuzione irregolare del carico metatarsale
\end{itemize}

Sono caratterizzare tutte dal CALLO, perché sia che ci sia un aumento del carico, quantitativo o temporale, la pelle rimane schiacciata, si crea un'ischemia e reagisce aumentando lo spessore dello strato corneo, per cui il callo è una reazione al sovraccarico. Per cui per togliere i calli si deve intervenire chirurgicamente o con dei plantari per riequilibrare la distribuzione del carico.

N.B.: NON E' ASSOLUTAMENTE VERO CHE IL PESO APPOGGIA SOLO SUL 1\textsuperscript{o} E SUL 5\textsuperscript{o} DITO, TUTTE LE TESTE METATARSALI DEVONO APPOGGIARE IN MODO UGUALE (SE UNO HA UN ARCO TRASVERSO, CIOE' 1\textsuperscript{o} E 5\textsuperscript{o} E' PATOLOGIA; FISIOLOGICAMENTE ESISTE SOLO L'ARCO LONGITUDINALE)

\begin{itemize}
\item
  \textbf{\emph{sovraccarico globale dell'avampiede}}: tutte le teste metatarsali soffrono. Cause:
\begin{itemize}
\item
  calzature con tacco eccessivo: per ogni cm in altezza del tacco il peso si sposta di una certa \% a livello della zona metatarsale (tacco fisiologico 3-4 cm).
\item
  piede equino rigido, per mancata dorsi-flessione del piede, trattabile con un allungamento del tendine di achille.
\item
  piede cavo, slivellamento tra avampiede e retro piede, con sovraccarico dell'avampiede e del tallone anteriore inteso come zona di appoggio anteriore metatarsale.
\end{itemize}
\item
  \textbf{\emph{distribuzione irregolare del carico metatarsale}}: \textbf{insufficienza del 1\textsuperscript{o} raggio} (tra cui alluce valgo), che porta al sovraccarico dei raggi esterni. Alluce valgo: deviazione laterale 1\textsuperscript{o} dito sostenuto da un allargamento del ventaglio metatarsale (metatarso varo), aumento dell'angolo intermetatarsale (tra 1\textsuperscript{o} e 2\textsuperscript{o} metatarso); le cause sono numerose, ricordiamo:
\begin{itemize}
\item
  cause biomeccaniche (più frequenti) come durante la gravidanza e dopo la menopausa;
\item
  predisposizione genetica (a seconda dell'anatomia abbiamo diversi tipi di piede, tutti normali: 
\begin{itemize}
\item formula digitale: 
\begin{itemize}
\item piede egizio - primo dito più lungo del secondo,
\item piede greco - primo dito più corto del secondo,
\item piede quadrato - primo e secondo dito uguali;
\end{itemize}
\item formula metatarsale: 
\begin{itemize}
\item index plus-minus - primo e secondo metatarsali sono
uguali
\item index minus - primo metatarsale più corto del secondo
\item index plus - primo metatarsale più lungo del secondo);
\end{itemize}
Tra queste formule digitali e metatarsali la condizione sfavorevole per la patogenesi dell'alluce valgo è l'associazione piede egizio + index minus.
\end{itemize}
\item pronazione del retro piede (es nel piede piatto).
\end{itemize}

Tutte le altre, come i tacchi alti sono concause, no cause di alluce valgo.

Quello che succede a livello del 1\textsuperscript{o} raggio, può succedere anche a livello del 4\textsuperscript{o} e 5\textsuperscript{o} raggio, con un allargamento dell'angolo tra quarto e quinto,
per cui c'è un allargamento del primo e del quinto con formazione dell'avampiede triangolare, associato a deformità delle dita esterne.

Una volta che c'è insufficienza del 1\textsuperscript{o} raggio si verificherà sovraccarico, per trasferimento del 2\textsuperscript{o} e 3\textsuperscript{o} raggio, provocando meta tarsalgia con callosità + squilibrio dei muscoli della gamba e del piede con azione sulle dita con formazione quindi del dito ad artiglio che porta al peggioramento della meta tarsalgia e callosità delle dita a seguito dell'attrito con la calzatura.

Trattamento: osteotomia con il significato di riavvicinare 1\textsuperscript{o} metatarsale al 2\textsuperscript{o} e riportate i tendini lungo il loro asse di movimento + osteotomia del calcagno (se c'è pronazione del retro piede) per riallineare il carico a livello del retro piede ed evitare le recidive + eventuale correzione delle dita esterne.

\item sovraccarico dei raggi centrali
che non dipende da un'insufficienza del 1\textsuperscript{o}, cause:
\begin{itemize}
\item
  anomalie di lunghezza: index minus. Trattamento dell'ipermetria:osteotomia di accorciamento.
\item
  anomalie di inclinazione: fratture. Meta tarsalgie post-traumatiche.
\item
  rigidità a livello ad esempio del 2\textsuperscript{o} metatarsale, nell'alluce rigido da artrosi della prima metatarso-falangea: quando cammino ho bisogno almeno di 65\textsuperscript{o}-75\textsuperscript{o} di dorso flessione dell'alluce quando stacco il tallone, se tale movimento è dolente a un certo punto si verifica la rotazione all'interno del piede con il trasferimento di carico, quindi rotazione interna dell'avampiede, diminuzione della spinta e meta tarsalgia da trasferimento.
\end{itemize}
Trattamento: osteotomia.

\item sovraccarico 1\textsuperscript{o} raggio

tipico caso del piede cavo antero - interno, che può provocare la rottura del sesamoide trattamento: osteotomia base 1\textsuperscript{o} metatarsale +
sollevamento 1\textsuperscript{o} metatarsale + fasciotomia plantare

\item insufficienza raggi centrali

arco trasverso (situazione patologica), callosità sotto il 1\textsuperscript{o} e sotto il 5\textsuperscript{o}.

Si verifica nell'effetto tripode (piede cavo antero-interno) in cui si verifica un sovraccarico del 1\textsuperscript{o}, per caduta del 1\textsuperscript{o} + rotazione interna del piede, con varismo del retro piede + sovraccarico del 5\textsuperscript{o}. Può essere
anche cause congenite (2 brachi-metatarsie) oppure per cause iatrogene (eccessivo accorciamento metatarsali centrali).

\item sovraccarico 5\textsuperscript{o} raggio

cause:
\begin{itemize}
\item griffe congenita del 5\textsuperscript{o} dito
\item 5\textsuperscript{o} dito varo, congenita o nell'ambito dell'avampiede triangolare
\item 5\textsuperscript{o} dito addotto, congenita mono o bi-laterale, è un dito schiacciato, nel bambino si fanno dei cerottaggi, mentre nell'adulto si ricorre alla
chirurgia
\end{itemize}
\end{itemize}


\section{Algie di origine discale}

\subsection{Definizione} 
Sono dolori neuropatici conseguenti la compressione da parte del nucleo polposo erniato (o del disco intervertebrale degenerato protrusi posteriormente) delle radici nervose che impegnano i forami di coniugazione intervertebrali o che percorrono il canale midollare. Può essere coinvolto direttamente il midollo spinale.

\subsection{Patogenesi}

Con l'invecchiamento, il carico e i frequenti movimenti tutte le articolazioni tra le vertebre subiscono una degenerazione da usura. In particolare il disco fibroso dell'articolazione intersomatica subisce una degenerazione delle fibre collagene che compongono la circonferenza dell'anulus e una disidratazione del nucleo polposo ( questo normalmente è molto ricco di acqua, fattore che gli permette di subire delle deformazioni senza una modificazione di volume, ed agisce dunque come
perno intervertebrale durante i movimenti di flesso-estensione e di lateralità).

L'algia è dunque una tra le conseguenze dell'involuzione biomeccanica delle articolazioni intervertebrali.

\begin{figure}[!ht]
\centering
\includegraphics[width=0.4\textwidth]{020/image2.png}
\end{figure}

\begin{figure}[!ht]
\centering
\includegraphics[width=0.4\textwidth]{020/image3.png}
\end{figure}

\subsection{Ernia cervicale}

Il processo patogenetico prima descritto si traduce in questo segmento del rachide con una riduzione della fisiologica lordosi, con una instabilità segmentaria e con la presenza endomidollare o interforaminale di ernie dure ossia formazioni composte non di nucleo polposo molle ma da materiale osseo che si crea quando i corpi
vertebrali hanno attrito reciproco -\/-\textgreater{}
\textbf{sindesmofiti e osteofiti interapofisari (ernia dura).}

\subsubsection{Quadri clinici e relative terapie}

\begin{itemize}
\item
  \textbf{Cervicalgia}: il dolore è limitato al collo nella sua faccia posteriore fino alla nuca ed è esacerbato dalla palpazione dei processi trasversi, si associa spesso a vizi posturali che hanno un significato antalgico. Il trattamento è sostanzialmente composto da FANS , miorilassanti, massoterapia, applicazione locale di calore, programma fisioterapeutico.
\item
  \textbf{Cervico-brachialgia:} oltre al dolore al collo c'è un dolore irradiato all'arto superiore in quanto il processo degenerativo delle articolazione intervertebrali ha portato all'interessamento delle radici nervose con conseguente alterazione della sensibilità, della forza, e dei riflessi profondi regionali.E' dunque indispensabile eseguire un controllo neurologico dell'arto superiore e identificare la radice probabilmente coinvolta ricordando che alcuni riflessi indagano prevalentemente alcuni mielomeri ( bicipitale = C5, brachioradiale = C6 , tricipitale = C7).Il trattamento può avvalersi di un uso combinato di cortisonici ad alte dosi per pochi giorni e analgesici per controllare la fase irritativa della radiculopatia, si può prescrivere pure un collare morbido che applichi una decompressione delle radici nervose stesse.Nei casi gravi si può ricorrere all'intervento chirurgico decompressivo che, con l'accesso chirurgico anteriore o posteriore, permette una discectomia o una laminoplastica.
\end{itemize}

\begin{figure}[!ht]
\centering
\includegraphics[width=0.4\textwidth]{020/image4.png}
\end{figure}

\begin{itemize}
\item
  \textbf{Mielopatia spondilosica e sindromi neuro vascolari:} sono quadri molto gravi che necessitano di un rapido inquadramento clinico e di un altrettanto rapida soluzione chirurgica. Sono conseguenti alla protrusione dentro il canale midollare di un importante volume di materia che riesce a comprimere il sacco durale e dunque il midollo e i vasi sanguigni corrispondenti. Si presentano con paraparesi spastica, iperreflessia e turbe della sensibilità termico dolorifica (fascio spinotalamico). Le sindromi neurovascolari sono talvolta attribuibili a disordini psicoemotivi del paziente piuttosto che all'interessamento dei nervi simpatici; sono invece frequenti i drops attacks riferibili a disordini circolatori nel territorio di distribuzione delle arterie vertebrali. La terapia chirurgica è quella prima esposta per le forme gravi di cervico-brachialgia.
\end{itemize}

In tutte queste forme la diagnosi si basa sulla clinica e sull'imaging:
radiografie in posizione statica anteroposteriore e latero-laterale, in
massima flessione e in massima estensione per valutare il grado di
instabilità articolare. Per la corretta interpretazione del dolore
neuropatico ci si avvale di indagini neurologiche quali
l'elettromiografia e i potenziali evocati sensitivo-somatici. Tutto
questo vale anche per la prossima sezione.

\begin{figure}[!ht]
\centering
\includegraphics[width=0.4\textwidth]{020/image5.png}
\end{figure}

\subsection{Ernia lombare}

In questa sezione del rachide si concentrano due fattori favorenti la degenerazione discale: l'ampia articolarità e la sopportazione di tutto il peso del tronco e degli arti superiori.

Molto spesso la sintomatologia è dipendente all'erniazione del nucleo polposo la quale può essere caratterizzata per tragitto compiuto (contenuta se il nucleo è rimasto al di sotto del legamento longitudinale posteriore, espulsa se ha superato il legamento ma ha
mantenuto una continuità con il nucleo discale, migrata se se ne è completamente distaccata ) o per posizione raggiunta (posteriore, postero-laterale, intra ed extra foraminale). Ad ogni tipologia di ernia corrisponde una presentazione clinica conseguente all'interessamento delle strutture nervose incontrate, il quadro può presentarsi acutamente o raggiungere l'acme in giorni-settimane.

\begin{figure}[!ht]
\centering
\includegraphics[width=0.4\textwidth]{020/image6.png}
\end{figure}

\begin{figure}[!ht]
\centering
\includegraphics[width=0.4\textwidth]{020/image7.png}
\end{figure}

\begin{figure}[!ht]
\centering
\includegraphics[width=0.4\textwidth]{020/image8.png}
\end{figure}

\begin{itemize}
\item
  \textbf{Lombocruralgia e lombosciatalgia}: sono caratterizzate da dolore in sede lombare con irradiazione o alla faccia anteriore della coscia o alla faccia posteriore di tutto l'arto inferiore rispettivamente. Come per la cervicobrachialgia è molto importarte per fini terapeutici e prognostici caratterizzare il danno neurologico conseguente lo schiacciamento radicolare che evolte secondo le tre classiche fasi di IRRITAZIONE (clinicamente manifestano dolore), DEFICIT (sia il versante motorio che quello sensitivo non sono più funzionali e si analizzano investigando la sensibilità tattico-termico- dolorifica-pallestesica e la mobilità dei gruppi muscolari dei territori innervati da tale radice, si devono inoltre investigare i riflessi osteo-tendinei per avere indicazione di sede), PARALITICA (completo deficit sensitivo-motorio e atrofia muscolare da denervazione + deformità e non funzionalità segmetaria). Clinicamente si valutano pure la dolorabilità alla palpazione del decorso dei singoli nervi interessati, ossia il femorale per la lombocruralgia e lo sciatico-popliteo per la lombosciatalgia, e la dolorabilità evocata durante particolari manovre semeiologiche : manovra di Wasserman (paziente prono , dolore all'estensione dell'anca con ginocchio flesso per stiramento di L4), manovra di Lasegue (paziente supino, dolore alla flessione dorsale del piede a ginocchio esteso e anca sollevata a 45\textsuperscript{o}).La terapia in questi casi si avvale di farmaci analgesici e antiinfiammatori non steroidei per i quadri lievi e di cortisonici per i quadri deficitari, associati alla prescrizione di un busto semirigido che trazioni il rachide e ad un regime di riposo con divieto assoluto di sollevare carichi che andrebbero ad aggravare l'erniazione del nucleo polposo. I quadri inveterati o particolarmente gravi sono indicazioni all'esecuzione di interventi decompressivi della cavità midollare o dei forami di coniugazione: discectomia.
  Questi interventi sono gravati da complicazioni serie quali l'instabilità del segmento trattato con conseguente lombalgia cronica.
\item
  \textbf{Sciatica paralitica e sindrome della cauda equina}: sono quadri molto gravi che si verificano più facilmente in soggetti che costituzionalmente hanno già una stenosi del canale vertebrale lombare e dunque sono più soggetti allo sviluppo di dolore neuropatico da compressione.
\end{itemize}

Si caratterizzano per una ipoanesetsia "a sella" che dalla zona perianale si distribuisce alle zone mediali limitrofe, per turbe sfinteriche e per paraparesi flaccide. Loro stesse sono indicazioni all'esecuzione dell'intervento chirurgico che non sempre è risoluitivo.

\subsection{Diagnosi e diagnosi differenziale}

le tecniche di imaging risescono a caratterizzare l'enria e a visualizzare la radice nervosa compressa. Facendo questo riescono ad escludere altre possibili cause di dolore neuropatico tra le quali:
tumori ossei e nervosi, cisti midollari, tumori retroperitoneali, malattie infettive.

Particolare attenzione va rivolta a situazioni cliniche che possono presentarsi con un dolore simile : nefrolitiasi, arteriopatia periferica con claudicatio intermittens, complicazioni ovarico-annessiali,
coxartrosi e borsiti intertrocanteriche.
