\documentclass[]{article}
\usepackage{lmodern}
\usepackage{amssymb,amsmath}
\usepackage{ifxetex,ifluatex}
\usepackage{fixltx2e} % provides \textsubscript
\ifnum 0\ifxetex 1\fi\ifluatex 1\fi=0 % if pdftex
  \usepackage[T1]{fontenc}
  \usepackage[utf8]{inputenc}
\else % if luatex or xelatex
  \ifxetex
    \usepackage{mathspec}
  \else
    \usepackage{fontspec}
  \fi
  \defaultfontfeatures{Ligatures=TeX,Scale=MatchLowercase}
\fi
% use upquote if available, for straight quotes in verbatim environments
\IfFileExists{upquote.sty}{\usepackage{upquote}}{}
% use microtype if available
\IfFileExists{microtype.sty}{%
\usepackage{microtype}
\UseMicrotypeSet[protrusion]{basicmath} % disable protrusion for tt fonts
}{}
\usepackage[unicode=true]{hyperref}
\hypersetup{
            pdfborder={0 0 0},
            breaklinks=true}
\urlstyle{same}  % don't use monospace font for urls
\usepackage{graphicx,grffile}
\makeatletter
\def\maxwidth{\ifdim\Gin@nat@width>\linewidth\linewidth\else\Gin@nat@width\fi}
\def\maxheight{\ifdim\Gin@nat@height>\textheight\textheight\else\Gin@nat@height\fi}
\makeatother
% Scale images if necessary, so that they will not overflow the page
% margins by default, and it is still possible to overwrite the defaults
% using explicit options in \includegraphics[width, height, ...]{}
\setkeys{Gin}{width=\maxwidth,height=\maxheight,keepaspectratio}
\IfFileExists{parskip.sty}{%
\usepackage{parskip}
}{% else
\setlength{\parindent}{0pt}
\setlength{\parskip}{6pt plus 2pt minus 1pt}
}
\setlength{\emergencystretch}{3em}  % prevent overfull lines
\providecommand{\tightlist}{%
  \setlength{\itemsep}{0pt}\setlength{\parskip}{0pt}}
\setcounter{secnumdepth}{0}
% Redefines (sub)paragraphs to behave more like sections
\ifx\paragraph\undefined\else
\let\oldparagraph\paragraph
\renewcommand{\paragraph}[1]{\oldparagraph{#1}\mbox{}}
\fi
\ifx\subparagraph\undefined\else
\let\oldsubparagraph\subparagraph
\renewcommand{\subparagraph}[1]{\oldsubparagraph{#1}\mbox{}}
\fi

% set default figure placement to htbp
\makeatletter
\def\fps@figure{htbp}
\makeatother


\date{}

\begin{document}

\emph{Scoliosi (trattamento)}

\emph{La parte più consistente dell'epidemiologia è rappresentata dalle
scoliosi infantili idiopatiche che sono le più frequenti e sono quelle
che traggono più in inganno perché sono moltissime, ma in realtà quelle
che vanno curate sono pochissime. Il problema quindi è riconoscere
quelle che richiedono effettivamente un trattamento.}

Vi ricordo che c'è un rapporto tra peggioramento attivo e età
scheletrica: una bambina di 14-15 anni già sviluppata da due anni,
probabilmente, non dovendo più crescere, non ha più la possibilità di
avere un peggioramento di tipo attivo.

Nella lezione precedente abbiamo visto anche il peggioramento passivo
che però è percentualmente molto inferiore rispetto a quello che è il
peggioramento attivo durante il periodo dell'accrescimento.

\includegraphics[width=4.11458in,height=2.32292in]{media/image1.jpeg}

Vi ricordo che non esiste prevenzione per la scoliosi \emph{e questo
confuta tutto quanto comunemente si crede circa la possibilità di
prevenire il peggioramento della stessa con la ginnastica correttiva, il
nuoto, dormendo su un letto rigido oppure con terapie come
l'elettrostimolazione della muscolatura della convessità della curva.}

Il trattamento dello scoliosi può essere così schematizzato:

\begin{itemize}
\item
  \textbf{Diagnosi precoce}
\item
  \textbf{Controllo evolutivo}
\item
  \textbf{Trattamento ortopedico}
\item
  \textbf{Trattamento chirurgico (eventualmente)}
\end{itemize}

\begin{itemize}
\item
  \textbf{DIAGNOSI PREOCE}
\end{itemize}

\begin{quote}
\emph{Questa è l'unica terapia utile! Non significa solo individuare la
curva e riconoscere sia una scoliosi vera o organica (questa fase
potrebbe essere anche relativamente agevole) piuttosto
\textbf{\emph{capire quella scoliosi}} e \textbf{\emph{valutare se
necessiti di un trattamento o meno}}. Possono infatti presentarsi due
evenienze:}
\end{quote}

\begin{itemize}
\item
  \emph{La \emph{gravità} della scoliosi è \emph{modesta} per cui sarà
  fondamentale il \emph{controllo evolutivo} in relazione
  all'accrescimento. Questo sarà realizzato tramite controlli periodici
  e solo \emph{eventualmente} si ricorrerà alla terapia ortopedica e/o
  chirurgica}
\item
  \emph{La \emph{scoliosi} si presenta \emph{già in una forma severa}
  \emph{o} comunque \emph{significativa} così da imporre fin da subito
  la necessità di un trattamento ortopedico e/o chirurgico \emph{anche
  se lo sviluppo e la fase di accrescimento non si sono ancora
  concluse}}
\end{itemize}

\begin{itemize}
\item
  \textbf{CONTROLLO EVOLUTIVO}
\end{itemize}

\begin{quote}
\emph{Durante una visita ortopedica si valutano i fattori di
tipizzazione all'Rx} (\emph{vedere sbobina precedente sulle scoliosi per
i fattori di tipizzazione}). \emph{Se il bambino ha una
\textbf{\emph{lieve scoliosi}} ed è \textbf{\emph{in età di
accrescimento}}, si fa un controllo dopo 6-8 mesi per stimare il
rapporto tra eventuale accrescimento (si misura l'altezza) e
peggioramento (si chiede una radiografia). Se la scoliosi è rimasta
invariata allora la bambina o il bambino tornano a casa e continuano la
loro vita normale. \emph{Si procederà con questi controlli periodici
fino al termine della fase di accrescimento che è il periodo critico}.
Il prof parla di \textbf{Energica Terapia di Attesa} (IMPARARE QUESTA
ESPRESSIONE!).} Se invece la scoliosi è importante già alla prima
visita, si procede con il trattamento ortopedico e/o chirurgico anche se
lo sviluppo non è completato.

Il concetto fondamentale \textbf{\emph{non è quello di curare le
scoliosi tutte e subito}}, ma di \textbf{\emph{capirle}}: dobbiamo
capire se è
\end{quote}

\begin{itemize}
\item
  Una di quelle poche che \emph{così sono e così rimarranno} e che sono
  di circa 8 gradi in un adulto (curva non perfetta, ma normale)
\item
  Una \textbf{forma evolutiva} che nella fase successiva \emph{andrà
  incontro a peggioramento.} In questo ultimo caso bisogna cercare di
  \emph{\emph{bloccare o limitare il peggioramento sfruttando se
  possibile l'accrescimento e quindi la possibilità di un miglioramento.
  }}
\end{itemize}

\begin{quote}
Il primo obiettivo quindi non è mai quello di raddrizzare la colonna
bensì di \textbf{bloccare un eventuale peggioramento}: su 100 scoliosi
che vediamo in 3-4 c'è rischio di peggioramento e quindi la necessità di
mettere dei busti o fare trattamenti. Invece le altre si accompagnano e
seguono fino alla fine del periodo ``pericoloso'' dell'accrescimento
attraverso controlli periodici e ne residueranno delle scoliosi sì, ma
modeste e assolutamente compatibili con una buona qualità di vita.
{[}\emph{Il prof si sofferma particolarmente su questo aspetto anche in
relazione alle conseguenze negative cui il paziente andrà in corso in
caso di errata interpretazione della forma di scoliosi. Inoltre ricorda
che il danno medico non consiste solo nel danno organico, ma c'è anche
un danno di tipo \emph{diagnostico-evolutivo} nel momento in cui non si
comprende quel tipo di scoliosi. {]}}

Quindi si curano solo quelle peggiorano? \textbf{\emph{In parte sì}}!
Può anche capitare di vedere \textbf{\emph{una bambina di 10 anni già
con una curva di moderata gravità (15-20 gradi). Anche se si deve ancora
sviluppare comunque ha già una scoliosi importante quindi facciamo
subito un trattamento ortopedico con corsetto o un trattamento
chirurgico.}} In entrambi gli approcci \emph{l'accrescimento diventa un
alleato utile} nella terapia. È facile quindi in questo campo fare la
diagnosi, ma i problemi sorgono poi quando si tratta di fare il
trattamento. Le variabili in campo sono tantissime e vanno considerate
tutte insieme in modo tale da tirar fuori una valutazione, mediata anche
dall'esperienza, che porti poi a fare trattamento o meno evitando
indicazioni inutili come il nuoto, la ginnastica correttiva, il
materasso duro (che male non fanno, ma fanno perdere tempo e in qualche
caso a distanza di 2-3 anni si vedono colonne che peggiorano e devono
essere operate). \emph{Queste indicazioni non hanno alcun valore né
sull'innesco, né sulla prevenzione della scoliosi, né sulla prevenzione
del peggioramento. }

Un tentativo terapeutico proposto, ma che non ha avuto alcun successo,
consiste nello stimolare i muscoli che sono dalla parte della convessità
della curva: si parte dal presupposto che i muscoli della concavità
tirino di più, mentre quelli della convessità tirino di meno. Si pensò
così di fare delle elettrostimolazioni sui muscoli che sostengono la
colonna dalla parte dove sono più sgonfi. In realtà il risultato è stato
deludente.

Ricapitolando: punto imprescindibile è una \textbf{\emph{precoce
diagnosi}} e una \textbf{\emph{corretta}} \textbf{\emph{valutazione}}
del tipo di scoliosi \textbf{\emph{così da trattare}} solo quelle che
effettivamente lo richiedono:
\end{quote}

\begin{itemize}
\item
  Scoliosi per le quali esiste un \textbf{\emph{rischio accertato di
  aggravamento}}
\item
  Scoliosi in cui \textbf{\emph{già alla prima osservazione si abbiano
  gradi elevati di curvatura}} anche se lo sviluppo non fosse ancora
  completato (tipo di 30 gradi)
\end{itemize}

\begin{itemize}
\item
  \textbf{TRATTAMENTO ORTOPEDICO}
\end{itemize}

Questo consiste prevalentemente nell'utilizzo dei corsetti o busti. Ce
ne sono di diversi tipi e ogni scuola usa quelli che ritiene più giusti
in base all'esperienza e alla filosofia di partenza. Normalmente i
corsetti prendono il nome della città dove sono stati inventati.

Il corsetto deve:

\begin{itemize}
\item
  Essere fatto \emph{su misura} (non standard, ma individuale)
\item
  Essere \emph{rigido} altrimenti è il paziente che piega il busto e non
  è il busto che raddrizza il paziente
\item
  Avere delle sbarre di \emph{sostegno}, \emph{anelli cervicali, ecc}.
  (ci sono varie forme). Il corsetto agisce dall'esterno sulle
  principali deformazioni caratterizzanti la scoliosi ovvero il gibbo
  costale e la prominenza lombare: il pressore spinge sulle coste e le
  va a de-ruotare andando a de-ruotare ,attraverso le coste, anche le
  vertebre dorsali. Il meccanismo d'azione si basa su una
  \textbf{de-rotazione}, una \textbf{trazione} e una \textbf{spinta}
\item
  C'è una \textbf{fisiochinesiterapia} in corsetto ovvero guidata dal
  corsetto stesso
\item
  Avere una \emph{precocità}
\item
  Essere applicato \emph{in maniera esatta}
\item
  \emph{\emph{Va usato e usato correttamente (non basta comprarlo!)}}
\end{itemize}

\includegraphics[width=3.47917in,height=1.95833in]{media/image2.jpeg}

\begin{quote}
Inoltre saranno necessari dei controlli periodici del paziente e
dell'ortesi \emph{soprattutto nei bambini} che con il corsetto ci fanno
di tutto perciò alla lunga questo può perdere di rigidità perché ad
esempio si allentano le viti.

Inoltre si potrebbe osservare un \textbf{miglioramento} perciò la spinta
iniziale data con il pressore può non essere più sufficiente dopo sei
mesi. Ciò rende necessario aumentare tale spinta magari aumentando lo
spessore del pressore.

Importanti sono:
\end{quote}

\begin{itemize}
\item
  Lo \textbf{\emph{svezzamento}}: occorre un adeguato ``svezzamento''
  dal corsetto cioè lo si porta fin quando non finisce l'accrescimento.
  Ad esempio lo mettiamo ad un bambino di 12 anni e questa lo porta per
  due anni fin quando non si osserva un miglioramento, a quel punto lo
  togliamo. Così facendo però sbaglieremmo perché il bambino non è
  ancora pienamente sviluppato \textbf{\emph{per cui con il successivo
  accrescimento perderemmo tutti i buoni risultati raggiunti}}. Quindi
  lo svezzamento \textbf{\emph{deve essere ragionato e graduale}} (es:
  portandolo solo la notte, ecc.)
\item
  \emph{La \textbf{correttezza e esattezza delle indicazioni}}: non
  bisogna mettere il corsetto a tutti i casi di scoliosi solo per
  ``avere la coscienza pulita''. Bisogna considerare che
  nell'adolescente non esiste solo l'implicazione organica, ma
  soprattutto quella \textbf{psicologica} per cui è importante
  individuare solo i casi in cui il trattamento è davvero necessario.
\end{itemize}

\includegraphics[width=2.84375in,height=1.60278in]{media/image3.jpeg}Esempio
di corsetto: a livello del gibbo toracico è posto un pressore e
dall'altra parte agisce un pressore lombare fisso. Il corsetto esercita
una trazione e delle pressioni sulle coste che permettono la
de-rotazione della colonna.

Vediamo ora i vari tipo di corsetto:

\begin{quote}
\textbf{Corsetto Milwakee,} \emph{ideale per le bambine piccole:
esercita una trazione del rachide, una pressione localizzata sulla
convessità della curva e una pressione nella cintura pelvica a livello
lombare (teoria dei tre punti). In una curva di scoliosi la parte
interna della vertebra cresce meno perché sottoposta a maggiore
pressione rispetto alla parte esterna perciò la colonna viene trazionata
e spinta attraverso le coste in senso di de-rotazione, così che la
deformità trapezoidale tenda a recuperare}.

\includegraphics[width=2.84722in,height=1.60417in]{media/image4.jpeg}Questi
rappresentati sono esempi di esercizi in corsetto (di trazione, di
allungamento, ma anche ad esempio alzare una spalla poi alzare l'altra e
l'esercizio controlaterale) per cercare di ridurre queste curve.
Ultimamente questi esercizi sono stati però un po' superati.

\includegraphics[width=3.62569in,height=2.05347in]{media/image5.png}\textbf{Corsetto
lombare basso (Bolognese)}, \emph{inventato a Bologna.}

\emph{È simile al corsetto Milwakee senza però un attacco superiore e
serviva solo per le scoliosi lombari.}

\includegraphics[width=2.89583in,height=1.63542in]{media/image6.jpeg}\textbf{Corsetto
Lionese,} \emph{si tratta di un corsetto molto più statico, ma che
agisce come gli altri (con una pressione localizzata che determina la
de-rotazione delle vertebre). Ideale per le ragazze già sviluppate
perché più rigido e correttivo.}

A seconda delle metodiche usate nelle diverse scuole e della gravità
della patologia, il corsetto può essere utilizzato:
\end{quote}

\begin{itemize}
\item
  Solo di notte
\item
  Part-time (con 6-8 ore libere dal corsetto da gestirsi a piacimento
  nella giornata)
\item
  Tempo pieno
\end{itemize}

\begin{quote}
Dipende dalle varie situazioni: in una bambina di 13 anni magari
sviluppata già da due mesi (quindi abbiamo ancora un anno, un anno e
mezzo per sfruttare l'accrescimento) si cercherà di ridurre il tempo di
libertà perché farle portare il corsetto solo di notte sarebbe poco. In
questo caso dal momento che il trattamento si presuppone di massimo un
anno/un anno e mezzo (perché già sviluppata) quindi breve, si applica il
corsetto a tempo pieno.

Se invece la bambina ha 9/10 anni, c'è a disposizione più tempo per
poterla trattare e allora si può decidere di utilizzare il corsetto
part-time permettendo anche otto ore libere durante la giornata.

Se poi il busto ha migliorato la curva, si può aumentare il tempo di
libertà ad esempio facendoglielo portare solo la notte.
\end{quote}

\includegraphics[width=5.35417in,height=3.02083in]{media/image7.jpeg}

\begin{quote}
Quella riportata è una classica radiografia dove si vede bene la curva:
si tratta di una scoliosi vera. Di fianco la radiografia in corsetto per
vedere se quel corsetto funziona o meno: non dobbiamo aspettarci di
vedere una curva ``raddrizzata'' perché abbiamo a che fare con strutture
che hanno una certa elasticità e più di tanto non si può fare.
Normalmente per vedere se quel corsetto lavora bene o no occorre
studiare la plasticità dei dischi intervertebrali (trucco super
specialistico): se un disco intervertebrale è ben ``aperto'' allora più
di tanto non posso fare. Io posso agire spingendo solo sull'elasticità
data dai dischi quindi non mi aspetterò in un simile caso di vedere una
schiena dritta. C'è una certa elasticità che va rispettata.

Il trattamento non chirurgico può prevedere anche l'utilizzo dei gessi
che ovviamente non potranno essere tolti. Il principio è lo stesso:
\emph{sfruttare meccanismi di pressione localizzata e trazione. Si
mettono dei pressori che vengono poi ingessati nei busti. }

Vengono usati per esempio laddove il gibbo è molto accentuato oppure
quando la scoliosi è più rigida e strutturata perché il corsetto di
plastica lavora di più invece sulla parte estetica.
\end{quote}

\includegraphics[width=5.45833in,height=3.08333in]{media/image8.jpeg}

\begin{quote}
Si controbilancia il fatto che non possano essere tolti facendone ad
esempio tre in un inverno, di due mesi in due mesi, in modo tale da
poter fare una doccia tra uno e l'altro. In estate solitamente non
vengono fatti, se non nei casi più severi. È una correzione più statica
che si usa nei casi gravi.
\end{quote}

\includegraphics[width=4.25000in,height=2.39583in]{media/image9.jpeg}

\begin{quote}
Questi sono dei gessi la cui confezione è stata standardizzata a Pisa.
\end{quote}

\begin{itemize}
\item
  \textbf{TRATTAMENTO CHIRURGICO}
\end{itemize}

\begin{quote}
Le indicazioni sono:
\end{quote}

\begin{itemize}
\item
  Scoliosi grave, oltre i 40 gradi
\item
  Scoliosi di moderata entità in cui però il trattamento ortopedico
  sarebbe troppo lungo oppure se ci sono condizioni socio-economiche
  della famiglia per cui non si riesce a portare a termine il
  trattamento ortopedico (non fa portare il busto, non porta il bambino
  ai controlli) o un problema di mentalità per il quale non viene
  compresa la necessità del trattamento che quindi non viene effettuato.
\item
  Scoliosi in cui è fallito il trattamento ortopedico. Ci sono dei casi
  che in gergo si chiamano ``maligni'' in cui nonostante sia stato fatto
  tutto bene, c'è comunque un aggravamento della curva.
\end{itemize}

\includegraphics[width=5.58333in,height=3.14583in]{media/image10.jpeg}

\begin{quote}
Un tempo prima del trattamento chirurgico veniva usato questo sistema di
corde con una pedana. Era realizzato in maniera tale che il paziente,
allungando le gambe, attraverso questa carrucola potesse esercitare una
trazione, ovviamente non dolorosa. Serviva per fare in modo che la
colonna fosse più elastica così che poi nel campo chirurgico potesse
essere corretta meglio. Ricordate che dentro la colonna c'è il midollo
spinale e non è una bella idea pensare di ``raddrizzare'' una colonna
storta tutta in uno stesso momento perché \emph{una delle complicanze di
questo intervento è la paraplegia}. Questo sistema serviva quindi a
rendere più elastica la colonna, ma è stato poi abbandonato.
\end{quote}

\includegraphics[width=3.59375in,height=2.02569in]{media/image11.jpeg}

Esempio classico di scoliosi di una ragazza di 15-16 anni trattata con
la chirurgia.

Si tratta di un intervento di \textbf{artrodesi} cche realizza un
\textbf{blocco dell'allungamento della colonna}. Questa è la ragione per
cui vanno eseguiti quando l'accrescimento è terminato altrimenti il
risultato finale potrebbe essere compromesso ad esempio il paziente
potrebbe avere le gambe troppo lunghe rispetto al busto Questa è la
radiografia: si vede che è una scoliosi severa in cui anche il polmone è
compromesso.

\includegraphics[width=3.51042in,height=1.97778in]{media/image12.jpeg}

\begin{quote}
Ci sono poi i così detti bending test (dall'inglese flettere) o test di
elongazione che vengono eseguiti per scegliere l'area in cui agire:
\emph{servono per capire quali sono le curve principali}.

\includegraphics[width=3.45556in,height=1.94792in]{media/image13.jpeg}\includegraphics[width=3.48958in,height=1.96667in]{media/image14.jpeg}Bisogna
stare attenti a non sbagliare con le scoliosi perché se una forma che
necessitava di correzione con trattamento ortopedico, non venisse
riconosciuta, poi andrà incontro a peggioramento e quando non si potrà
più fare nulla, andrà operata con questo intervento che prevede circa
80cm/un metro di incisione.

Si tratta di un intervento di \textbf{artrodesi posteriore} ovvero
\textbf{\emph{blocco dell'area del rachide scelta che comprende
ovviamente le curve principali. }}
\end{quote}

Solo per realizzare il campo chirurgico sono necessarie due ore: occorre
puntellare la colonna con viti e chiodi peduncolari \emph{che penetrano
posteriormente attraverso i peduncoli raggiungendo il corpo vertebrale}.
Si costruisce quindi una struttura metallica che \emph{traziona il
rachide ed esercita una serie di pressioni e decompressioni localizzate
così da correggere le deformità.}

\begin{quote}
Queste \emph{viti} hanno delle teste con delle scanalature che
permettono di raccordarle ( collegarle ) meccanicamente con un sistema
di \emph{barre} che possono essere in acciaio o in titanio. Questo è lo
strumentario che aiuta nell'artrodesi.

L'artrodesi in sé si effettua \emph{cruentando} l'osso cioè
\emph{scavando la parte posteriore delle vertebre} e si realizza quindi
una colata di callo osseo sull'area interessa sfruttando anche del
tessuto osseo spongioso prelevato dalla banca dell'osso (la componente
inorganica di questo materiale osseo funziona infatti da
\emph{osteoinduttore} = stimola la formazione di osso e
\emph{osteoconduttore} = intelaiatura per la colonizzazione di cellule
del ricevente).

Nell'esempio illustrato dal prof l'osso dei processi spinosi viene
rimosso ed usato come materiale per il trapianto d'osso. In questo caso
sono state usate nel trapianto d'osso anche delle teste di femore
prelevate da pazienti operati di coxartrosi con protesi d'anca.
Quest'osso viene spezzettato: si decorticano le lamine e viene
realizzata la colata di callo osseo sia con l'osso delle spinose sia in
parte con l'osso spezzettato del femore (provenienti da banche
dell'osso).

\emph{Quello che tiene meccanicamente nel tempo non è l'acciaio o il
titanio, ma è l'artrodesi che finisce per inglobare la struttura fatta
di viti e barre}: paradossalmente questa strumentazione, nel momento in
cui si è creata una distesa di osso, potrebbe anche essere tolta. Non si
fa ovviamente perché l'osso ingloba tutto il metallo e sarebbe un
intervento più impegnativo di quello iniziale.

La mobilità è solo lievemente ridotta, ma la qualità della vita è
nettamente migliorata rispetto a prima anche in vista delle complicanze
artrosiche della scoliosi.

Ci potrebbero essere delle complicanze perché se non tiene
l'impalcatura, l'osso si potrebbe rompere mentre a volte si ha una
pseudoartrosi: se l'osso non si fonde, lo strumentario per conto suo non
riesce a resistere a tutte le sollecitazioni di una vita normale e prima
o poi si rompe, con perdita della correzione.
\end{quote}

\includegraphics[width=4.75000in,height=2.67708in]{media/image15.jpeg}

\begin{quote}
Qui vediamo il risultato finale: le viti partono da dietro e terminano
nel corpo vertebrale. Bisogna stare attenti a non piantare queste viti
nel midollo spinale. Quest'area di artrodesi \textbf{\emph{viene
bloccata}}, ma è ovvio che questa parte di colonna dal momento che non è
così mobile non darà grossi danni ai fini ad esempio del raccogliere una
matita. Bloccando però fino alla zona lombare, le ultime quattro
vertebre libere sicuramente non godono e vanno incontro a un processo di
degenerazione.

Sono interventi che devono essere fatti in centri specializzati. I primi
interventi si facevano con dei chiodi che prevedevano sei mesi di gesso:
due mesi stando a letto, due mesi iniziando a muoversi e due mesi
svolgendo le normali attività. Oggi invece con i nuovi strumentari,
siccome viene distribuito tutto a vari livelli, i pazienti possono
rimettersi in piedi dopo 4-5 giorni a seconda di quanto sangue abbiano
perso.

Qualche volta le cicatrici non sono molto evidenti: quando dopo
l'intervento erano previsti sei mesi di gesso, le cicatrici erano
migliori perché erano protette da questo. Con i nuovi interventi, a
seguito dei quali il paziente può muoversi subito, i lembi della ferita
si muovono di più e la cicatrice tende a dare reazioni cheloidee più
significative.
\end{quote}

\includegraphics[width=5.35417in,height=3.02083in]{media/image16.jpeg}

\begin{quote}
Qui vediamo il prima e dopo: osserviamo come la colonna si è de-ruotata.
Questo non è un discorso estetico, ma soprattutto funzionale. La gabbia
toracica si è normalizzata e il polmone risente meno della costrizione.
In ortopedia l'estetica conta poco. L'ortopedico è un chirurgo
funzionale.

Il messaggio fondamentale è che la scoliosi è una cosa grave e anche se
il 99\% degli ortopedici ha a che fare con le forme che non peggiorano,
bisogna sapere che esistono anche forme che peggiorano e che vanno
trattate nel modo adeguato.

Domanda: ma le viti messe durante l'intervento sono riassorbibili? No,
anche se meccanicamente si potrebbero anche rimuovere (è capitato
qualche volta perché possono dare fastidi e gonfiori). Quello che tiene
infatti è l'artrodesi! Liberare ogni vite dall'osso sarebbe un
intervento lungo e pericoloso che non viene fatto, se non per necessità.

Domanda: ci possono essere complicanze a livello di midollo osseo? Sì,
se stiri troppo lo mandi in paresi. Ci sono due modi per capire se c'è
stata o meno una lesione spinale:
\end{quote}

\begin{itemize}
\item
  Quando non esistevano i potenziali evocati, si svegliava il malato
  attraverso un alleggerimento dell'anestesia e un'infermiera con un
  cocker (pinza con i denti) andava a dare morsi sugli alluci. Se il
  malato muoveva le gambe, voleva dire che magari potevi andare a tirare
  ancora un po'
\item
  Oggi si va in sala operatoria con dei neurologi e si sfruttano i
  \textbf{potenziali evocati}: senza svegliare il malato si usano gli
  elettrodi e un'apparecchiatura di elettrofisiologia così si riesce a
  capire quanto si può stirare il midollo.
\end{itemize}

\begin{quote}
Al prof è capitato una solo volta di dover tornare in sala tre ore dopo
l'intervento per un problema di coagulazione intravasale disseminata. Ha
dovuto riaprire e detendere, ma non per un problema di trazione bensì di
coagulazione. Erano i primi tentativi che si facevano per il recupero
intraoperatorio del sangue: succedeva che per rendere il sangue più
fluido, venivano aggiunti degli anticoagulanti al sangue che veniva
riciclato e reinfuso. Purtroppo questi scatenarono una coagulazione
intravasale disseminata con consumo di tutti i fattori della
coagulazione e sanguinamento eccessivo. Si creò così un ematoma nel
canale vertebrale.

\emph{Domanda: esiste un cut-off di angolo di scoliosi a partire dal
quale è consigliato il trattamento ortopedico? No, perché il valore
assoluto non è significativo. Dipende tutto dall'età anagrafica e
soprattutto dall'età di accrescimento scheletrico. E' chiaro però che,
se il tessuto familiare del bambino con una scoliosi di 20 gradi non
garantisce una corretta terapia ortopedica, conviene operare subito per
non perdere tempo prezioso.}
\end{quote}

\emph{Torcicollo}

\emph{Scoliosi (trattamento} \emph{Scoliosi (trattamento}

\emph{Il prof ci tiene a sottolineare che per lui, in sede di esame, è
importantissimo sapere la definizione: così facendo ci si è quasi
assicurati la promozione.}

Il torcicollo viene definito come una deviazione laterale permanente
(quindi deformità) del capo e del collo.

\emph{Si tratta quindi di una deformità dove per deformità si intende
una posizione o situazione che l'individuo non può correggere
volontariamente (a differenza di una postura, o di un atteggiamento).
Nel caso del torcicollo, infatti, si parla di una patologia vera e
propria.}

La classificazione lo distingue in:

\begin{itemize}
\item
  \textbf{Congenito}, che può essere:
\end{itemize}

\begin{itemize}
\item
  Miogeno che (lo sottolineiamo fin da subito) non dà un gran problema
  funzionale, ma soprattutto di tipo estetico
\item
  Osseo
\end{itemize}

\begin{itemize}
\item
  \textbf{Acquisito}, che può essere:
\end{itemize}

- reumatico

- osteoarticolare

- nervoso

- isterico

- oculare

- otogeno

- miopatico.

Questi hanno valenza per una diagnosi differenziale: in tutte queste
possiamo avere caratteristiche simile al torcicollo inteso come
deformità sia ossea che miogena.

\emph{La classificazione è utile dal punto di vista clinico perché aiuta
nella diagnosi differenziale. Infatti spesso in ortopedia si presentano
casi che hanno in comune sintomi (tutti i torcicolli si presentano come
deviazioni del collo, le scoliosi come deviazioni della colonna
vertebrale, ecc.), ma che hanno eziologie ed origini diverse: ci possono
essere patologie secondarie, primitive o anche idiopatiche. Sulla base
delle informazioni derivanti dalla causa della patologia possiamo
mettere a punto un trattamento adeguato.}

\emph{\textbf{TORCILLO OSSEO}}

Deriva da \textbf{\emph{malformazioni congenite vertebrali}}
sovrapponibili a quelle delle scoliosi.

\includegraphics[width=3.31250in,height=1.86458in]{media/image17.jpeg}
\includegraphics[width=3.19792in,height=1.80208in]{media/image18.jpeg}

Può derivare da:

\begin{itemize}
\item
  \textbf{Difetti di formazione}: si forma solo una mezza vertebra e non
  l'intera ovvero una struttura detta emispondilo
\item
  \textbf{Difetti di segmentazione} quindi sinostosi ovvero fusioni che
  \emph{in realtà sono mancate separazioni tra un osso e l'altro perciò
  difetti di segmentazione: le vertebre sono formate, ma non separate e
  possono essere fuse in corrispondenza o dell'intera superficie di
  confine o di una parte di questa. }
\end{itemize}

\includegraphics[width=4.69792in,height=2.64583in]{media/image19.jpeg}

La distinzione fra emispondilo e sinostosi è importante per la prognosi:

\begin{itemize}
\item
  Per l'emispondilo \emph{può esserci la possibilità che l'individuo non
  cresca con una deviazione del collo} perché per esempio possono
  esserci due emispondili in grado di controbilanciarsi e riassettare la
  linea del collo. In sostanza si ha una \emph{compensazione} del
  difetto.
\item
  Per la sinostosi la prognosi è sicuramente \emph{più sfavorevole}
  perché \emph{l'arresto della crescita della porzione delle vertebre
  che non si è segmentata è parallelo alla progressione della crescita
  della porzione segmentata.} La tendenza alla flessione laterale del
  capo e del collo peggiorerà
\end{itemize}

All'esame clinico vediamo:

\begin{itemize}
\item
  Una \textbf{\emph{mobilità limitata del collo}} che però si vede meno
  nel miogeno
\item
  \textbf{\emph{Assenza dei segni di retrazione dello
  sternocleidomastoideo}} che invece caratterizzano il torcicollo
  miogeno.
\end{itemize}

\includegraphics[width=4.77083in,height=2.68750in]{media/image20.jpeg}

Vediamo un esempio di radiografia. Studiare il rachide dal punto di
vista radiografico è praticamente impossibile! Se si volesse
approfondire lo studio, bisognerebbe eseguire una TAC.

Ci si accorge del problema osseo perché quando muoviamo la testa del
bambino, l'arresto è più brusco, il movimento è più limitato. Mentre nel
\textbf{miogeno} vedremo che il \textbf{collo è più elastico} perché la
colonna è normale.

\emph{\textbf{TORCICOLLO MIOGENO}}

Deriva da una \textbf{\emph{retrazione del muscolo
sternocleidomastoideo}} ed è detto anche ostetrico o congenito-muscolare
(congenito significa connatale e non ereditario).

Può essere dovuto ad alterazioni prevalenti del:

\begin{itemize}
\item
  Capo clavicolare
\item
  Capo sternale
\item
  Capo mastoideo
\item
  Di tutti e tre (il muscolo ha tre inserzioni: sternale, clavicolare e
  mastoidea).
\end{itemize}

\includegraphics[width=5.78125in,height=3.26042in]{media/image21.jpeg}

L'eziopatogenesi non è chiara:

\begin{itemize}
\item
  Fattori genetici (\emph{di familiarità, non ereditarietà}) non ancora
  scoperti
\item
  Ipoplasia congenita del muscolo sternocleidomastoideo (\emph{anche se
  questa condizione da sola non spiega la progressiva retrazione del
  muscolo che invece si presenta nel torcicollo})
\item
  Compressione intrauterina: \emph{per \emph{ischemia da oligoidramnios}
  ovvero quantità esigua di liquido amniotico che condiziona determinate
  posizioni anomale del feto che riducono l'apporto ematico alla massa
  muscolare oppure \emph{per mal posizione}}
\item
  Fibromatosi che a oggi rappresenta la teoria più accreditata
  (\emph{per amartoma si intende un lesione pseudo tumorale, non
  maligna, dovuta allo sviluppo delle cellule normali di quel tessuto in
  maniera anarchica, ma non si tratta di cellule anaplastiche. Sia
  l'amartoma che l'ematoma tendono a rapprendersi nel tempo e quindi a
  retrarre il muscolo})
\item
  Trauma ostetrico per questo detto anche ostetrico. Era dovuto ad un
  ematoma da distocia: in parti detti distocici veniva usato uno
  strumento chiamato forcipe che afferrava il feto per la testa. Si
  creavano o \emph{trazioni eccessive sul collo} o \emph{una
  compressione sul muscolo sternocleidomastoideo}. Questo creava un
  ematoma che con l'andare del tempo si organizza, diventava fibroso e
  si rapprendeva così da accorciare il muscolo.
\end{itemize}

\begin{quote}
\includegraphics[width=4.35417in,height=2.44792in]{media/image22.jpeg}

\emph{Anche in caso di parto non distocico si poteva avere torcicollo:
si è notato che questo ematoma si può formare anche in neonati con parto
normale o cesareo quindi il traumatismo può non essere la causa.}

L'ipotesi più accreditata è quello di un \emph{amartoma} cioè un'isola
di tessuto indifferenziato che con il tempo da un punto di vista
biologico, ma soprattutto meccanico, ha la stessa evoluzione
dell'ematoma cioè si rapprende e dà una retrazione del muscolo
sternocleidomastoideo.
\end{quote}

\emph{Quadro clinico ed evoluzione}

\includegraphics[width=4.32292in,height=2.43750in]{media/image23.jpeg}

\emph{Il quadro clinico del torcicollo miogeno è povero di segni alla
nascita:} non si vede un granché. \emph{A volte è palpabile l'ematoma o
l'amartoma al collo}, si può percepire un pallino, ma di fatto non ce ne
accorgiamo. La diagnosi è clinica mentre gli esami strumentali sono
difficili da compiere alla nascita: al massimo un'ecografia ci può far
vedere o l'amartoma o l'ematoma.

\begin{quote}
\includegraphics[width=3.61458in,height=2.03750in]{media/image24.jpeg}
\end{quote}

Dopo con l'accrescimento, vediamo che dalla parte dove ``tira'', il
volto e il collo si sviluppano meno. Dall'altra parte invece crescono di
più. Si inizia a vedere l'atteggiamento della testa in flessione dalla
parte del muscolo interessato e di rotazione dalla parte opposta.
Infatti col progressivo accrescimento si verifica:

\begin{itemize}
\item
  \emph{Flessione omolaterale} del capo
\end{itemize}

\begin{quote}
\emph{E}
\end{quote}

\begin{itemize}
\item
  \emph{Rotazione controlaterale} del capo.
\end{itemize}

\includegraphics[width=3.77083in,height=2.12500in]{media/image25.jpeg}

Queste condizioni tendono a peggiorare con l'accrescimento e poi compare
il \textbf{segno della corda} (attorno ai 5-6-7 anni di età): tenendogli
ferme le spalle, la testa è fatta ruotare dalla parte opposta al muscolo
affetto portando così la deformità in correzione. \emph{Forzando la
rotazione dalla parte della deformità, ho la limitazione del movimento
opposto, e si può rilevare tensione in superficie in corrispondenza del
muscolo retratto (lo sternocleidomastoideo)}. C'è una limitazione del
movimento opposto, ma è una limitazione che, rispetto al torcicollo
osseo, è molto più elastica. \emph{La limitazione nel caso del
torcicollo miogeno è di tipo elastico, differente dal caso del
torcicollo osseo nel quale non ci sarebbe segno della corda e inoltre la
limitazione di movimento sarebbe molto più dura.}

\includegraphics[width=5.69792in,height=3.20833in]{media/image26.jpeg}

Con il passare del tempo si assiste ad un'alterazione delle linee di
accrescimento osseo \emph{che si strutturano secondo linee di forza
alterate}: dalla parte del muscolo interessato c'è un freno ai vettori
dell'accrescimento mentre dalla parte opposta no. Come conseguenza di
questa crescita asimmetrica (il capo cresce più dalla parte opposta)
avremo una scoliosi del capo detta \emph{plagiocefalia} e una scoliosi
del volto detta \emph{plagioprosopia} ovvero asimmetria degli assi della
bocca e degli occhi che si vede tirando due linee immaginarie: una che
passa gli occhi e una per gli angoli della bocca. Queste linee non sono
parallele, ma sono convergenti dalla parte del muscolo interessato.

\includegraphics[width=5.10417in,height=2.87500in]{media/image27.jpeg}

Vediamo un esempio di come si presenta nell'adulto. È molto difficile
vederli dopo l'infanzia. Qui si vede molto bene il segmento che tira. A
questo punto non si fa più nulla perché al di là del problema estetico e
della limitazione della rotazione del collo, non ci sono problematiche
importanti.

\emph{Trattamento}

Il trattamento non è solitamente necessario ed è solo chirurgico. Si
esegue la \textbf{tenotomia tripolare}: si tagliano tutti e tre i poli
del tendine del muscolo (sostanzialmente significa allungarlo, ma non
potendolo allungare, si taglia). Si taglia a livello di inserzione
mastoide, sternale e scapolare.

\includegraphics[width=4.40625in,height=2.47917in]{media/image28.jpeg}

Vediamo che l'orecchio viene divaricato, con uno specillo si prende il
tendine che viene prima isolato con un cocker e poi tagliato così da
eliminare un pezzo di muscolo. La stessa cosa viene fatta poi a livello
clavicolare e sternale.

Non è un intervento semplice: bisogna rimanere superficiali dal momento
che sotto ci sono strutture come la succlavia che non vanno tagliate.

\includegraphics[width=3.40625in,height=1.92014in]{media/image29.jpeg}In
seguito alla tenotomia, per evitare la recidiva, vi è anche
l'asportazione di 1 cm di tendine da tutti e tre i poli perché assieme
al trattamento post-operatorio, l'asportazione del cm evita che il
muscolo si ``riattacchi'' dov'era prima (recidiva).

Nel post-operatorio si utilizza il \emph{collare morbido di Schanz} (che
supporta il collo) finché le ferite sono fresche e finché non si fa la
desutura (circa dieci giorni). È un collare di ovatta, cotone da gesso e
garza.

\includegraphics[width=3.45833in,height=1.94931in]{media/image30.jpeg}

Importante è l'uso di una \emph{minerva gessata} (apparecchio gessato
Minerva) che tiene immobilizzato il capo e il collo in
\textbf{ipercorrezione} cioè dal lato opposto alla parte lesa quindi
\emph{il collo sarà in flessione controlaterale e rotazione omolaterale}
(la deformità è il contrario). Questo dispositivo viene mantenuto per 2
mesi. Sono libere le braccia fino allo sterno e la testa viene
immobilizzata.

L'asportazione di 1 cm dai poli, la tenotomia e la posizione per circa
due mesi in ipercorrezione faranno sì che i due poli non si riattacchino
esattamente dove li abbiamo tagliati.

L'uso della minerva ingessata nel post-operatorio è il motivo per cui
l'intervento chirurgico non può essere eseguito prima che il bambino
abbia raggiunto una \emph{autonomia alimentare} cioè prima dei due /due
anni e mezzo di età perché prima di allora i bambini si nutrono
prevalentemente con la suzione per cui con l'apparecchio gessato
risulterebbe difficoltosa la gestione alimentare.

Quindi fino ai due anni cosa faccio?

\begin{itemize}
\item
  Collare di Schanz
\item
  Collare in plastica
\item
  Caute mobilizzazioni del capo
\end{itemize}

Tutto per limitare l'aggravamento della retrazione.

\emph{\textbf{TORCICOLLO ACQUISITO}}

\emph{Torcicollo reumatico}

\includegraphics[width=3.38542in,height=1.90764in]{media/image31.jpeg}

È il più frequente, si tratta di un'infiammazione del perimisio (ovvero
la guaina) del muscolo a seguito di colpi di freddo dati da aria
condizionata, finestrino aperto, ecc. Basta prendere un'aspirina.

\emph{Torcicollo osteoarticolare}

Può essere dato da

\begin{itemize}
\item
  \emph{Alterazioni ossee}: rachitismo, lesioni infiammatorie,
  neoplasie, traumatismi
\item
  \emph{Alterazioni articolari}: spondilite anchilopoietica, artrite
  reumatoide, ernia discale.
\end{itemize}

\begin{quote}
\includegraphics[width=3.54167in,height=1.99583in]{media/image32.jpeg}In
questi casi esiste un duplice meccanismo: c'è la lesione organica, ma
anche la contrattura antalgica, per cui si crea torcicollo per avere
meno dolore.
\end{quote}

Una causa relativamente comune può essere la sublussazione rotatoria
C1-C2, in questo caso c'è un segno caratteristico: il \textbf{paziente
che si tiene la testa}. Non è grave ma va riconosciuto.

Quest'ultimo si osserva spesso nei bambini: a seguito di uno schiaffo,
il bambino gira violentemente la testa e si crea una sublussazione
rotatoria tra atlante ed epistrofeo. Il bambino si tiene la testa perché
il rachide cervicale non riesce a tenerla su. Classica condizione di
insufficienza del rachide che va indagata perché potrebbe sottendere una
situazione più grave (va messa in relazione con gli eventi).

\emph{Torcicollo nervoso}

Per paralisi, spasticismo, nevralgia del trigemino o altre cause.

\emph{Torci}\includegraphics[width=3.42708in,height=1.93194in]{media/image33.jpeg}\emph{collo
isterico}

L'isteria può mimare tante patologie. Non c'è nulla di organico. Vanno
trattati con Valium.

\emph{Torcicollo oculare}

\includegraphics[width=3.42708in,height=1.93125in]{media/image34.jpeg}

Lo \textbf{strabismo lieve} dà una \emph{diplopia} quindi per
sovrapporre l'immagine il paziente piega il capo e questo porta il
bambino a inclinare la testa per sovrapporre i margini della visione.

Lo \textbf{strabismo grave} invece porta a \emph{esclusione di un
occhio} dalla visione da parte del cervello quindi non elabora le
informazioni visive.

In questi casi si chiude l'occhio e si vede se il torcicollo migliora.
Il torcicollo oculare scompare correggendo il difetto oculare.

\emph{Torcicollo otogeno}

Le otiti purulente e le mastoiditi danno grosso dolore che può portare
ad un atteggiamento antalgico che simula un torcicollo

\emph{Torcicollo miopatico}

\includegraphics[width=2.98958in,height=1.68472in]{media/image35.jpeg}\emph{Le
miopatie possono causare con un'alterazione dell'equilibrio (squilibri)
dei due sternocleidomastoidei.}

\end{document}
