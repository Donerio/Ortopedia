\documentclass[]{article}
\usepackage{lmodern}
\usepackage{amssymb,amsmath}
\usepackage{ifxetex,ifluatex}
\usepackage{fixltx2e} % provides \textsubscript
\ifnum 0\ifxetex 1\fi\ifluatex 1\fi=0 % if pdftex
  \usepackage[T1]{fontenc}
  \usepackage[utf8]{inputenc}
\else % if luatex or xelatex
  \ifxetex
    \usepackage{mathspec}
  \else
    \usepackage{fontspec}
  \fi
  \defaultfontfeatures{Ligatures=TeX,Scale=MatchLowercase}
\fi
% use upquote if available, for straight quotes in verbatim environments
\IfFileExists{upquote.sty}{\usepackage{upquote}}{}
% use microtype if available
\IfFileExists{microtype.sty}{%
\usepackage{microtype}
\UseMicrotypeSet[protrusion]{basicmath} % disable protrusion for tt fonts
}{}
\usepackage[unicode=true]{hyperref}
\hypersetup{
            pdfborder={0 0 0},
            breaklinks=true}
\urlstyle{same}  % don't use monospace font for urls
\usepackage{graphicx,grffile}
\makeatletter
\def\maxwidth{\ifdim\Gin@nat@width>\linewidth\linewidth\else\Gin@nat@width\fi}
\def\maxheight{\ifdim\Gin@nat@height>\textheight\textheight\else\Gin@nat@height\fi}
\makeatother
% Scale images if necessary, so that they will not overflow the page
% margins by default, and it is still possible to overwrite the defaults
% using explicit options in \includegraphics[width, height, ...]{}
\setkeys{Gin}{width=\maxwidth,height=\maxheight,keepaspectratio}
\IfFileExists{parskip.sty}{%
\usepackage{parskip}
}{% else
\setlength{\parindent}{0pt}
\setlength{\parskip}{6pt plus 2pt minus 1pt}
}
\setlength{\emergencystretch}{3em}  % prevent overfull lines
\providecommand{\tightlist}{%
  \setlength{\itemsep}{0pt}\setlength{\parskip}{0pt}}
\setcounter{secnumdepth}{0}
% Redefines (sub)paragraphs to behave more like sections
\ifx\paragraph\undefined\else
\let\oldparagraph\paragraph
\renewcommand{\paragraph}[1]{\oldparagraph{#1}\mbox{}}
\fi
\ifx\subparagraph\undefined\else
\let\oldsubparagraph\subparagraph
\renewcommand{\subparagraph}[1]{\oldsubparagraph{#1}\mbox{}}
\fi

% set default figure placement to htbp
\makeatletter
\def\fps@figure{htbp}
\makeatother


\date{}

\begin{document}

\emph{Ginocchio (Pogliacomi)}

L'articolazione del ginocchio è un'articolazione che, dal punto di vista
anatomico e biomeccanico, può essere considerata una doppia
articolazione condiloidea fra i due condili del femore e i due piatti
tibiali, a cui si associa poi l'articolazione trocleare fra la troclea
femorale e la rotula.

L'articolazione del ginocchio dal punto di vista osseo non è
particolarmente stabile, ossia la convessità dei condili femorali non
corrisponde precisamente con la concavità dei due piatti tibiali.

(Al contrario a livello dell'anca la congruenza articolare è
particolarmente importante, la testa del femore forma una sorta di vuoto
con l`acetabolo.)

Per dare stabilità a questo tipo di articolazione subentrano degli
elementi stabilizzatori passivi e degli elementi stabilizzatori attivi.
Attivi sono i gruppi muscolari la cui contrazione rinforza
l'articolazione, tra questi vanno ricordati:

\begin{itemize}
\item
  i tendini del semimembranoso e i tendini della zampa d'oca medialmente
\item
  il sistema quadricipite-rotula-tendine rotuleo anteriormente,
\item
  il tendine del bicipite femorale in posizione laterale
\item
  i gastrocnemi in posizione posteriore.
\end{itemize}

Mentre i passivi sono la capsula e i legamenti. La capsula è
essenzialmente un manicotto fibroso che unisce il femore alla tibia,
rinforzata da diversi legamenti, cioè fasci di tessuto fibro-collageno
denso, simili ai tendini, la cui posizione e disposizione corrisponde
alle linee di tensione che entrano in funzione durante i movimenti
articolari del ginocchio, ed analoga struttura hanno anche i legamenti
crociati.

In generale, il ginocchio è un'articolazione complessa, anche se il suo
movimento avviene praticamente solo sul piano sagittale, in un
meccanismo di flesso-estensione che va da 180° a 30°, e in questo
movimento la superficie articolare della tibia ruota su quella del
femore, ma il centro di tale rotazione non è fisso, ma tende a spostarsi
in dietro col progredire della flessione. Quando il ginocchio è
completamente esteso, ogni movimento che non sia di flessione è
impossibile, mentre quando il ginocchio è flesso, sono concessi dei
movimenti di rotazione della tibia rispetto al femore: nella rotazione
interna, i due legamenti crociati si avvolgono tra loro e così la
limitano; nella rotazione esterna, invece, si svolgono e la limitazione
è allora principalmente affidata alla capsula mediale e al punto
d'angolo postero-interno.

È importante tenere a mente che, in condizioni normali, le strutture di
sostegno dell'articolazione del ginocchio agiscono in modo sinergico tra
loro, e la sezione di un solo legamento non produce alcuna instabilità
del ginocchio, con l'eccezione del legamento crociato posteriore, la cui
sezione determina immediata instabilità articolare, in quanto
rappresenta il vero asse portante di questa articolazione. Il legamento
crociato posteriore è un legamento robusto, si rompe solo con forze
molto intense, e quindi la sua interruzione crea instabilità del
ginocchio con sublussazione posteriore della tibia, ma fortunatamente è
un legamento extra-articolare e le sue lesioni tendono a cicatrizzare
piuttosto bene. Il legamento crociato anteriore, invece, è più fragile,
e molto più esposto alla rottura, perché partecipa alla stabilizzazione
del ginocchio in tutte le posizioni; una sua interruzione isolata non
determina instabilità articolare, ma tale instabilità può comparire in
caso di lesioni capsulari mediali o laterali, o se viene tolto il
menisco mediale. Le lacerazioni del crociato anteriore tendono a
cicatrizzare male, perché è una struttura poco vascolarizzata, e spesso
i suoi monconi tendono a sfilacciarsi e a retrarsi.

Legamenti

\includegraphics[width=3.42708in,height=4.50000in]{media/image1.jpeg}

A livello dell'articolazione i legamenti più importanti sono:

1. il \textbf{legamento} \textbf{collaterale mediale}, legamento
nastriforme che origina dalla faccia esterna del condilo femorale
mediale e si va ad inserire a ventaglio sulla faccia interna
dell'emipiatto tibiale mediale. In posizione mediale la capsula è
rinforzata anche dalle inserzioni tendinee del semimembranoso e dal
menisco mediale, e questa area prende il nome di ``punto d'angolo
postero-interno''

2. il \textbf{legamento collaterale laterale}, legamento cordoniforme
che origina dalla faccia esterna del condilo femorale laterale e si
inserisce in corrispondenza della testa del perone. In posizione
laterale la capsula è rinforzata anche dadalla banderella di Maissiat,
che è una prosecuzione della fascia lata e che si inserisce sul
tubercolo tibiale di Gerdy), dal legamento popliteo arcuato e dal
tendine del muscolo popliteo, e questi ultimi costituiscono quello che
prende il nome di ``punto d'angolo postero-esterno''.

Questi due legamenti controllano prevalentemente la \textbf{stabilità
medio-laterale} del ginocchio.

In particolare il legamento collaterale mediale si oppone al valgo
stress del ginocchio, mentre il legamento collaterale laterale si oppone
al varo stress del ginocchio.

Legamenti che controllano invece la \textbf{stabilità in senso
antero-posteriore e rotazionale} del ginocchio sono rappresentati da:

3. il \textbf{legamento crociato antero esterno}, che origina dalla
spina tibiale anteriore, si porta in alto, indietro e verso l'est e si
va ad inserire sulla faccia mediale del condilo femorale esterno.

4. il \textbf{legamento crociato posteriore}, che origina dalle spine
tibiali posteriori sul piatto tibiale, si porta in alto, in avanti e
verso l'interno e si va ad inserire sulla faccia mediale del condilo
femorale mediale.

Menischi

Oltre ai legamenti a livello dell'articolazione del ginocchio troviamo i
menischi, strutture fibrocartilaginee a sezione triangolare. Si
interpongono tra i condili femorali e i due emipiatti tibiali e
dall'alto hanno la tipica forma di ``C''. Sono all'interno del
ginocchio, ma all'esterno della cavità articolare, poiché rivestiti
dalla membrana sinoviale. I menischi sono costituiti da tessuti
fibro-cartilagineo e privi di vasi nella loro porzione centrale, mentre
la loro porzione periferica, più spessa ed unita alla capsula
articolare, contiene una modesta vascolarizzazione, da cui ne consegue
che solo la parte più esterna dei menischi può eventualmente
cicatrizzare.

Hanno una doppia funzione:

\begin{itemize}
\item
  sono ammortizzatori tra i condili femorali e gli emipiatti tibiali e
  perciò assorbono le forze che si ripercuotono a livello
  dell'articolazione durante il cammino o la corsa.
\item
  aumentano la congruenza articolare tra i condili e gli emipiatti.
  (stabilizzatori passivi)
\end{itemize}

Queste capsule legamentose meniscali possono lesionarsi e rompersi
durante i traumi distorsivi del ginocchio. Le lesioni meniscali possono
essere isolate oppure associate a lesioni legamentose e a loro volta le
lesioni legamentose possono essere isolate o associate ad una lesione
meniscale. Il menisco mediale è più fisso rispetto al menisco laterale,
nel senso che si muove meno nei movimenti di flesso-estensione del
ginocchio, ed anche per questa sua caratteristica, è più soggetto alle
lesioni nei traumi distorsivi del ginocchio.

Assi meccanici e anatomici

Per capire l'esame obiettivo del ginocchio dobbiamo inquadrare
fisiologicamente l'asse dell'arto inferiore. Si parla di asse meccanico
e anatomico.

\includegraphics[width=4.83333in,height=3.51042in]{media/image2.jpeg}

L'\textbf{asse meccanico} è rappresentato da una retta che parte dal
centro della testa del femore e arriva al centro della caviglia. In
condizioni di normalità passerà per il centro del ginocchio o appena
lateralmente dal centro del ginocchio.

L'\textbf{asse anatomico} è una linea che nasce dall'unione di due
rette, una che passa all'interno del canale femorale del femore e arriva
al ginocchio e l'altra che passa all'interno del canale femorale della
tibia e arriva al ginocchio. Queste due rette non sono parallele, ma
incontrandosi formano l'asse anatomico del ginocchio con un angolo di
valgismo fisiologico di circa 173-178 gradi.

Ginocchio Varo

\includegraphics[width=5.23958in,height=3.66667in]{media/image3.jpeg}

Si parla di \textbf{ginocchio varo} nel momento in cui vengono alterati
gli assi meccanici e anatomici.

L'asse meccanico è spostato all'interno del ginocchio e l'asse di
valgismo fisiologico sarà superiore ai 178 gradi considerati normali.

Detto in maniera semplicistica siamo di fronte al ginocchio a parentesi
del calciatore.

Questa condizione implica che le forze di carico che agiscono
sull'articolazione del ginocchio saranno tutte concentrate nella parte
interna del ginocchio stesso. Questo comporterà una maggior usura della
cartilagine e una maggior possibilità di andare incontro a degenerazione
articolare artrosica precoce.

Ginocchio Valgo

\includegraphics[width=4.95833in,height=3.81250in]{media/image4.jpeg}

Discorso opposto si ha per il \textbf{ginocchio valgo} o ginocchio a
``X''.

In questo caso l'asse meccanico passa molto esternamente dal ginocchio e
l'angolo di valgismo fisiologico sarà inferiore rispetto ai 172 gradi
considerati normali. In questa condizione il peso e le forze di carico
che agiscono sul ginocchio saranno più concentrate all'esterno del
ginocchio e con maggior probabilità si potrà andare incontro ad una
degenerazione articolare della componente femoro tibiale esterna.

\emph{Distorsioni e Lussazioni }

Per quanto riguarda i traumi distorsione e lussazione sono due termini
differenti.

\textbf{Lussazione}: è la perdita permanente dei rapporti tra i capi
articolari di un'articolazione. Di solito comporta una manovra di
riduzione per riportare i capi articolari in sede.

\textbf{Distorsione}: è la perdita temporanea dei rapporti tra i capi
articolari di un'articolazione. Non comporta nessuna manovra di
riduzione per riportare i capi articolari in sede.

A livello dell'arto inferiore i traumi distorsivi sono particolarmente
frequenti, ritroviamo soprattutto traumi di caviglia e ginocchio.

Traumi distorsivi del ginocchio

I traumi distorsivi del ginocchio colpiscono generalmente pazienti
giovani sportivi, ma non solo. Gli sport più frequentemente colpiti sono
gli sport di contatto ad alta energia come calcio, football americano,
rugby, sci, basket. A volte sono causati anche da infortuni sul lavoro.

Nei politraumi come negli incidenti stradali questi sono associati ad
altre lesioni, spesso ossee.

Possono dare un'instabilità articolare acuta che può cronicizzare e che
può essere associata a lesioni meniscali

Dal punto di vista patogenetico, i traumi possono essere indiretti, con
forze applicate a distanza dal ginocchio, oppure appoggiati con forza
che agisce direttamente sulla tibia all'altezza del ginocchio (questi
ultimi sono in genere più gravi).

Avvengono generalmente attraverso 4 diversi meccanismi eziopatogenetici.

Trauma in valgismo extrarotazione

\includegraphics[width=4.80208in,height=3.44792in]{media/image5.jpeg}

La forma più frequente è il trauma in valgismo ed extrarotazione.(tipico
trauma da inforcata degli sci). La lesione più lieve determina solo uno
stiramento con ecchimosi della fascia e della capsula articolare.

In questi tipi di traumatismi la prima struttura che cede è il legamento
collaterale mediale.

Se il trauma è ad alta energia può essere interessato anche il menisco
interno, il legamento crociato anteriore e questa associazione è
definita come triade tipica del trauma distorsivo del
ginocchio.(chiamata triade interna) Nei traumi ad altissima energia ci
può essere anche la lesione del crociato posteriore, del menisco esterno
(pentade interna) e di tutti gli altri legamenti fino ad una lussazione
del ginocchio con interessamento delle strutture vascolo nervose del
cavo popliteo.

Trauma in varismo intrarotazione

. \includegraphics[width=4.80208in,height=3.66667in]{media/image6.jpeg}

Meno frequente è invece il trauma in varismo intrarotazione che è
l'opposto di quello precedente. In questo caso la prima struttura che
cede è il legamento collaterale laterale a cui può seguire la lesione
del menisco esterno e una lesione del crociato anteriore. (triade
esterna)

Anche qui se il meccanismo è dotato di altissima energia i crociati
possono rompersi entrambi, può essere anche associata la lesione del
menisco mediale, del collaterale mediale fino ad avere una lussazione
del ginocchio. Possiamo avere in caso di trauma efficiente
interessamento delle strutture del compartimento posteriore e delle
strutture vascolari.

Trauma in iperestensione

\includegraphics[width=4.70833in,height=3.54167in]{media/image7.jpeg}

Altro trauma presente soprattutto negli sport di contatto è il trauma in
iperestensione. può capitare ai calciatori quando saltano di testa e
atterrano male nella discesa, oppure per eventi traumatici diretti.

Trauma in cui la prima struttura ad essere interessata è il legamento
crociato anteriore, a cui può seguire una lesione del crociato
posteriore e poi lo stiramento di tutte le strutture comprese nel cavo
popliteo (quindi lesioni vascolari e nervose associate).

Trauma diretto sulla faccia anteriore della tibia

\includegraphics[width=4.64583in,height=3.47917in]{media/image8.jpeg}

Ultimo trauma è il trauma diretto sulla faccia anteriore della tibia. È
la tipica ``lesione da cruscotto'': quando in macchina si fa un
incidente si punta con i piedi e con il ginocchio si va a sbattere
contro il cruscotto. In questo caso la tibia viene traslata in maniera
acuta posteriormente e anche in questo caso le prime strutture che si
lacerano sono i legamenti crociati (posteriore prima) a cui può seguire
una lesione di tutte le strutture del cavo popliteo posteriore.

Diagnosi e terapia

Le lesioni croniche rappresentano l'esito di uno o più lesioni acute non
riparate, per cui è essenziale riconoscere la qualità delle lesioni
acute e trattarle nel modo più adeguato per garantire la loro
cicatrizzazione ottimale, ed in generale si deve tenere a mente che
cicatrizzano tutte le interruzioni non distanziate o approssimate con
sutura, ad eccezione di quelle del crociato anteriore e della parte
centrale dei menischi.

Esame clinico

\begin{itemize}
\item
  racconto anamnestico riguardo le modalità traumatiche, sensazione di
  "crack",
\end{itemize}

\begin{itemize}
\item
  \textbf{atteggiamento antalgico in semiflessione} con impossibilità
  alla completa estensione,
\item
  \textbf{impotenza funzionale}
\item
  \textbf{tumefazione} ed eventuale presenza di \textbf{versamento
  intrarticolare} (se precoce, entro la prima
\end{itemize}

ora è emartro; se tardivo, dopo 24-36 ore è idrartro)

\begin{itemize}
\item
  \textbf{dolore} diffuso, più o meno accentuato e localizzato
\end{itemize}

Per quanto riguarda la sintomatologia, subito dopo al trauma, occorre
soprattutto distinguere le distorsioni ``benigne'' da quelle gravi: le
prime, infatti, guariscono con semplice riposo o immobilizzazione in
gesso, mentre le seconde richiedono spesso un trattamento chirurgico.

Se dovesse capitarvi di fare i medici sportivi su un campo da gioco e un
atleta si infortuna con una distorsione del ginocchio, dovrete capire
sul campo, se questo atleta può tornare a fare attività sportiva subito

o meno. Lo sportivo, se vuole riprendere subito a giocare probabilmente
ha avuto un trauma distorsivo di

lieve entità; se invece lamenta dolore al movimento, sensazione di
instabilità (si può subito formare anche un

versamento capsulare) e non chiede lui stesso di tornare, dobbiamo
considerarlo un campanello d'allarme,

perché gli sportivi di solito vogliono continuare a giocare (sopratutto
se rugbisti!)

In quest'ultimo caso, si mette il ghiaccio ad intervalli, dopodichè si
applica un bendaggio compressivo e si

accompagna fuori dal campo

Gli indizi che indicano una lesione grave sono l'anamnesi di un trauma
appoggiato, una sensazione immediata di ``crack'' e di lussazione del
ginocchio, l'immediata impotenza funzionale ed il senso di instabilità
permanente dopo il trauma. È quasi costante un versamento articolare
ematico, che può mancare nelle lesioni più lievi, oppure in quelle più
gravi dove l'ampia lacerazione della capsula lascia diffluire l'ematoma
nei tessuti molli circostanti.

L'esame obiettivo è tipicamente ostacolato dal dolore, dalla contrattura
muscolare antalgica, dall'ematoma e dall'edema che infiltrano i tessuti
articolari e peri-articolari. Nelle lesioni più lievi, si può trovare un
dolore palpatorio elettivo sul punto dove è lacerato un legamento
collaterale, altrimenti il dolore è più diffuso.

Esame obiettivo

1. La prima cosa da fare è valutare se c'è \textbf{tumefazione}, oppure
no.

2. Secondo passaggio è il \textbf{test del ballottamento rotuleo}: si
corica il paziente sul lettino, così che il

liquido capsulare si raccolga al di sotto della rotula; dopodiché si
preme la rotula verso il basso con

due dita. Se è presente una raccolta di liquido, questo verrà spinto
contro le pareti laterali, attorno

alla rotula. Questa valutazione deve essere fatta anche nel ginocchio
controlaterale; se non vi è

versamento la rotula non avrà questo movimento.

3. Passaggio successivo, possibilmente in Pronto Soccorso (ambiente
sterile), consiste nel fare

un'\textbf{artrocentesi}, ossia uno svuotamento dell'articolazione del
ginocchio, al fine di determinare anche

le caratteristiche del liquido capsulare.

In un trauma distorsivo di un giovane atleta ci aspettiamo di trovare
del sangue (\textbf{emartro}). A volte

riscontriamo la presenza di gocce di grasso quando il trauma distorsivo
è associato ad una frattura.

Se invece si trova un liquido sinoviale si parla di \textbf{idrartro}.
In alcuni casi si può trovare del pus

all'interno del ginocchio, in questo caso si parla di \textbf{artrite
settica}.

L‟iter diagnostico iniziale quindi è:

trauma distorsivo≥valutazioni in situ≥test del
ballottamento≥artrocentesi≥riscontro di emartro

(più frequente).

Dopo lo svuotamento del ginocchio il paziente starà subito meglio perché
verrà meno la tensione

legata alla tumefazione, successivamente si potranno eseguire test
specifici.

Test specifici

\begin{itemize}
\item
  reperi anatomici e ricerca dei punti dolorosi
\item
  \textbf{valutazione della stabilità articolare}:
\end{itemize}

\begin{quote}
- test per la stabilità \textbf{medio-laterale}: test in valgismo o
abduzione, test in varismo o adduzione;
\end{quote}

-test per la stabilità \textbf{antero-posteriore}: test del cassetto
anteriore, test del cassetto posteriore;

\begin{itemize}
\item
  Jerk test;
\item
  \textbf{Lachman test;}
\item
  Test di gravità
\end{itemize}

L'esame si baserà prima di tutto sulla ricerca dei punti dolorosi:

\begin{itemize}
\item
  si valuterà inizialmente l'emirima articolare mediale, punto doloroso
  del \textbf{menisco mediale};
\item
  si palperà poi, l'emirima articolare laterale, punto doloroso del
  \textbf{menisco laterale};
\item
  infine si palperà il decorso del \textbf{legamento collaterale
  mediale} e \textbf{laterale}.
\end{itemize}

Di solito, il \textbf{LCM} fa male a livello del condilo femorale
mediale, mentre il \textbf{LCL} fa male a livello

della testa del femore.

\begin{itemize}
\item
  Si deve palpare anche l'integrità del \textbf{legamento rotuleo}: se
  si rileva una discontinuità palpatoria del
\end{itemize}

legamento rotuleo il paziente non riuscirà ad estendere completamente il
ginocchio

\begin{itemize}
\item
  . A volte, associata alla lesione del legamento rotuleo, può esserci
  anche la rottura del \textbf{tendine}
\end{itemize}

\textbf{quadricipitale}, ed anche in questo caso il paziente non sarà in
grado di estendere il ginocchio.

Test clinici per valutare stabilità del ginocchio

Test per valutare stabilità laterale:

\begin{itemize}
\item
  \begin{quote}
  Test in Valgismo (o in Abduzione), in cui si valuta l'integrità del
  legamento collaterale mediale; il test prevede di valgizzare il
  ginocchio esteso e flesso, e se è presente una lesione del legamento
  collaterale mediale, il ginocchio si aprirà di più rispetto al contro
  laterale.
  \end{quote}
\item
  \begin{quote}
  Test in Varismo (o in Adduzione), che valuta l'integrità del legamento
  collaterale laterale, che anche in questo caso permette di valutare un
  angolo di apertura superiore al contro laterale.
  \end{quote}
\end{itemize}

Questi due test vengono fatti col ginocchio flesso di circa 20°-30° e
bisogna sempre confrontare i risultati col ginocchio contro laterale.

Abbiamo quindi dei test che sono specifici per la valutazione della
stabilità antero-posteriore:

\begin{itemize}
\item
  \begin{quote}
  Test del Cassetto Anteriore e Posteriore, che serve a dimostrare una
  maggior lassità del ginocchio, appressabile manualmente e visibilmente
  durante l'esecuzione della manovra. Questi test vanno eseguiti
  flettendo il ginocchio a 90°, si blocca la gamba del paziente e poi si
  trasla anteriormente (cassetto anteriore) e posteriormente (cassetto
  posteriore) l'epifisi prossimale tibiale, la quale ``scassetta'' in
  direzione in avanti (lesione del crociato anteriore) o in dietro
  (lesione del crociato posteriore).
  \end{quote}
\item
  \begin{quote}
  Test di Lachman, è un altro test che si fa abitualmente per valutare
  l'integrità dei crociati, e si esegue senza bloccare il piede, con una
  mano si blocca l'estremità distale del femore e con l'altra di trasla
  anteriormente e posteriormente l'estremità prossimale della tibia. La
  sensazione che si riscontra se il paziente ha una lesione dei crociati
  è la mancanza di stop, cioè la mancanza di un blocco palpatorio con
  una maggior lassità dell'articolazione.
  \end{quote}
\end{itemize}

Esami strumentali

\begin{itemize}
\item
  radiografia in proiezioni ortogonali
\item
  ecografia
\item
  TC
\item
  RM
\item
  valutazione artroscopica
\end{itemize}

A questo punto abbiamo un quadro generale di questo paziente: il trauma
che ha avuto, test meniscali

positivi, se c'è stabilità o meno.

La prima cosa da fare, dopo una valutazione clinica accurata, è quella
di richiedere una \textbf{radiografia} che può

mettere in evidenza:

\begin{itemize}
\item
  se ci sono delle fratture associate al piatto tibiale,
\item
  se ci sono dei segni indiretti di lesione legamentosa, ad esempio
  l'avulsione dei legamenti nel punto
\end{itemize}

in cui si inseriscono sull'osso (es. avulsione delle spine tibiali).

Dev'essere ripetuta spostando il ginocchio in più direzioni.

L'\textbf{ecografia} di solito non è indicata per i traumi distorsivi
del ginocchio.

Può essere chiesta una \textbf{TC} qualora si sospetti una frattura non
vista dalle lastre.

Ad ogni modo, l'esame principale per studiare le lesioni
capsulo-legamentose del ginocchio è la \textbf{risonanza magnetica} -
che non è un esame di prima istanza in pronto soccorso - ma viene
richiesto successivamente e normalmente viene eseguito in regime
ambulatoriale Qualora voi facciate i medici di medicina generale, o
comunque medici non specialisti in ortopedia, molto spesso vi capiterà
di richiedere delle RM per

mettere in evidenza delle fratture che non vengono evidenziate nelle
radiografie, dal momento che la RMN è in grado di mettere in evidenza
delle linee di frattura o dell'edema osseo post-traumatico che alle
radiografie non sono visibili.

Terapia

• riposo, bendaggio elastico, crioterapia, FANS;

• Immobilizzazione con apparecchi gessati, tutore;

• Intervento chirurgico in artroscopia;

• intervento chirurgico "a cielo aperto".

Uno dei referti più comuni, soprattutto nel giovane atleta, è la lesione
del \textbf{legamento crociato anteriore}, e il

trattamento consigliato è quello di ricostruire il legamento, ossia di
sostituire il legamento lesionato con uno

ex novo, di solito un legamento autologo prelevato dal proprio corpo.

Sono due i tipi di trapianti che possono essere effettuati:

1. in passato veniva effettuato quasi sempre il \textbf{trapianto da
tendine rotuleo}: si preleva una porzione di

tendine rotuleo dal ginocchio omolaterale alla lesione che viene
riposizionato al posto del crociato

anteriore.

2. Oggi, è di uso comune l'utilizzo dei \textbf{tendini della zampa
d'oca} (gracile e semitendinoso).

In tutti i casi, il trattamento consiste nella sostituzione e nella
ricostruzione del legamento crociato anteriore,

e la sostituzione avviene artroscopicamente.

L'\textbf{artroscopia} è una tecnica chirurgica a minima invasività, che
permette di guardare facilmente diversi tipi di lesioni articolari. Si
posizionano tre cannule attraverso tre mini-incisioni a livello del

ginocchio (Fig. 58). Con la prima incisione viene inserita una cannula
che serve ad iniettare della

soluzione fisiologica; le altre due incisioni si effettuano
rispettivamente in posizione anteromediale

ed antero-laterale: in una va inserita l'ottica, un apparecchio che
serve a far vedere quello che c'è all'interno del ginocchio; nell'altra
incisione invece, si inseriscono degli strumenti particolari, che
serviranno ad effettuare la ricostruzione.

Per prelevare il tendine rotuleo, si asporta anche un frammento d'osso
appartenente sia alla rotula che alla

spina tibiale anteriore, lo si prepara con delle tecniche particolari, e
poi viene impiantato al posto del legamento crociato anteriore e fissato
con delle viti.

Perciò:

• \textbf{Lesione del legamento crociato anteriore} (LCA): ricostruzione
artroscopica, generalmente con

tendine autologo. A volte, nelle lesione recidivanti - soprattutto negli
atleti che magari hanno subito

multipli e ripetuti infortuni ad entrambe le ginocchia - può essere
indicato utilizzare:

- un legamento prelevato da cadavere (con tutte le limitazioni legate ad
un trapianto

eterologo),

- dei legamenti sintetici.

Qualche giorno dopo l'intervento chirurgico il paziente può subito
iniziare la fisioterapia ed il ritorno all'attività sportiva è variabile
dai 4 ai 6 mesi. Le \textbf{lesioni meniscali} eventualmente associate,
vengono trattate durante il medesimo atto chirurgico con delle
\textbf{pulizie artroscopiche}:

- si fa una \textbf{meniscotomia parziale} quando le lesioni meniscali
sono marginali;

- in altri casi, soprattutto nelle plurinserzioni meniscali della
capsula conviene effettuare una

\textbf{sutura meniscale}.

Poiché il menisco ha una sezione triangolare, la parte più alta del
triangolo è quella adesa

alla capsula ed è anche quella più vascolarizzata, infatti si chiama
"\textbf{zona rossa} del menisco";

per cui, le lesioni vicino alla capsula possono guarire se suturate,
mentre la parte più verso

l'articolazione non è vascolarizzata ("\textbf{zona bianca} del
menisco") per cui può essere ripulita.

• Le \textbf{lesioni del collaterale mediale} (LCM) di solito non
vengono trattate chirurgicamente, guariscono

con un iniziale periodo di immobilizzazione associato a della
fisioterapia e alla terapia fisica. Questo

perché il LCM è un legamento nastriforme e, quando si rompe, i capi si
riavvicinano, e ha un'ottima

possibilità di cicatrizzazione spontanea.

• Il \textbf{legamento collaterale laterale} (LCL), se ha lesioni
importanti, deve essere riparato

chirurgicamente dal momento che è un legamento cordoniforme, per cui in
caso di rottura i due capi

non si riavvicineranno mai.

• Le \textbf{lesioni del crociato posteriore} (LCP) possono anche non
essere operate, ad es. ci sono molti

calciatori che giocano con il crociato posteriore rotto. In questo caso,
la lesione è vicariata da un

ottimo tono-trofismo muscolare, anche se esistono delle tecniche
chirurgiche artroscopiche o a cielo

aperto che ne consentono la ricostruzione.

Nelle lesioni più lievi, limitate alla capsula, è sufficiente
un'artrocentesi per estrarre il versamento ematico ed
un'immobilizzazione gessata, e questa terapia è indicata anche nelle
lesioni di media gravità (triadi) se il paziente è anziano. Nelle
lesioni più gravi e nei pazienti giovani è invece indicata la
riparazione chirurgica immediata, ed in generale i risultati sono buoni,
ma dipendono dalla gravità delle lesioni, anche se in quelle gravi
permane una certa lassità e non è possibile riprendere gli sports
agonistici.

Lesioni meniscali

\includegraphics[width=4.88542in,height=3.48958in]{media/image9.jpeg}

Esistono poi le lesioni meniscali isolate o associate. Possono essere di
2 tipi:

\begin{itemize}
\item
  Le \textbf{lesioni traumatiche} sono di solito lesioni nette e vengono
  classificate in base al decorso della lesione: radiale, longitudinale,
  obliqua.
\end{itemize}

Le lesioni traumatiche dei menischi sono condizioni relativamente
frequenti, e la lesione del menisco mediale è circa 7 volte più
frequente di quella del menisco laterale, il quale viene lesionato o
alterato in genere o come conseguenza di malformazioni congenite
(menisco discoide) o per processi di degenerazione mixoido-cistica del
menisco stesso. In generale, comunque, le lesioni meniscali sono più
frequenti nel sesso maschile e nell'età adulta giovane, in particolare
negli sportivi, mentre sono rare nei bambini, fatta eccezione per le
malformazioni congenite dei menischi.

Dal punto di vista patogenetico, la lesione traumatica di un menisco è
sempre causata da un trauma distorsivo, mai da un trauma diretto.
Condizioni predisponenti alla lesione meniscale sono i fenomeni
degenerativi del menisco, un pregresso trauma, che può aver determinato
una lassità delle inserzioni meniscali o una piccola lacerazione del
menisco. Durante il trauma distorsivo, il menisco viene rimane
schiacciato tra il piatto tibiale ed il condilo femorale e non può
seguire la tibia, per cui si fissura o si strappa dalle sue inserzioni
tibiali o capsulari. Per quanto riguarda l'anatomia patologica, le
lesioni traumatiche meniscali, in ordine di frequenza, sono le seguenti:

\begin{itemize}
\item
  \begin{quote}
  Rottura Longitudinale, ``a manico di secchia'';
  \end{quote}
\item
  \begin{quote}
  Strappo del Corno Posteriore;
  \end{quote}
\item
  \begin{quote}
  Disinserzione del menisco dalla capsula;
  \end{quote}
\item
  \begin{quote}
  Strappo del Corno Anteriore;
  \end{quote}
\item
  \begin{quote}
  Rottura Trasversale.
  \end{quote}
\end{itemize}

\begin{itemize}
\item
  Le \textbf{lesioni degenerative} invece non conseguono traumi
  distorsivi importanti, non danno rotture nette, ma danno una sorta di
  \emph{sfrangiamento} della superficie del menisco. Formano una sorta
  di ``pelucchi'' che sfioccano dalla superficie del menisco stesso.
  Dopo i 40 anni, a livello meniscale cominciano delle modificazioni
  artrosiche, con progressiva disidratazione della sua fibro-cartilagine
  che diventa meno elastica e compare una tipica fessurazione
  orizzontale, come una specie di ``slaminamento'' da usura.
\end{itemize}

Molto spesso nei traumi importanti il frammento del menisco si può
ribaltare all'interno della gola femorale e questa lesione si chiama ``a
manico di secchio". Può determinare un blocco articolare tra femore e
tibia.

Diagnosi e terapia

Dal punto di vista clinico, la diagnosi di lesione meniscale si basa
sull'anamnesi e su di un accurato esame clinico, mentre i radiogrammi
non sono utili, e vengono in genere eseguiti solo per escludere lesioni
diverse da quella meniscale, mentre più utili sono l'artrografia, oggi
poco usata, e l'artroscopia.

L'anamnesi rivela costantemente il trauma discorsivo, associato in
genere ad altri elementi:

\begin{itemize}
\item
  Episodi di Blocco Articolare, infatti il paziente riferisce spesso che
  il ginocchio, una o più volte, si è ``bloccato'', quasi sempre in
  flessione, e tale blocco è causato da una lussazione di parte del
  menisco lacerato, verso il centro dell'articolazione, così che si va
  ad interporre tra le superfici articolari e ne impedisce la completa
  escursione. Il blocco spesso può essere risolto con opportune
  manipolazioni del ginocchio, che talvolta il paziente esegue da sé.
\item
  Senso di Cedimento del Ginocchio, che il soggetto avverte sotto carico
  e specialmente sotto sforzo.
\item
  Sensazione di Scroscio o di avere un corpo che si sposta all'interno
  dell'articolazione durante i movimenti del ginocchio.
\end{itemize}

L'esame obiettivo dimostra:

\begin{itemize}
\item
  dolore esattamente localizzato sull'emirima articolare mediale (se è
  interessato il menisco mediale) o laterale (se invece è interessato il
  menisco laterale), e questo segno è pressoché costante. Il dolore,
  nella stessa sede, si accentua forzando il ginocchio in estensione e
  ruotando la gamba, a ginocchio flesso verso l'esterno (per il menisco
  mediale) o verso l'interno (per le lesioni del menisco laterale); ed
  il dolore si accentua anche forzando il ginocchio, esteso, in varismo
  (per il menisco mediale) o in valgismo (per il menisco laterale),
  poiché queste manovre causano compressione del menisco leso.
\item
  Talora il ginocchio non può essere esteso completamente, vale a dire
  che è in fase di blocco, e questo segno, anche se appena accennato, ha
  valore diagnostico di certezza.
\item
  Talvolta si nota un lieve versamento intra-articolare ed inspessimento
  della membrana sinoviale, che reagisce allo stimolo meccanico
  rappresentato dalla fluttuazione del menisco rotto.
\item
  Se la lesione è già datata, è spesso presente ipotrofia del muscolo
  quadricipite femorale.
\end{itemize}

Utile è anche l'esecuzione di alcuni test meniscali puri, in cui il
menisco provoca dolore nel momento in cui si iperestende o iperflettendo
il ginocchio, e a seconda della localizzazione del dolore, mediale
piuttosto che laterale, si può individuare l'eventuale lesione
meniscale. Tra questi va ricordato

\begin{itemize}
\item
  il test di Apley, in cui si mette il paziente prono e si piega il
  ginocchio a 90°, si prende la pianta del piede, si schiaccia contro al
  lettino e si fanno dei movimenti di intra/extra-rotazione del piede.
  Tutto questo determina una compressione del femore sulla tibia, cui
  consegue uno schiacciamento dei menischi che, se rotti, fanno male.
\end{itemize}

Terapia ed Esiti delle Lesioni Meniscali

La maggior parte delle lesioni meniscali si trattano in artroscopia, ed
occorre sempre valutare globalmente le condizioni del ginocchio. Se, ad
esempio, una rottura del menisco mediale si associa ad interruzione del
crociato anteriore è controindicato asportare il menisco senza
ricostruire il crociato, perché così si evidenzierebbe o aggraverebbe
l'instabilità articolare. Nelle persone anziane, spesso di sesso
femminile, se non vi è stato un vero e proprio trauma discorsivo, se il
menisco non è chiaramente rotto, è molto probabile che il dolore sia
dovuto ad un'iniziale artrosi femoro-tibiale mediale, e l'escissione del
menisco, in tale condizione, non farebbe altro che peggiorare la
condizione.

Regola fondamentale nella terapia è che la funzione del menisco deve
essere rispettata: se la lesione consiste nel distacco del menisco della
capsula o in una lacerazione della parte periferica del menisco, queste
vengono suturate con filo riassorbibile, mentre le lacerazioni più
centrali vengono trattate in artroscopia, rimuovendo la parte di menisco
distaccata, lussata o fluttuante, e regolarizzando il bordo del menisco
residuo, che viene conservato.

L'intervento in artroscopia ha il vantaggio di evitare o ridurre al
minimo l'ospedalizzazione e di permettere una ripresa funzionale quasi
immediata.

Instabilità Croniche

Quando la lesione non è ben guarita, inoltre, permane una certa
instabilità cronica, che può manifestarsi con dolori che simulano una
sindrome meniscale, e in questi casi sarebbe un grave errore asportare
il menisco, poiché quest'operazione aggraverebbe la stabilità, che viene
avvertita soprattutto nello scendere le scale, nel discendere per un
terreno accidentato, nella corsa o sotto sforzo atletico. Le instabilità
croniche possono essere generalmente suddivise in diversi tipi:

\begin{itemize}
\item
  \begin{quote}
  Instabilità Anteriori, con cassetto anteriore, dovuto ad interruzione
  del crociato anteriore;
  \end{quote}
\item
  \begin{quote}
  Instabilità in valgo-rotazione esterna;
  \end{quote}
\item
  \begin{quote}
  Instabilità in varo-rotazione interna;
  \end{quote}
\item
  \begin{quote}
  Instabilità Posteriori, da interruzione del crociato posteriore.
  \end{quote}
\end{itemize}

Per quanto riguarda la terapia e gli esiti del trattamento delle
instabilità croniche, l'indicazione è difficile, ed il trattamento
raramente è perfetto, e la lesione meniscale, spesso concomitante, va
trattata a parte, ma tenendo sempre a mente che l'assenza di un menisco
destabilizzerebbe ulteriormente il ginocchio. In generale, il crociato
anteriore viene ricostruito con innesto prelevato dal tendine rotuleo o
con un impianto sintetico, mentre la mancanza del crociato posteriore
può essere in parte compensata retro ponendo sul piatto tibiale una
parte del tendine rotuleo, mentre la lassità della capsula mediale o
laterale può essere corretta con plastiche passive (della capsula
stessa) o attive (dei tendini).

\end{document}
