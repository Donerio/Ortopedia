\documentclass[]{article}
\usepackage{lmodern}
\usepackage{amssymb,amsmath}
\usepackage{ifxetex,ifluatex}
\usepackage{fixltx2e} % provides \textsubscript
\ifnum 0\ifxetex 1\fi\ifluatex 1\fi=0 % if pdftex
  \usepackage[T1]{fontenc}
  \usepackage[utf8]{inputenc}
\else % if luatex or xelatex
  \ifxetex
    \usepackage{mathspec}
  \else
    \usepackage{fontspec}
  \fi
  \defaultfontfeatures{Ligatures=TeX,Scale=MatchLowercase}
\fi
% use upquote if available, for straight quotes in verbatim environments
\IfFileExists{upquote.sty}{\usepackage{upquote}}{}
% use microtype if available
\IfFileExists{microtype.sty}{%
\usepackage{microtype}
\UseMicrotypeSet[protrusion]{basicmath} % disable protrusion for tt fonts
}{}
\usepackage[unicode=true]{hyperref}
\hypersetup{
            pdfborder={0 0 0},
            breaklinks=true}
\urlstyle{same}  % don't use monospace font for urls
\usepackage{graphicx,grffile}
\makeatletter
\def\maxwidth{\ifdim\Gin@nat@width>\linewidth\linewidth\else\Gin@nat@width\fi}
\def\maxheight{\ifdim\Gin@nat@height>\textheight\textheight\else\Gin@nat@height\fi}
\makeatother
% Scale images if necessary, so that they will not overflow the page
% margins by default, and it is still possible to overwrite the defaults
% using explicit options in \includegraphics[width, height, ...]{}
\setkeys{Gin}{width=\maxwidth,height=\maxheight,keepaspectratio}
\IfFileExists{parskip.sty}{%
\usepackage{parskip}
}{% else
\setlength{\parindent}{0pt}
\setlength{\parskip}{6pt plus 2pt minus 1pt}
}
\setlength{\emergencystretch}{3em}  % prevent overfull lines
\providecommand{\tightlist}{%
  \setlength{\itemsep}{0pt}\setlength{\parskip}{0pt}}
\setcounter{secnumdepth}{0}
% Redefines (sub)paragraphs to behave more like sections
\ifx\paragraph\undefined\else
\let\oldparagraph\paragraph
\renewcommand{\paragraph}[1]{\oldparagraph{#1}\mbox{}}
\fi
\ifx\subparagraph\undefined\else
\let\oldsubparagraph\subparagraph
\renewcommand{\subparagraph}[1]{\oldsubparagraph{#1}\mbox{}}
\fi

% set default figure placement to htbp
\makeatletter
\def\fps@figure{htbp}
\makeatother


\date{}

\begin{document}

\begin{quote}
\emph{\textbf{Traumatologia}}
\end{quote}

\emph{Generalità delle fratture}

Introduzione

\protect\hypertarget{_bookmark1}{}{\protect\hypertarget{_bookmark2}{}{}}\emph{La
pratica ortopedica risale a epoche lontane eppure la parola "ortopedia"
viene usata solo dal 1741: fu coniata dal medico francese Nicolas Andry,
a partire da due parole greche (orthòs: diritto; pàis: bambino) perché
aveva come obiettivo quello di correggere le deformità del fisico nei
bambini.}

\emph{Questa disciplina si è sviluppata soprattutto durante le guerre
che fornirono molti casi traumatologici su cui "sperimentare".}

\emph{Altri eventi che hanno contribuito allo sviluppo di tale
disciplina sono stati l'avvento delle moderne tecniche anestesiologiche,
della radiologia e degli antibiotici.}

\emph{Oggi la traumatologia e l'ortopedia sono sostenute in maniera
significativa dalla tecnologia. Per rendersi conto dell'importante
contributo di quest'ultima alla pratica ortopedica basti ricordare che i
materiali prima usati su aerei militari e aerei civili, sono ora parte
delle protesi usate in ortopedia.}

\emph{Grandi speranze sono riposte nella biotecnologia, nelle ricerca
sulle cellule staminali e sui fattori di crescita che in parte sono già
utilizzati, ma di cui ci si augura di poter sfruttare al massimo le
potenzialità nel futuro. Noi (futuri) medici abbiamo il dovere di
informarci su questi nuovi orizzonti per dare al paziente consigli
giusti e per offrire la migliore soluzione sul piano terapeutico.}

Definizione e trattamento

Una frattura è definita come una \emph{soluzione di continuità di un
segmento osseo} conseguente ad un trauma che agisce sul segmento stesso.
La forza e l'energia meccanica del trauma (perché si abbia tale
frattura) devono essere di intensità tale da superare i limiti di
deformabilità e resistenza del segmento osseo colpito così da
determinarsi la frattura. (Ad esempio, un motociclista che cade e urta
contro un palo è quasi certo che avrà una frattura).

\emph{Le fratture possono essere suddivise in:}

\begin{itemize}
\item
  \emph{Traumatiche}
\item
  \emph{Patologiche, e ``da fragilità''}
\item
  \emph{Da stress (o da durata)}
\item
  \emph{Iatrogene (osteotomie)}
\end{itemize}

Nell'immagine: radiografia della gamba che mostra un segmento osseo
discontinuo. Si tratta di una frattura del terzo distale della tibia e
del terzo distale del perone.

\includegraphics[width=2.21597in,height=2.09167in]{media/image1.jpeg}

Diagnosi ( questa parte è stata ripresa e approfondita nella lezione del
08-03 )

\emph{Per la diagnosi di frattura vanno sempre considerati più aspetti.}

\emph{\textbf{Segni di probabilità:}}

\begin{itemize}
\item
  \emph{Atteggiamento: un paziente con un arto (o l‟intero corpo) in
  posizione antalgica può far sospettare una frattura; ad esempio
  l'atteggiamento tipico delle fratture del collo del femore: l'arto è
  extra ruotato, addotto e accorciato.}
\item
  \emph{Deformità grossolane, ad esempio la gamba in un senso e il piede
  nell'altro.}
\item
  \emph{Lesioni cutanee e tumefazioni.}
\item
  \emph{Impotenza funzionale.}
\item
  \emph{Dolore.}
\end{itemize}

\emph{\textbf{Segni di certezza:}}

\begin{itemize}
\item
  \emph{Crepitio: la cauta mobilizzazione del segmento evoca il rumore
  delle superfici ossee a confronto.}
\item
  \emph{Motilità preternaturale.}
\end{itemize}

\emph{Per la \textbf{diagnosi definitiva} ci si avvale delle techiche
radiologiche: inizialmente radiografica, eventualmente TC o RM (se
dovessero permanere dubbi o in caso si sospettino lesioni ai tessuti
circostanti).}

Nel momento in cui si ha un sospetto di frattura occorre procedere come
di seguito:

\begin{itemize}
\item
  Immobilizzazione provvisoria sul luogo dell'incidente
\item
  Radiografia convenzionale (che permette la diagnosi di frattura).
\end{itemize}

\begin{quote}
N.B. La radiografia permette anche di definire e analizzare le
caratteristiche di tale frattura
\end{quote}

\begin{itemize}
\item
  Trazione se necessario
\end{itemize}

Trattamento

\begin{quote}
Una volta fatta diagnosi di frattura inizia il processo di
\textbf{riparazione della frattura}. \emph{Il trattamento delle frattura
ha come scopo il recupero funzionale completo del segmento fratturato
senza deformità residue o alterazioni significative della morfologia
scheletrica e con il pieno recupero della funzione muscolare e
articolare.}
\end{quote}

Ci sono due aspetti da considerare e che garantiscono la guarigione di
una frattura:

\begin{itemize}
\item
  \textbf{Riduzione della frattura}: si tratta del complesso di manovre
  messe in atto nel caso di frattura scomposta o fratture-lussazioni al
  fine di riposizionare i segmenti ossei nella corretta posizione
  anatomica. Tale manovra viene può essere eseguita dall'ortopedico
  manualmente ed è appunto detta manuale oppure essere eseguita
  chirurgicamente. A seguito di questo bisogna procedere con la
  stabilizzazione.
\end{itemize}

\begin{itemize}
\item
  \textbf{Stabilizzazione} dei vari elementi della frattura. Si attua
  per mezzo di un apparecchio gessato o chirurgicamente attraverso
  l'utilizzo di diversi mezzi di sintesi in genere mezzi metallici
  tramite cui si fissa la frattura una volta che è stata ridotta. Tali
  mezzi di sintesi comprendono viti metalliche, placche associate a
  viti, chiodi endomidollari, fissatori esterni, fili metallici che
  passano attraverso la cute e bloccano la frattura.
\end{itemize}

In base alle modalità per cui si opta distinguiamo un
\textbf{\emph{trattamento conservativo}} e uno \textbf{\emph{chirurgico
(osteosintesi)}}

\textbf{\emph{Trattamento conservativo}}

\emph{Può essere così schematizzato:}

\begin{itemize}
\item
  \emph{\textbf{Immobilizzazione} generalmente durante il trasporto in
  Pronto Soccorso, }
\item
  \emph{\textbf{Riduzione}: \textbf{manuale} oppure mediante
  \textbf{trazione progressiva}. In questo ultimo caso mediante
  applicazione ai capi ossei di un bendaggio adesivo (\textbf{trazione a
  cerotto}) oppure mediante un \textbf{filo transcheletrico} che
  permette di applicare fino a 14-15 Kg di peso.}
\item
  \emph{\textbf{Stabilizzazione} (o contenzione): tramite apparecchi
  gessati (classico), gessi funzionali interrotti laddove ci sono
  articolazioni (utili nel ginocchio) o gessi sintetici (che sono però
  poco plasmabili).}
\end{itemize}

\textbf{\emph{Trattamento chirurgico (osteosintesi)}}

\emph{Comporta l'esposizione del focolaio di frattura e la
\textbf{riduzione} viene eseguita a cielo aperto.}

\begin{quote}
\emph{Per la \textbf{stabilizzazione} dei frammenti si può eseguire}
\end{quote}

\begin{itemize}
\item
  \emph{\textbf{Fissazione interna}: si applicano, tramite piccolo
  taglio, svariati mezzi di osteosintesi costituiti da fili, viti
  libere, placche e viti (Nell' immagine in alto a dx: viti libere e
  placca e viti).}
\item
  \includegraphics[width=1.24861in,height=6.04167in]{media/image2.jpeg}\emph{\textbf{Fissazione
  esterna}: si applicano viti o chiodi stabilizzati tra loro mediante un
  fissatore esterno che permette anche la riduzione della frattura.
  Viene preferita quando si ha un grosso rischio di infezione o in caso
  di fratture in determinate sedi: bacino, tibia e avambraccio
  (Nell'immagine: fissatore esterno).}
\item
  \emph{\textbf{Sintesi endomidollari}: è la più attuale! Sfrutta
  infibuli intra midollari che stabilizzano le fratture dall'interno.
  Questa metodica sfrutta la riparazione biologica col callo osseo
  periosteo diversamente dalla riparazione interna in cui si attua
  un'interruzione del processo ripartivo (Nell'immagine: chiodo Gamma).}
\end{itemize}

\emph{In determinati casi le fratture non sono riparabili o meglio non
conviene ripararle. Questo succede soprattutto negli anziani perché
l'allettamento e la riabilitazione determinerebbero un crollo
psicofisico ben peggiore della frattura stessa (sindrome da
allettamento).}

A questo punto il professore mostra un video di un intervento chirurgico
di riduzione di una frattura e stabilizzazione chirurgica della stessa.
(Alcuni concetti precedenti vengono ripresi) Si tratta di una frattura
del terzo distale della tibia. Viene mostrata la radiografia che
conferma il sospetto diagnostico: si tratta di una frattura spiroide
scomposta (tipica frattura da torsione) in cui i due segmenti ossei non
sono a contatto tra di loro. La frattura è stata trattata in sala
operatoria sotto controllo endoscopico e per l'immobilizzazione la
scelta è stata quella della sintesi con chiodo endomidollare.

\includegraphics[width=2.45694in,height=2.16736in]{media/image8.jpeg}La
manovra di riduzione molto spesso viene eseguita sotto controllo dei
raggi e si ricorre a posizioni particolari che facilitano la manovra
stessa di riduzione e il posizionamento del chiodo (in questo caso il
paziente è disteso sul lettino con le gambe sollevate). Molto spesso di
fronte a una frattura delle ossa lunghe il paziente viene messo in
\emph{trazione trans-scheletrica}: metodologia attuata per ridurre la
frattura e generalmente applicata in pronto soccorso in anestesia
locale. Lo scopo di tale manovra è quello di \emph{trazionare
progressivamente la frattura in vari segmenti corporei}: vengono
progressivamente sgranati i frammenti della frattura e tesi i tessuti
molli al fine di ridurre il gonfiore. Si tratta a tutti gli effetti di
una parziale riduzione che verrà completata in seguito chirurgicamente.
Ha il vantaggio di agevolare la fase chirurgica stessa. In questo caso è
stato posizionato il filo di trazione al calcagno e al letto vengono
applicati pesi in kg che garantiscano la trazione.

Tali pesi devono essere pari al 5-10\% del peso corporeo per cui ad
esempio per una frattura scomposta di femore in un soggetto di 70-80 kg
si utilizzeranno pesi di 3,5/7 Kg. Importante ricordare i rischi legati
alla procedura stessa: un'eccessiva trazione può determinare eccessivo
stiramento e parestesie o formicolio degli arti che devono essere
necessariamente risolte.

A questo punto viene spiegata la procedura eseguita.. Attraverso
proiezioni antero-posteriori e laterali il chirurgo controlla la fase di
inserimento e procede alla riduzione della frattura. Questa manovra può
essere realizzata con l'ausilio di pinze o attraverso la cute previa
incisione per raggiungere il focolaio della frattura. Nel video mostrato
il chirurgo procede con incisone cutanea tramite la quale accede al
focolaio di frattura. A seguito della riduzione si procede con la
\emph{stabilizzazione} che nel caso mostrato viene realizzata attraverso
l'inserimento di un chiodo endomidollare che viene bloccato sopra e
sotto la frattura per dare stabilità. I chiodi sono cannulati, con buchi
all'interno e necessitano dell'inserimento di un filo guida nel canale
midollare. Prima dell'inserimento del chiodo, si fresa il canale
midollare di uno o due mm in più del diametro del chiodo scelto, manovra
chiamata \textbf{ALESAGGIO DEL CANALE MIDOLLARE}.

\includegraphics[width=2.71875in,height=1.94306in]{media/image9.jpeg}NB:
Il chiodo a battuta viene posizionato solo se in tutte le proiezioni la
frattura è riportata in posizione corretta! La punta del chiodo supera
il focolaio di frattura e una volta posizionato con precisione verrà
bloccato prossimalmente e distalmente con delle viti. Nel chiodo ci sono
fori per le viti che posizionate perpendicolarmente vanno a bloccare il
chiodo sotto la frattura e in alto, sopra la stessa.

PER RIASSUMERE:

- Diagnosi radiografica

- Riduzione della frattura manuale o chirurgica (a cielo aperto con
incisione o percutaneo con pinze)

- Stabilizzazione della frattura: attraverso tutori, gesso, mezzi di
sintesi inseriti chirurgicamente per favorire la guarigione.

La maggior parte delle fratture viene oggi trattata chirurgicamente,
soprattutto se interessano le ossa lunghe. La chirurgia permette una
riduzione migliore e una riabilitazione più rapida per cui minor
incidenza di complicanze legate all'allettamento delle persone anziane e
di complicanze locali come rigidità articolari. Il lavoro
dell'ortopedico deve sempre andare di pari passo al lavoro del
fisiatra/fisioterapista per ottenere il pieno recupero funzionale.

\emph{Informazioni sul corso, tirocinio, modalità di svolgimento
dell'esame:}

\emph{Ortopedia e traumatologia: 3 cfu }

\emph{Medicina fisica e riabilitazione: 2 cfu }

\emph{TIROCINIO: Il tirocinio ha una durata di una settimana (Lun- Ven).
Sul sito è presente l'elenco delle settimane disponibili. Non più
disponibile reumatologia. Si viene suddivisi al mattino nelle varie
postazioni: ambulatorio traumi, pronto soccorso, medicina fisica e
riabilitazione.}

\emph{A tirocinio ogni giorno c'è un modulo in cui far apporre timbro e
firma del tutor, a fine settimana bisogna lasciare il libretto in
Segreteria (secondo piano Ortopedia, signora Giovanna). Ricordarsi di
presentare il libretto firmato all'esame!!}

\emph{In sala operatoria si può eccedere solo se interessati, è
necessario fare richiesta. C'è la possibilità di fare tirocinio a
Fidenza o Piacenza, comunicarlo per tempo al prof Coordinatore del corso
(prof. Pogliacomi).}

\emph{Se interessati alla materia si può frequentare il reparto.
Possibilmente iscriversi il prima possibile ai tirocini, in vicinanza
dell'esame. \emph{NB: Da sostenere entro l'anno accademico!} Le date dei
tirocini sono tutte sul sito, iscrizione online.}

\emph{ESAME: orale, iscrizione online e registrazione online. L'esame si
compone di 3 domande, riguardanti le materie del corso (una di
ortopedia, una di traumatologia, una di medicina fisica e
riabilitazione). Un prof può chiedere tutte e tre le domande o solo una:
è variabile! Il voto finale è costituito dalla media dei tre voti.
\emph{Tutte e tre le domande devono essere sufficienti per passare
l'esame}.}

\emph{I testi consigliati si trovano sul sito del corso.}

\end{document}
