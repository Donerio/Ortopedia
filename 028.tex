\documentclass[]{article}
\usepackage{lmodern}
\usepackage{amssymb,amsmath}
\usepackage{ifxetex,ifluatex}
\usepackage{fixltx2e} % provides \textsubscript
\ifnum 0\ifxetex 1\fi\ifluatex 1\fi=0 % if pdftex
  \usepackage[T1]{fontenc}
  \usepackage[utf8]{inputenc}
\else % if luatex or xelatex
  \ifxetex
    \usepackage{mathspec}
  \else
    \usepackage{fontspec}
  \fi
  \defaultfontfeatures{Ligatures=TeX,Scale=MatchLowercase}
\fi
% use upquote if available, for straight quotes in verbatim environments
\IfFileExists{upquote.sty}{\usepackage{upquote}}{}
% use microtype if available
\IfFileExists{microtype.sty}{%
\usepackage{microtype}
\UseMicrotypeSet[protrusion]{basicmath} % disable protrusion for tt fonts
}{}
\usepackage[unicode=true]{hyperref}
\hypersetup{
            pdfborder={0 0 0},
            breaklinks=true}
\urlstyle{same}  % don't use monospace font for urls
\IfFileExists{parskip.sty}{%
\usepackage{parskip}
}{% else
\setlength{\parindent}{0pt}
\setlength{\parskip}{6pt plus 2pt minus 1pt}
}
\setlength{\emergencystretch}{3em}  % prevent overfull lines
\providecommand{\tightlist}{%
  \setlength{\itemsep}{0pt}\setlength{\parskip}{0pt}}
\setcounter{secnumdepth}{0}
% Redefines (sub)paragraphs to behave more like sections
\ifx\paragraph\undefined\else
\let\oldparagraph\paragraph
\renewcommand{\paragraph}[1]{\oldparagraph{#1}\mbox{}}
\fi
\ifx\subparagraph\undefined\else
\let\oldsubparagraph\subparagraph
\renewcommand{\subparagraph}[1]{\oldsubparagraph{#1}\mbox{}}
\fi

% set default figure placement to htbp
\makeatletter
\def\fps@figure{htbp}
\makeatother


\date{}

\begin{document}

\emph{\textbf{Massoterapia}}

\emph{Complesso di manipolazioni che si praticano sui tessuti molli a
fini terapeutici, allo scopo di migliorare la circolazione arteriosa e
intervenire sull'apparato muscolare}.

Per molti la massoterapia viene considerata una chinesiterapia (terapia
del movimento): il massaggio non crea movimento, ma è una terapia
manuale che viene spesso adoperata per preparare alla chinesiterapia.

Si tratta di manipolazioni dei tessuti molli che hanno effetto sia
\textbf{locale} (cute e tessuti sottostanti) ma anche \textbf{a
distanza}, su diversi organi e apparati.

Sono state sviluppate tante scuole di pensiero e tante tecniche come il
\emph{massaggio connettivale, il massaggio trasversale profondo, la
digito-pressione} e tante altre (vedi oltre).

Non può essere utilizzato per aumentare la forza, ma per riattivare la
circolazione locale e fare in modo che il muscolo possa rispondere ad
esempio in tutte quelle condizioni in cui il paziente è stato costretto
ad una immobilizzazione particolarmente prolungata. \emph{Non va
utilizzata nei caso di ipotonia muscolare.}

Molti atleti fanno uso del massaggio come abitudine prima di una gara
per sciogliere la tensione e dare un certo rilassamento muscolare oppure
alla fine della gara per riassorbire le tossine che sono state prodotte
durante lo sforzo fisico.

Si tratta di una disciplina ampiamente diffusa già in passato: le prime
documentazioni risalgono a 3000 anni fa in Cina. Sono stati ampiamente
utilizzati durante i giochi ellenici in cui veniva praticata la frizione
verso l'alto. Questo tipo di considerazione che è stata tramandata dai
greci è stata poi utilizzata nella pratica clinica; infatti \emph{il
massaggio deve sempre essere eseguito dalla periferia alla zona
dolorosa, mai direttamente sulla zona dolorosa.}

Essendo il massaggio una manovra manuale ha un effetto:

\begin{itemize}
\item
  diretto o meccanico, rivolta all'apparato muscolare, ai vasi
  sanguigni, alle terminazioni nervose, alla cute
\item
  azione riflessa o indiretta su numerosi organi e apparati, che avviene
  tramite il sistema nervoso centrale e periferico
\end{itemize}

EFFETTI BIOLOGICI

Sono molto numerosi e sono tutti sfruttati in molte situazioni
patologiche:

\begin{enumerate}
\def\labelenumi{\arabic{enumi}.}
\item
  azione diretta o meccanica:
\end{enumerate}

\begin{itemize}
\item
  \textbf{elasticizzare la cute}, renderla più morbida;
\item
  \textbf{eliminare secrezioni} in condizioni in cui ci sono
  infiammazioni delle borse;
\item
  \textbf{migliorare la circolazione locale di cute, muscoli e
  articolazioni sottostanti} e quindi favorire sia il drenaggio venoso e
  linfatico se ci sono infiammazioni locali sia l'aumento
  dell'irrorazione arteriosa e quindi l'ossigenazione cellulare;
\item
  \textbf{effetto sedativo} cosi come può avere un effetto
  eccito-motorio a seconda dell'intensità del massaggio e a seconda
  della tecnica utilizzata;
\item
  oltre ad un effetto sui grandi vasi, ha anche un importante
  \textbf{effetto su capillari e arteriole}. Si tratta chiaramente di
  differenze molto sottili e quindi molto dipendenti dalla capacità
  dell'operatore di poter agire a diversi livelli con diverse tecniche e
  a diverse pressioni;
\item
  se praticato con intensità tende ad avere degli \textbf{effetti anche
  sui tessuti profondi}: c'è un effetto sulla produzione di istamina;
\item
  \textbf{effetto algogeno locale} perché viene stimolata la produzione
  di sostanze algogene;
\item
  \textbf{effetto antalgico e analgesico} attraverso la stimolazione dei
  meccanocettori cutanei, tendinei e fasciali
\item
  \textbf{azione meccanica sui tendini e sui legamenti} con liberazione
  da involucri cicatriziali e aderenze capaci di limitare l'escursione
  articolare;
\item
  \begin{quote}
  \textbf{riassorbimento di versamenti} nelle borse sierose e nelle
  cavità articolari perché operando proprio in modo centrifugo c'è un
  effetto di drenaggio e quindi favorisce la liberazione ed eliminazione
  di queste sostanze;
  \end{quote}
\item
  \begin{quote}
  sull'apparato muscolare: \textbf{elimina i cataboliti acidi} (acido
  lattico) attraverso l'azione sulla microcircolazione ma anche su vasi
  venosi e su quelli linfatici. In questo modo libera più rapidamente
  dalla sensazione dolorosa di stanchezza e permette un più rapido
  recupero; ha un effetto sul tono e le proprietà contrattili del
  muscolo; effetti anche sulla \textbf{temperatura corporea} (se
  praticato con una certa intensità può naturalmente aumentare la
  temperatura corporea), sull'apparato digerente, sulle vie urinarie.
  \end{quote}
\end{itemize}

\begin{enumerate}
\def\labelenumi{\arabic{enumi}.}
\item
  Azione indiretta o riflessa
\end{enumerate}

\begin{itemize}
\item
  Iperemia attiva attraverso la stimolazione dei nervi vasomotori
\item
  Stimolazione degli esterocettori e propriocettori → risposta dei
  sistemi neuromotorio e neurovegetativo
\end{itemize}

\begin{itemize}
\item
  \begin{quote}
  azione trofica muscolare e miglioramento dell'eccitabilità delle fibre
  per azione riflessa sulle fibre massaggiate.
  \end{quote}
\end{itemize}

\begin{itemize}
\item
  Ha un'influenza sul ricambio generale dell'organismo attraverso le
  azioni biochimiche e ormoniche riflesse e infine un'attenuazione della
  dolorabilità cutanea e profonda
\end{itemize}

NORME PRELIMINARI

\begin{itemize}
\item
  \begin{quote}
  Corretta prescrizione medica: il massaggio deve essere prescritto
  dallo specialista con una finalità ben precisa;
  \end{quote}
\item
  \begin{quote}
  deve essere eseguito da personale adeguatamente preparato con tempi e
  modi ben definiti e anche in ambienti ben definiti (l'ambiente del
  massaggio deve essere rilassante e consentire al fisioterapista di
  poter agire liberamente);
  \end{quote}
\item
  \begin{quote}
  prevede l'uso di talco e sostanze oleose per favorire lo scivolamento
  delle mani;
  \end{quote}
\item
  \begin{quote}
  non deve mai essere praticato in regioni poco pulite;
  \end{quote}
\item
  \begin{quote}
  ogni massaggio deve iniziare con una metodica di sfioramento locale e
  deve terminare con la stessa metodica;
  \end{quote}
\item
  \begin{quote}
  deve iniziare lentamente e poi aumentare progressivamente come
  intensità;
  \end{quote}
\item
  \begin{quote}
  deve iniziare in periferia nella zona sana per raggiungere
  progressivamente la zona interessata perché in questa ci possono
  essere reazioni locali di aumento della sintomatologia dolorosa;
  \end{quote}
\item
  \begin{quote}
  la durata è variabile, ma normalmente dovrebbe essere di 20-30 minuti.
  I cicli di massoterapia sono cicli di 10 sedute come tutti i
  trattamenti fisioterapici (possono naturalmente aumentare a seconda
  delle condizioni) e le sedute dovrebbero essere generalmente a cadenza
  giornaliera ma possono anche essere utilizzate ad intervalli ben
  definiti soprattutto in quelle forme in cui c'è una tecnica molto
  impegnativa.
  \end{quote}
\end{itemize}

MANUALITà

In ambito fisioterapico il tipo di massaggio più usato è quello
``classico'' o ``svedese'' che ha un effetto \emph{decontratturante}.
Creato da \emph{Pehr Henrik Ling} (1776-1839) in Svezia, consiste in una
serie di manovre che migliorano la circolazione sanguigna e linfatica,
sciolgono le contratture muscolari e attenuano le aderenze tra i muscoli
superficiali e quelli profondi.

Si può trattare tutto il corpo o solamente certe zone, in questo caso
generalmente si esegue sul colo, sulla schiena e sulla coscia
(quadricipite).

Le mani del fioterapista devono muoversi nella direzione delle
\emph{fibre} dei muscoli superficiali, seguendo il senso della
\emph{circolazione venosa}, cioè verso l'atrio destro del cuore.

Questa terapia dà ottimi risultati nel trattamento delle
\emph{cervicalgie} e \emph{lombalgie}, diminuisce la rigidità e il
dolore per l'eccessivo stress e temsione emotiva e attenua i sintomi
dell'infiammazione.

I pazionti che soffrono di cefalea muscolo-tensiva derivata dalla
tensione del collo quando ricevono il massaggio terapeutico cervicale di
solito riferiscono anche l'attenuazione del dolore.

È utilizzato per il trattamento di cicatrici perché contribuisce a
sciogliere le aderenze tissutali rendendo più morbido e fluido il
movimento.

Oltre ai benefici sul corpo, ha effetto rilassante in particolare per le
persone emotive e ansiose, migliora l'umore e libera dalle
preoccupazioni quotidiane e dallo stress.

4 manovre:

\begin{itemize}
\item
  \textbf{sfioramento;}
\item
  \textbf{frizione;}
\item
  \textbf{impastamento};
\item
  percussione.
\end{itemize}

A queste sono state poi aggiunte:

\begin{itemize}
\item
  \textbf{vibrazione;}
\item
  \textbf{pressione}.
\end{itemize}

\emph{\emph{Sfioramento}}:

\begin{itemize}
\item
  \begin{quote}
  Precede e termina tutti i movimenti del massaggio; se si iniziasse il
  trattamento con una frizione profonda il paziente tenderebbe a
  irrigidirsi, in contrasto con lo scopo del massaggio, il fine di
  questa manovra è quella di stabilire un contatto in maniera graduale e
  di entrare in sintonia con il paziente.
  \end{quote}
\item
  \begin{quote}
  si tratta di movimenti leggerissimi, continui, eseguiti con la faccia
  palmare della mano in estensione o in semi-flessione con movimento
  tangenziale. L'abilità dell'operatore è quella di adeguare il
  movimento delle mani ai rilievi che incontrerà a contatto con la
  superficie cutanea da trattare (ad esempio un paziente affetto da
  scoliosi può avere una salienza che deve essere affrontata in maniera
  adeguata);
  \end{quote}
\item
  \begin{quote}
  le 2 mani sfiorano la cute dandosi alternativamente il cambio a
  cadenza costante e lentamente, procedendo in senso centripeto. Si
  possono alternare movimenti tangenziali con movimenti incrociati.
  \end{quote}
\item
  \begin{quote}
  La velocità e la pressione aumentano in maniera progressiva, viceversa
  al termine calano lentamente il ritmo e l'intensità
  \end{quote}
\end{itemize}

Lo sfioramento, a secondo della pressione esercitata, può essere:

\begin{itemize}
\item
  superficiale;
\item
  profondo.
\end{itemize}

Possiamo avere inoltre tecniche particolari come quella detta ``a
pettine'' in cui si possono utilizzare le nocche e le falangi a pugno
chiuso con il polso in flessione.

Lo sfioramento superficiale ha un'azione sulla circolazione cutanea,
agisce eliminando i detriti e stimolando la secrezione ghiandolare e la
rigenerazione tissutale. Esplica, inoltre, un effetto sedativo adendo
sulle terminazione sensitive.

Lo sfioramento profondo invece serve per spostare liquidi che altre
manovre possono aver spremuto come l'impastamento e la percussione.

\emph{\emph{Frizione}}

Se aumentiamo l'energia durante l'applicazione terapeutica eseguiamo una
frizione.

Si esercita una decisa pressione sulla zona da trattare e muovendo
contemporaneamente la mano in senso trasversale e circolare.

Attraverso questa otteniamo un'azione anche sui tessuti profondi. È
utilizzata per mobilizzare ematomi, essudati, per ridare elasticità ai
tessuti o per liberare delle aderenze tissutali o cicatriziali. Ampiezza
e intensità sono proporzionali alle resistenze tissutali del paziente.

Esistono diverse tecniche di applicazione della frizione. Si possono
usare:

\begin{itemize}
\item
  polpastrelli dell'indice e del medio affiancati;
\item
  indice sovrapposto al medio o viceversa;
\item
  polpastrello del pollice;
\item
  i 2 indici con movimenti circolari;
\item
  indice e medio con movimento rotatorio
\item
  tutto il palmo della mano.
\end{itemize}

È particolarmente adatta per trattare la schiena che è composta da 3
strati di muscoli sovrapposti. Dopo la fase di cicatrizzazione
post-lesione muscolare, il massaggio trasverso profondo utilizza un tipo
di frizione con lo scopo di eliminare le aderenze, ridare elasticità ai
muscoli e al tessuto connettivo, ridurre l'infiammazione.

\emph{\emph{Pressione}}

È una manovra lenta e profonda, condotta con polpastrelli di pollice o
di altre dita unite, con il pugno chiuso o con la base della mano. La
pressione si esercita localmente con manovre circolari o ellittiche.

\emph{\emph{Impastamento}}

Applicabile in corrispondenza di masse voluminose come ad esempio la
coscia, natiche o regione deltoidea. Serve a spremere i tessuti ed è
necessaria una buona forza di applicazione con dei tempi e dei ritmi ben
definiti.

Naturalmente con l'impastamento, il tessuto muscolare viene ampiamente
fluidificato in quanto migliora l'afflusso ma anche il deflusso di
sangue, liberando più rapidamente i muscoli dalle scorie azotate
responsabili dell'affaticamento. Ne migliora sia l'elasticità che la
contrattilità attraverso modificazioni vascolari e umorali.

La tecnica consiste nell'afferrare con il pollice e le altre dita i
tegumenti e i muscoli, sollevarli e compiere un movimento di torsione.
Si appoggiano i palmi delle mani affiancati, con le dita distese, sulla
cute del paziente e avvicinando l'indice di una al pollice dell'altra
con una leggera pressione, tale da sollevare una plica di pelle. Per una
miglior sensazione da parte del paziente è meglio l'eminenza di una mano
insieme al pollice e tutto l'indice fino alla nocca in modo da
esercitare una pressione su una superficie maggiore. A livello del collo
, a causa delle dimensioni ridotte dei muscoli si esegue solo con i
polpastrelli. Questa tecnica viene utilizzata in maniera continuativa
con le mani sempre aderenti alla superficie cutanea, disposte una
affianco all'altra secondo il margine mediale, perpendicolarmente alla
direzione delle fibre muscolari da trattare.

\emph{\emph{Percussione}}

Consiste in colpi ravvicinati, ritmici, rapidi, eseguiti con movimenti
di flesso-estensione del polso.

Come nelle manovre precedenti esistono tecniche differenti a seconda del
tessuto da trattare. L'azione più efficace della percussione è
sicuramente sul muscolo: agisce stimolandolo il muscolo, non
migliorandone in maniera importante il trofismo, ma dando comunque la
possibilità al muscolo di poter reagire agli stimoli. C'è, inoltre, un
effetto riflesso sulla circolazione.

Viene utilizzata nelle \emph{ecchimosi e nei versamenti} grazie alla sua
azione sui vasi linfatici e su quelli sanguigni.

\textbf{\emph{Vibrazione}}

Consiste in scuotimenti rapidi alternando pressione e rilasciamenti,
mantenendo costante il contatto con la cute del paziente. Ha un'azione
sedativa sulle terminazioni nervose.

\emph{Pincè Roulè}

Si esegue con due mani, generalmente sulla schiena, sul coolo e
sul'addome in senso caudo-craniale oppure sul quadricipite e sul
tricipite surale in senso trasversale. Si solleva una plica cutanea tra
i polpastrelli dei pollici inferiormente e quelli delle altre dita
superiormente. Con i pollici si preme solamente sul corpo del paziente,
mentre con le altre dita si ``cammina'' innalzando una porzione cutanea
successiva fino alla fine del movimento. È una manovra che serve per
rimuovere le aderenze tissutali e le prime volte può essere fastidiosa
per alcuni pazienti.

Le controindicazioni sono di 2 tipi: generali o locali, cioè solo nella
zona trattata.

Generali

\begin{itemize}
\item
  febbre
\item
  pressione alta, si può fare un massaggio molto leggero, privilegiando
  lo sfioramento
\item
  malattie infettive.
\end{itemize}

Locali

\begin{itemize}
\item
  infiammazioni, il massaggio può aggravare questa condizione
\item
  fratture ossee:si può praticare solo un massaggio leggero evitando la
  zona interessata
\item
  problemi dermatologici: tipo rash, ustioni, ferite, lividi e vesciche
\item
  cancro, perché può velocizzare la diffusione attraverso il sistema
  linfatico a causa della velocità di aumento del flusso sanguigno
\item
  patologia venosa a rischio di distacco di emboli (tromboflebiti,
  flebotrombosi)
\end{itemize}

MASSAGGIO CONNETTIVALE

Si tratta di una tecnica particolare, scoperta da una fisioterapista
tedesca che capi che questo massaggio poteva avere diverse effetti
terapeutici quando vide che eseguendo manovre su un paziente, aumentando
la pressione e utilizzando delle linee di applicazione ben definite si
creavano delle zone di iper-vascolarizzazione sfruttabili come effetto
terapeutico.

Si tratta di un massaggio particolarmente doloroso perché è
particolarmente energico.

Questo massaggio si applica su zone dolorose, i cosiddetti tender
points, in modo da portare queste zone dolorose verso la normalizzazione
riducendo la tensione elastica e normalizzando quel tessuto. Questa
tecnica utilizza soprattutto i polpastrelli delle dita con movimenti di
traslazione avanti indietro venendo a creare la cosiddetta ``striscia
diagnostica'' che ha un effetto di miglioramento sulla circolazione
locale.

Questi movimenti posso essere graduati in modo diverso in modo da agire
a diversi livelli. Infatti a seconda della pressione esercitata possiamo
avere effetti:

\begin{itemize}
\item
  superficiali;
\item
  medi;
\item
  profondi.
\end{itemize}

Si tratta di un massaggio che normalmente dura di più degli altri,
un'oretta circa ed è molto impegnativo per l'operatore ma anche per il
paziente. Gli effetti sono:

\begin{itemize}
\item
  decontratturanti;
\item
  analgesici (sembra paradossale visto che è una tecnica dolorosa, ma si
  è visto come in esiti traumatici, naturalmente non in acuto ma dopo
  qualche giorno, è molto efficace perché agisce sui versamenti, sugli
  ematomi, sulle aderenze e in particolare in tutte quelle condizioni
  che riguardano i muscoli e le articolazioni);
\item
  sugli organi interni grazie alla sua azione di aumento dell'afflusso
  di sangue superficiale.
\end{itemize}

Con questa tecnica si ha il sollevamento della zona dei tessuti che
vengono trattati e si formano a grandi linee delle pieghe cutanee in
modo tale che si venga a scollare il tessuto.

Normalmente questo tipo di massaggio viene applicato tangenzialmente
alla cute da trattare.

A differenza delle manovre precedenti deve essere eseguita senza
l'utilizzo di sostanze oleose o talco.

Questa tecnica può essere sfruttata soprattutto sulle regioni
paravertebrali. Si uncina il tessuto, si solleva e poi si rilascia.

Si fanno degli schemi ben definiti:

\begin{itemize}
\item
  la piccola costruzione a livello lombare;
\item
  la grande costruzione.
\end{itemize}

Una manovra simile al massaggio connettivale è il massaggio trasversale
che può essere superficiale o profondo a seconda dei livelli su cui
vogliamo agire.

MASSAGGIO TRASVERSALE PROFONDO

È una tecnica che viene eseguita a giorni alterni, ha una durata di
\textbf{5-20 minuti}. Nei primi 3-4 giorni può aumentare il dolore
perché è molto intenso e può dare a livello locale escoriazioni,
ecchimosi o ematomi.

Sulla base di questa, si sono poi elaborate altre tecniche come quella
della \emph{digito pressione}.

Un'altra \emph{tecnica} è quella \emph{di Vodder} che è un po' una via
di mezzo tra il massaggio connettivale e il linfodrenaggio. Si tratta di
una tecnica che viene utilizzata per il linfodrenaggio locale con un
sistema di applicazione che utilizza dallo sfioramento alla frizione
alla pressione in modo da liberare i vasi.

Quindi capiamo come una tecnica semplice come il massaggio, può
diventare in mani di esperti un trattamento terapeutico importante.
Addirittura hanno studiato la possibilità di utilizzare il massaggio a
livello del viso in soggetti che fanno uso di interventi terapeutici
come il \textbf{lipofilling o iniezione di acido ialuronico}, perché la
tecnica di sfioramento, se fatta in modo adeguato, riesce a distribuire
bene il preparato iniettato nella zona da trattare con un effetto
maggiore. In realtà, questo trattamento può creare delle irritazioni
locali e quindi non tutti lo usano.

\emph{Controindicazioni}

\begin{itemize}
\item
  \begin{quote}
  \textbf{Infezioni}, quindi tutte le malattie esantematiche non possono
  essere trattate con il massaggio;
  \end{quote}
\item
  \begin{quote}
  \textbf{reattività locali e cutanee};
  \end{quote}
\item
  \begin{quote}
  \textbf{lesioni cutanee}, la cute deve essere integra;
  \end{quote}
\item
  \begin{quote}
  \textbf{zone edematose in fase acuta};
  \end{quote}
\item
  \begin{quote}
  \textbf{allergie} ai preparati oleosi.
  \end{quote}
\end{itemize}

\end{document}
