\documentclass[]{article}
\usepackage{lmodern}
\usepackage{amssymb,amsmath}
\usepackage{ifxetex,ifluatex}
\usepackage{fixltx2e} % provides \textsubscript
\ifnum 0\ifxetex 1\fi\ifluatex 1\fi=0 % if pdftex
  \usepackage[T1]{fontenc}
  \usepackage[utf8]{inputenc}
\else % if luatex or xelatex
  \ifxetex
    \usepackage{mathspec}
  \else
    \usepackage{fontspec}
  \fi
  \defaultfontfeatures{Ligatures=TeX,Scale=MatchLowercase}
\fi
% use upquote if available, for straight quotes in verbatim environments
\IfFileExists{upquote.sty}{\usepackage{upquote}}{}
% use microtype if available
\IfFileExists{microtype.sty}{%
\usepackage{microtype}
\UseMicrotypeSet[protrusion]{basicmath} % disable protrusion for tt fonts
}{}
\usepackage[unicode=true]{hyperref}
\hypersetup{
            pdfborder={0 0 0},
            breaklinks=true}
\urlstyle{same}  % don't use monospace font for urls
\IfFileExists{parskip.sty}{%
\usepackage{parskip}
}{% else
\setlength{\parindent}{0pt}
\setlength{\parskip}{6pt plus 2pt minus 1pt}
}
\setlength{\emergencystretch}{3em}  % prevent overfull lines
\providecommand{\tightlist}{%
  \setlength{\itemsep}{0pt}\setlength{\parskip}{0pt}}
\setcounter{secnumdepth}{0}
% Redefines (sub)paragraphs to behave more like sections
\ifx\paragraph\undefined\else
\let\oldparagraph\paragraph
\renewcommand{\paragraph}[1]{\oldparagraph{#1}\mbox{}}
\fi
\ifx\subparagraph\undefined\else
\let\oldsubparagraph\subparagraph
\renewcommand{\subparagraph}[1]{\oldsubparagraph{#1}\mbox{}}
\fi

% set default figure placement to htbp
\makeatletter
\def\fps@figure{htbp}
\makeatother


\date{}

\begin{document}

\emph{Trattamento riabilitativo delle lesioni nervose periferiche}

Il trattamento riabilitativo delle lesioni nervose periferiche è
importante sia per la valutazione sia per il trattamento post
chirurgico.

Ci sono una serie di interventi da programmare che sono molto
importanti.

In una lesione nervosa periferica (per esempio del nervo radiale, del
nervo ulnare, del nervo mediano...) ci saranno dei \emph{deficit motori}
importanti che obbligheranno il paziente a mantenere delle posizioni per
un lungo periodo di tempo. (Vedi dopo esempio su lesione del nervo
radiale) Bisogna quindi fare in modo che questi atteggiamenti \emph{non
diano rigidità} o \emph{altre problematiche} e che il paziente
\emph{possa recuperare pienamente o in parte la funzione che è stata
compromessa} dal danno nervoso.

Nel dettaglio il quadro che si presenta a seguito di una lesione nervosa
periferica si caratterizzerà per:

\begin{itemize}
\item
  D\emph{eficit di contrattilità muscolare} perché se il muscolo non è
  innervato, non si attiva
\item
  \emph{Disturbo trofico: Ipotrofia} o addirittura ad \emph{atrofia}
  (conseguenza del deficit di contrattilità muscolare)
\item
  \emph{Deficit di forza }
\item
  \emph{Squilibrio tra muscoli agonisti e antagonisti} e questo si
  ripercuote a livello del \emph{movimento delle articolazioni}
\item
  \emph{Rigidità extrarticolare }
\item
  \emph{Deficit di sensibilità}
\item
  \emph{Deficit della funzione simpatica}
\item
  Disturbo \emph{vasomotorio} legato all'edema post lesionale.
\item
  \emph{Sintomatologia dolorosa} molto importante (a volte) e quindi
  bisognerà trattare il dolore sia dal punto di vista farmacologico che
  dal punto di vista terapeutico-riabilitativo con terapie ben
  specifiche.
\item
  In ultimo una seria problematica legata all'immobilità che è il
  \emph{non uso} di quel muscolo. C'è il rischio che si possano creare
  delle condizioni di disabilità in questo paziente.
\end{itemize}

Da quanto detto sopra si evince come a seguito di una lesione nervosa
periferica, il muscolo venga interessato fin da subito e vada incontro a
tutta una serie di alterazioni importanti delle proprietà fisiologiche e
dell'eccitabilità che esitano alla fine nell' \textbf{atrofia muscolare}
perché:

\begin{enumerate}
\def\labelenumi{\arabic{enumi}.}
\item
  \begin{quote}
  Il muscolo non si attiva, non si muove
  \end{quote}
\item
  \begin{quote}
  L'articolazione rimane immobile e va incontro a rigidità
  \end{quote}
\item
  \begin{quote}
  Si formano delle aderenze a livello dell'articolazione per cui non c'è
  movimento ed è tutto un sistema a innesco.
  \end{quote}
\end{enumerate}

Nell'atrofia muscolare c'è però \textbf{\emph{anche}} una componente
importante che riguarda per esempio alcuni \emph{ormoni} come la
vitamina D e gli ormoni tiroidei.

Si assiste a dei cambiamenti di tipo ormonale che non rappresentano
altro che una \emph{risposta adattativa} del muscolo alla lesione e sono
molto importanti in un muscolo che ha una componente mista (con fibre
veloci e fibre lente) e \emph{in cui si ha una trasformazione verso un
muscolo costituito interamente da fibre di tipo veloce. }

Questo tipo di variazione è importante e da non sottovalutare perché
\emph{darà delle problematiche di tipo riabilitativo.}

\emph{Stimolazione elettrica del muscolo denervato}

Nel trattamento riabilitativo è importante \textbf{stimolare il
muscolo}. Nello specifico:

\begin{itemize}
\item
  \begin{quote}
  Se il muscolo è totalmente denervato, occorreranno delle
  \emph{correnti di tipo esponenziale},
  \end{quote}
\item
  \begin{quote}
  Se il muscolo è parzialmente denervato, occorreranno delle
  \emph{correnti di tipo triangolare}.
  \end{quote}
\item
  \begin{quote}
  La corrente triangolare è una corrente che ha un picco altissimo:
  raggiunge molto rapidamente il picco e altrettanto velocemente scende
  \end{quote}
\end{itemize}

\begin{quote}
N.B. le correnti normalmente utilizzate \emph{per stimolare} un muscolo
e \emph{per rinforzarlo} non sono triangolari, ma sono rettangolari
perché devono mantenere un picco per un periodo di tempo più lungo.
\end{quote}

Solo stimolando il muscolo si otterranno degli effetti sulla
\textbf{rigenerazione} e sulla \textbf{re innervazione} che sono
fondamentali per il recupero funzionale e quanto più precocemente si
inizia il trattamento dopo una lesione nervosa periferica, più
possibilità di recupero si hanno. Nell'ambito di una lesione nervosa il
tempo di guarigione dipende dal tipo di lesione: a seconda che la
lesione interessi l'assone o il neurone, i tempi possono variare da
tempi minimi di 7-8 mesi fino a tempi di 2 anni.

Nel dettaglio è importante:

\begin{itemize}
\item
  Il \textbf{\emph{metodo di stimolazione}} ovvero
\end{itemize}

\begin{itemize}
\item
  Come stimolare la fibra
\item
  Con quali forme di onde stimolarla (le correnti possono essere
  triangolari, esponenziali o bifasiche)
\end{itemize}

\begin{quote}
Perché in base al \emph{metodo di stimolazione} si possono
\emph{\textbf{modificare} le \textbf{proprietà contrattili} e
\textbf{metaboliche}} del muscolo striato: \emph{la stimolazione lenta
di muscoli a fibre rapide converte le fibre, in fibre lente} quindi il
tipo di stimolazione è importante.
\end{quote}

\begin{itemize}
\item
  Le \textbf{\emph{frequenze d'onda}} utilizzate: stimolazioni a
  frequenze diverse produrranno a loro volta delle \emph{risposte
  diverse}:
\end{itemize}

\begin{itemize}
\item
  \begin{quote}
  A 10 Hz il muscolo tricipite surale mantiene le sue caratteristiche
  cioè di essere un muscolo a componente prevalentemente lenta.
  \end{quote}
\item
  \begin{quote}
  Se invece stimoliamo a 100 Hz abbiamo la trasformazione delle sue
  fibre in fibre veloci.
  \end{quote}
\end{itemize}

\begin{itemize}
\item
  \textbf{Quando} applicarla
\item
  Il \textbf{tempo} di applicazione. Questo cambia a seconda del tipo di
  lesione, ma anche di corrente per esempio quelle bifasiche devono
  essere applicate per 20-25 millisecondi.
\end{itemize}

Per prevenire l'atrofia muscolare si deve stimolare il muscolo a
frequenze ben definite: a frequenze basse (sotto i 20 Hz) o ad alte
frequenze (tra 40 e 100Hz) a seconda del grado di denervazione: Il
\emph{muscolo denervato} va stimolato con \emph{frequenze basse}:
massimo 20 Hz.

Questa stimolazione a bassa frequenza trasformerà la tipologia di fibre
che compone il muscolo, ma se questo reagisce bene al trattamento, il
problema si può affrontare in seguito con la riabilitazione facendo
lavorare il muscolo in maniera adeguata anche con le fibre trasformate.

\emph{Effetti sulla rigenerazione nervosa}

In numerosi studi è stato dimostrato che \emph{l'elettrostimolazione
inibisce l'espressione dei fattori di crescita} nonostante la
rigenerazione dell'assone prosegua.

Questo è importante perché l'inibizione dei fattori di crescita incide
sulla \emph{velocità di rigenerazione}.

E' stato dimostrato infatti che \emph{la velocità di rigenerazione è
\textbf{inversamente proporzionale} alla lunghezza del nervo e all'età
del soggetto}: nell'uomo adulto viaggia a 2,5 mm al giorno nel braccio
mentre a livello del polso e della mano si riduce.

Quindi la velocità di rigenerazione cambia a seconda di:

\begin{itemize}
\item
  \begin{quote}
  Localizzazione del danno nervoso (lunghezza del nervo)
  \end{quote}
\item
  \begin{quote}
  Età del soggetto
  \end{quote}
\item
  \begin{quote}
  Tipo di stimolazione che facciamo
  \end{quote}
\item
  \begin{quote}
  Tempo in cui applichiamo quel tipo di stimolazione
  \end{quote}
\item
  \begin{quote}
  Tipo di impulso
  \end{quote}
\item
  \begin{quote}
  Tempestività del trattamento.
  \end{quote}
\end{itemize}

\emph{Effetti sulla re innervazione}

Ci sono pareri discordanti circa gli effetti dell'elettrostimolazione
sulla reinnervazione. Secondo alcuni studi (la minoranza) essa inibisce
la reinnervazione. In realtà questo non è molto vero perché la può
ridurre, ma non abolire del tutto. Invece la maggior parte degli studi
afferma che l'elettrostimolazione \emph{favorisce la reinnervazione}.

Perciò attualmente si consiglia quindi di utilizzare
l'elettrostimolazione in fase precoce.

I fattori che possono ostacolare il processo di rigenerazione sono:

\begin{itemize}
\item
  \textbf{Cicatrice}
\item
  Eventuale formazione/presenza di \textbf{Neurinomi}. Ad esempio il
  \emph{neuroma di Morton} è un agglomerato di fibre nervose che dà una
  stimolazione dolorosa a livello locale; questo può verificarsi nei
  processi di rigenerazione oppure fisiologicamente per situazioni
  concomitanti al carico eccessivo che stimola meccanicamente la
  formazione di queste forme di aggregazione.
\item
  In più ci sono le \textbf{degenerazioni dei recettori} da cui la
  possibilità che non ci sia un'interfaccia e una perfetta
  sincronizzazione dell'effetto dettato dalle fibre motorie, sensitive e
  simpatiche.
\end{itemize}

Fondamentalmente comunque è il \emph{tipo di trauma}, l'\emph{età} del
paziente e il \emph{tipo di nervo} che condizionano la rigenerazione.

Ricordiamo che esiste una sindrome dolorosa molto importante a livello
regionale che può essere anche essa gestita mediante trattamento
elettrico con correnti antalgiche.

Le tipiche \textbf{correnti antalgiche} che vengono utilizzate, sono
delle correnti \emph{diadinamiche} che non sempre si riescono a
reperire. Molto frequentemente, nella maggior parte dei casi si usano le
\emph{TENS} che sono correnti antalgiche che possono dare beneficio se
associate ovviamente a terapia farmacologica.

\emph{Esempio di lesione: nervo radiale}

Lesioni del nervo radiale danno una posizione della mano ben definita
che è legata alla perdita della stabilizzazione del polso quindi non c'è
più la presa. Vediamo come procedere:

\begin{itemize}
\item
  Bisogna fare in modo che questa posizione non crei delle aderenze o
  delle rigidità per cui la prima cosa da fare è porre un \emph{tutore}.
\end{itemize}

\begin{quote}
Il tutore serve a \emph{mantenere la posizione fisiologica della mano} e
quindi ad \emph{evitare questa destabilizzazione del polso.}

I tutori possono essere statici o dinamici e in questo caso sarà
utilizzato un tutore di tipo \emph{statico}: non si usa il dinamico
perché non c'è bisogno di stimolare in quanto non ci sarebbe nessuna
risposta.
\end{quote}

\begin{itemize}
\item
  Elettroterapia
\item
  Nella fase successiva, dopo l'elettroterapia, si può optare per un
  tutore dinamico che magari permetta di stimolare i movimenti delle
  dita.
\end{itemize}

\begin{quote}
I tutori dinamici sono fatti in una maniera particolare: in questo caso
vengono confezionati con un sistema a ponte che corre sopra le
articolazioni interfalangee e con degli elastici agganciati alle dita
che hanno la funzione di stimolare il movimento. Il paziente deve
eseguire un minimo movimento e l'elastico darà il contro movimento
quindi l'azione dell'antagonista.

In questo modo si attivano gli agonisti (grazie al movimento del
paziente) e gli antagonisti (grazie agli elastici del tutore)
\end{quote}

Bisogna poi fare anche delle valutazioni in merito alle diverse funzioni
compromesse e agire sì sulla componente motoria, ma non solo:

\begin{itemize}
\item
  \textbf{Funzione simpatica: l}e alterazioni del simpatico si
  manifestano nelle prime 2-3 settimane dalla lesione:
\end{itemize}

\begin{itemize}
\item
  C'è un blocco della vasocostrizione (con tutte le condizioni legate ad
  essa)
\item
  Perdita di sudorazione/iperidrosi
\item
  Rallentata guarigione delle ferite
\end{itemize}

\begin{itemize}
\item
  \textbf{Funzione motoria.} E' importante definire quanto è compromessa
  la funzione motoria perché il processo di riparazione necessita di
  tempi lunghi (può durare molti mesi) in quanto la velocità di
  rigenerazione è molto lenta: circa \emph{1 mm al giorno}. Un segno
  precoce di reinnervazione è la \textbf{sensibilità del ventre
  muscolare alla pressione}. Quando si riesce a ottenere un po' di
  contrazione (segno di un inziale recupero della funzione motoria) si
  passa gradualmente al rinforzo muscolare: da una iniziale contrazione
  isometrica (il muscolo mantiene la stessa lunghezza, non si modifica)
  si passa alla contrazione isotonica (il muscolo si modifica). Si
  possono utilizzare a tal proposito dei test specifici per analizzare e
  misurare la forza del muscolo.
\item
  \textbf{Funzione sensitiva.} La funzione sensitiva è alterata in tutte
  le sue componenti (tattile, propriocettiva, discriminativa) per cui
  vanno utilizzate delle tecniche di \emph{stimolazione locale}. Si
  strofina la cute con diversi componenti e si fa in modo che il
  soggetto possa riconoscerli a occhi chiusi, questo per migliorare la
  discriminazione tattile. Si fa in modo che il soggetto possa spostare
  degli oggetti da una posizione all'altra. Ovviamente ci deve essere un
  minimo di attività motoria per fare questo. Quando non c'è attività
  motoria, l'unica stimolazione possibile, sarà quella locale dei
  recettori cutanei (esterocettivi, propriocettivi e dolorifici).
\end{itemize}

N.B. Da ricordare che anche la stimolazione locale dà un effetto sul
dolore. Questo è stato dimostrato ampiamente.

Il \textbf{programma di desensitizzazione} viene attuato
progressivamente ed è organizzato in modo tale da variare le situazioni
per esempio utilizzare la pressione profonda o la pressione superficiale
con tocchi di diversa intensità. Per esempio un tocco leggero in
movimento oppure un tocco statico senza movimento oppure andare a
localizzare il tocco. Si invita il soggetto a chiudere gli occhi e si
chiede ``Dove sto toccando?'' ``Cosa stai percependo?'' ``Che tipo di
sensazione stai avendo e che tipo di strumento sto usando? Un dito, un
puntale, senti il freddo o il caldo?''. E' quindi una discriminazione
localizzata.

\begin{itemize}
\item
  \textbf{Valutazione del dolore.} Il dolore locale legato alla lesione
  è molto importante. Vengono utilizzate delle scale per valutarne
  l'intensità e anche per valutare nel tempo l'efficacia del trattamento
  riabilitativo. Normalmente viene utilizzata la \textbf{scala analogica
  del dolore}: il paziente indica con una crocetta l'\emph{intensità}
  del dolore in una scala che può andare da 0 a 10 o da 1 a 100. Oppure
  si usa il \textbf{questionario di Mc Guill} che indica la
  \emph{qualità} del dolore percepito dal soggetto.
\end{itemize}

Riassumendo quanto sopra detto, il trattamento riabilitativo è un
trattamento complesso che deve iniziare precocemente:

\begin{enumerate}
\def\labelenumi{\arabic{enumi}.}
\item
  \begin{quote}
  Nella \textbf{\emph{fase acuta}} c'è un periodo di 3-4 settimane
  durante il quale si deve \emph{immobilizzare} il paziente con tutori
  in posizione fisiologica e \emph{stimolare} la zona.
  \end{quote}
\end{enumerate}

\begin{quote}
Per l'immobilizzazione si può utilizzare un tutore \emph{statico} per
3-4 settimane: le prime 2 in maniera continuativa e poi in maniera
intermittente.

Questi tutori devono essere posizionati in maniera adeguata in modo tale
da non esercitare azioni di compressione meccanica e quindi causare
delle lesioni locali (lesioni da decubito): questo allungherebbe i tempi
di recupero.

Attraverso diverse fasi di recupero si arriva poi ad utilizzare dei
tutori \emph{dinamici}.
\end{quote}

\begin{enumerate}
\def\labelenumi{\arabic{enumi}.}
\item
  \begin{quote}
  Dopo l'immobilizzazione inizia la \textbf{\emph{fase di recupero}}:
  \end{quote}
\end{enumerate}

\begin{itemize}
\item
  Desensitizzazione con rieducazione sensitiva
\item
  Recupero della funzione motoria.
\end{itemize}

\begin{enumerate}
\def\labelenumi{\arabic{enumi}.}
\item
  \begin{quote}
  Nella \textbf{\emph{fase finale}} bisogna valutare a distanza di tempo
  se il soggetto riesce a recuperare completamente la funzione oppure se
  la recupera parzialmente. In quest'ultimo caso il paziente dovrà
  compensare quella funzione mediante l'utilizzo di \textbf{ausili}
  oppure tramite \textbf{interventi chirurgici} (il chirurgo farà dei
  transfert tendinei per cercare di dare una funzione).
  \end{quote}
\end{enumerate}

Nella \textbf{\emph{fase di recupero}}, quando compaiono i segni di
reinnervazione, bisogna impostare il programma riabilitativo che si
baserà sullo \textbf{stimolare} il muscolo in maniera \textbf{attiva}
(se il paziente lo fa autonomamente) o \textbf{attiva assistita} (se lo
fa su indicazione di un terapista).

L'obiettivo è di recuperare la \emph{coordinazione}, la \emph{forza} e
la \emph{resistenza} di quel muscolo.

Se il paziente non recupera completamente, bisogna valutare quanta
funzionalità residua e quindi le potenzialità del soggetto.

Si misurano alcuni parametri: l'arco di movimento, la forza, il deficit
di quel muscolo.

Quindi bisognerà fare una serie di valutazioni a partire dalla
\emph{misurazione goniometrica} fino ad arrivare alla valutazione della
\emph{forza isometrica}. Ci sono degli apparecchi che si chiamano
DINAMOMETRI che permettono di fare una valutazione della forza di presa
della mano o della forza di un muscolo contro resistenza. I dinamometri
si usano per stabilire come agire poi sul paziente.

Per esempio per l'arto superiore si valuta la presa palmare utilizzando:

\begin{itemize}
\item
  \emph{Dei dinamometri idraulici} per misurare la forza oppure
\item
  I \emph{dinamometri isotonici} in cui c'è una leva e un elastico e il
  soggetto deve mantenere il movimento per misurare la forza (che viene
  registrata su un display).
\end{itemize}

Ci sono poi dei \emph{dinamometri isocinetici} che sono particolari e
permettono di valutare la forza di tutti i distretti muscolari.

Ad esempio si può studiare l'azione del muscolo agonista e del suo
antagonista (il prof considera la contrazione del quadricipite e
dell'ischiocrurale): si può valutare come si comporta il muscolo per
capire se ci sono deficit, se il rapporto tra questo muscolo e il suo
antagonista è ottimale (valutando la percentuale e il rapporto con
l'altro).

I dinamometri sono anche dotati di attrezzature che vengono utilizzate
per far simulare al paziente situazioni di tipo quotidiano o azioni
lavorative (per esempio prese, pinze, il volante\ldots{})

Per valutare il danno nervoso si possono utilizzare anche delle
\textbf{elettromiografie}. Normalmente l'elettromiografia non è
utilizzata nelle fasi iniziali, ma è consigliata in fasi più avanzate
soprattutto l'\emph{elettromiografia ad ago}. Questo tipo di
elettromiografia la si utilizza per valutare come si attivano i muscoli,
com'è la contrazione muscolare di un determinato distretto muscolare. E'
un'elettromiografia di superficie, semplice da applicare e che permette
di fare delle valutazioni molto fini.

Nell'\emph{elettromiografia con markers} viene fatto un filmato per
vedere il movimento nello spazio ad esempio di un pianista
professionista e quindi possiamo vedere come si comportano le
articolazioni metacarpo-falangee prossimali e distali, il movimento del
polso, del gomito e della spalla e dei muscoli flessori ed estensori del
polso e delle dita.

Questo permette di studiare il movimento nello spazio ed andare ad
analizzare chiaramente tutti i movimenti.

Poi singolarmente vengono analizzati tutti i tracciati per vedere come
comportarsi.

Viene fatto un programma riabilitativo generale proprio perché la
condizione è una condizione estremamente disabilitante, molto lunga e
particolarmente complessa da trattare.

\emph{Laserterapia}

In passato la laser terapia veniva utilizzata nel soggetto con lesioni
nervose periferiche che presentava lesioni cutanee o delle ulcere anche
se l'indicazione generale non è sicuramente per le ulcere. Oggi ci sono
a disposizione dei sistemi più efficace e meno complessi da gestire per
il trattamento di ulcere e piaghe

Viene invece utilizzata soprattutto per:

\begin{itemize}
\item
  \emph{Tendiniti}
\item
  \emph{Traumi}
\item
  \emph{Lesioni muscolari}
\item
  \emph{Ematomi}
\item
  \emph{Versamenti}
\item
  \emph{Riacutizzazioni algiche di patologie degenerative} come
  \emph{gonartrosi}, \emph{coxartrosi}, \emph{artrosi della
  tibiotarsica.}
\end{itemize}

\emph{Laser è un acronimo che sta per Light Amplification by Stimulated
Emission of Radiation ed è una}

\emph{Radiazione amplificata legata alla stimolazione di un mezzo
attivo; tale radiazione presenta inoltre diverse caratteristiche
fondamentali: }

\begin{itemize}
\item
  Il raggio laser è \textbf{\emph{monocromatico}} (cioè emette un solo
  colore che normalmente è nell'ambito dell'infrarosso). Il
  \textbf{colore} emesso dal raggio laser dipende dal \textbf{mezzo
  attivo} che dà il nome al tipo di laser: ad esempio il \emph{laser
  CO2} è una miscela di biossido di carbonio che è il mezzo attivo.
  Anche la \textbf{potenza} del raggio laser dipende dal \textbf{tipo di
  mezzo attivo} perché il mezzo attivo viene attivato da un
  \emph{sistema di pompaggio} e ha una certa capacità di generare un
  raggio laser con una determinata potenza. Quella potenza dà
  indicazione del tipo di laser da impiegare.
\item
  La \textbf{\emph{coerenza}} cioè i fotoni vibrano in concordanza di
  fase
\item
  La \textbf{\emph{direzionalità}} cioè la direzione del raggio laser
  emergente che è sempre parallela a quella del raggio incidente secondo
  un angolo di divergenza ben definito che è molto piccolo perché
  vedremo che la produzione del raggio laser avviene in un modo ben
  preciso.
\item
  \emph{\emph{\textbf{Brillanza}: è la luminosità, data dal concentrare
  una grande potenza in un'area molto piccola}}
\end{itemize}

Quando un laser viene applicato sui tessuti, si ha
l'\textbf{assorbimento} del raggio laser e una \textbf{dispersione}
dello stesso nei tessuti. Sia l'assorbimento che la dispersione
dipendono da:

- \textbf{Lunghezza d'onda} che a sua volta dipende dal \emph{mezzo
attivo} (il mezzo che genera il raggio laser e che gli da il nome: laser
CO2, laser elio-neon). Maggiore è la lunghezza d'onda, maggiore sarà la
capacità di penetrazione.

- \textbf{Resistività} del tessuto trattato: il tessuto bersaglio del
raggio laser va incontro a modificazioni che vanno dal riscaldamento
fino alla carbonizzazione a seconda della potenza usata. Raggi laser
vengono applicati oggi anche nell'ambito chirurgico ad esempio per il
trattamento delle neoplasie in ginecologia e anche in urologia.

Normalmente i raggi laser più potenti sono:

\begin{itemize}
\item
  \begin{quote}
  Il laser Neomidio-Yag
  \end{quote}
\item
  \begin{quote}
  I laser solidi, che possono arrivare anche a 1000 watt di potenza e
  raggiungere tessuti molto profondi. Arrivano a profondità superiori a
  6-7 cm e possono causare anche delle lesioni importanti perché
  l'assorbimento del laser crea delle \emph{condizioni progressivamente
  peggiorative}.
  \end{quote}
\end{itemize}

Infatti man mano che aumenta la potenza di un raggio laser si passa dal
riscaldamento locale di un tessuto (effetto termico) a un aumento dei
processi metabolici soprattutto aumento delle mitosi cellulari. Da
disidratazione e raggrinzimento del tessuto per proseguire con la
denaturazione delle proteine e arrivare alla carbonizzazione del tessuto
o all'evaporazione dello stesso.

Più la potenza del laser aumenta più gli effetti possono essere
distruttivi.

Non a caso attualmente gli apparecchi laser vengono utilizzati anche
nella chirurgia e anche per avere effetti locali: un raggio laser
particolare è quello ad eccimeri che viene utilizzato in oculistica, i
raggi laser CO2 con potenze molto elevate sono utilizzati per
cauterizzare le lesioni durante gli interventi chirurgici.

Ogni \textbf{apparecchio laser} è costituito da:

\begin{enumerate}
\def\labelenumi{\arabic{enumi})}
\item
  \textbf{Un mezzo attivo}: può essere solido \emph{(neodimio yag),}
  liquido \emph{(a coloranti),} gassoso (He-Ne; CO2; argon e cripton
  ormai si usano pochissimo); \emph{può essere rappresentato da atomi,
  molecole, ioni, semiconduttori che quando vengono colpiti da un fascio
  di \textbf{fotoni} si eccitano e a loro volta emettono altri fotoni
  per tornare allo stato di riposo.}
\end{enumerate}

\begin{quote}
L'\emph{He-Ne} è utilizzato soprattutto come \emph{raggio guida} perché
la sua frequenza è nell'ambito dell'infrarosso per cui si vede bene e
quindi può essere utilizzato oltre che direttamente anche per
posizionare altri tipi di laser.

Poi ci sono dei \emph{laser a diodi a semiconduttori} che possono avere
delle potenze fino a 80 watt a seconda del numero di diodi che vengono
utilizzati.
\end{quote}

\begin{enumerate}
\def\labelenumi{\arabic{enumi})}
\item
  \textbf{Una cavità ottica} costituita da due specchi contrapposti;
  presenta due specchi parabolici, uno che riflette al 100\%, l'altro al
  95\%; quel 5\% che non viene riflesso viene emesso e rappresenta la
  radiazione laser. \emph{La cavità ottica, o risonatore, serve per
  veicolare il raggio prodotto}
\item
  \textbf{Il sistema di pompaggio:} è il meccanismo di eccitazione del
  mezzo attivo, può essere ottico (fotoni di una lampada a gas, di
  solito allo xenon), elettrico (scarica elettrica emessa nel mezzo
  gassoso) o chimico (energia di legame rilasciata). \emph{Di solito è
  elettrico.}
\item
  \textbf{Un meccanismo di uscita} \emph{tramite il quale il raggio
  raggiunge il paziente }
\end{enumerate}

Ricapitolando, possiamo dire che il mezzo attivo è responsabile della
lunghezza d'onda del laser, del metodo di pompaggio e quindi della
potenza del laser stesso.

\emph{Abbiamo diverse \textbf{CLASSIFICAZIONI DELLE APPARECCHIATURE}
laser, importante sistema per}

\emph{Valutare l'efficienza dell'apparecchio. Possiamo
\textbf{classificarli in base a}:}

\begin{itemize}
\item
  Stato della materia del mezzo attivo
\item
  Lunghezza d'onda
\item
  Metodo d'eccitazione del mezzo attivo
\item
  \emph{Potenza del mezzo laser}
\item
  Numero di livelli energetici prodotti dal raggio laser
\item
  Caratteristiche della radiazione emessa.
\end{itemize}

I laser classificati in base alla potenza possono essere ad alta, media
e bassa potenza.

\emph{Un'altra classificazione importante è quella che si basa sulla
\emph{potenza relativa}: alcune apparecchiature, oltre ad una certa
potenza, devono sottostare ad alcune regole che riguardano il capitolo
della radioprotezione. Un apparecchio laser infatti può produrre delle
lesioni quindi:}

\begin{itemize}
\item
  \emph{Deve essere ubicato in un ambiente chiuso e privo di superfici
  riflettenti al suo interno perché il raggio laser può essere riflesso
  e c'è il pericolo che colpisca alcune strutture delicate come gli
  occhi causando cheratiti, congiuntiviti, ma anche lesioni retiniche.
  Per questo bisogna indossare degli occhiali protettivi.}
\item
  \emph{Nel locale dove è situato il laser deve esserci un sistema di
  segnalazione luminoso indicativo della presenza del macchinario che
  lampeggi quando questo è in funzione}
\item
  \emph{Deve essere presente in corrispondenza dell'uscita dalla stanza
  un sistema che permetta di interrompere l'emissione all'apertura della
  porta}
\item
  \emph{All'interno della stanza deve essere presente un sistema,
  generalmente a pedale o a pulsante, che permetta di interrompere in
  qualsiasi momento l'emissione del raggio}.
\end{itemize}

Ogni apparecchio laser presenta dei parametri che sono:

\begin{itemize}
\item
  \begin{quote}
  Frequenza, responsabile della potenza media del laser e della sua
  capacità di penetrazione nei tessuti
  \end{quote}
\item
  \begin{quote}
  Durata dell'impulso che \emph{determina la modalità di emissione del
  raggio} che può essere \emph{continua, pulsata, intermittente},
  \emph{a singolo impulso}, \emph{a scansione} oppure in un modo ben
  definito che si chiama \emph{modo bloccato,} ma normalmente non si
  utilizza, di solito si usano o quello a scansione o il singolo impulso
  ripetuto con un manipolo che viene applicato localmente.
  \end{quote}
\item
  \begin{quote}
  Potenza media e potenza di picco espressa in watt (poi in milliwatt a
  seconda della potenza)
  \end{quote}
\item
  \begin{quote}
  Dose di radiazione, che normalmente viene definita in J, o meglio in
  J/cm2. Il J corrisponde alla potenza dell'apparecchio al secondo e a
  seconda che questo sia intermittente o continuo può variare. \emph{Se
  la dose somministrata è eccessiva si può provocare una lesione locale
  inoltre si possono causare lesioni anche in profondità. Anche i tempi
  di applicazione variano a seconda della potenza, ad esempio un laser
  He-Ne a 12 Watt lo posso applicare per 20 minuti, ma se uso una
  potenza superiore ai 15/16 Watt devo stare sotto una durata di 15-20
  minuti. In alcuni tipi di laser l'applicazione può essere fatta ad
  intervalli di pochi secondi. Nel caso del laser solido dobbiamo
  applicare il raggio laser per pochi secondi 6-7, perché }
  \end{quote}
\end{itemize}

\emph{Tipi di laser particolari}

\emph{Laser He-Ne}

Costituito da una miscela di gas: elio al 90\% e neon al 10\%.

La sua lunghezza d'onda varia da 540 fino 680 nm quindi è nell'ambito
del rosso per cui il raggio è ben visibile.

Il sistema di pompaggio è costituito da una scarica elettrica in un tubo
contenente i due gas.

Ha una potenza di norma non molto elevata, anche se oggi ce ne sono
alcuni che raggiungono potenze di 70-80 Watt.

\emph{È spesso usato come raggio guida per altri laser che hanno una
frequenza nell'ambito dell'infrarosso e quindi non sono visibili. }

\emph{Il trattamento viene di solito fatto o con un manipolo oppure a
scansione, cioè impulsi continui scansionati su una superficie, con i
nuovi apparecchi si possono stabilire non solo le superfici ma anche le
modalità di somministrazione, perché il raggio laser può muoversi in
scansione diritta avanti e indietro oppure in direzioni diverse (a
stella, a triangolo, a farfalla).}

Adesso hanno introdotto dei laser a He Ne che chiamano \textbf{laser a
freddo} perché scaldano meno, penetrano in profondità, hanno frequenze
molto alte e raggiungono potenza fino a 100-

150 watt.

\emph{Laser a diodi }

\begin{itemize}
\item
  \begin{quote}
  E' quello più comunemente utilizzato
  \end{quote}
\item
  \begin{quote}
  Siamo nello spettro dell'infrarosso
  \end{quote}
\item
  \begin{quote}
  Hanno un basso costo
  \end{quote}
\item
  \begin{quote}
  La potenza può arrivare fino a 70-80 Watt
  \end{quote}
\item
  \begin{quote}
  Possono avere un raggio guida all'He-Ne rosso.
  \end{quote}
\item
  \begin{quote}
  Viene utilizzato con un manipolo oppure a scansione.
  \end{quote}
\end{itemize}

\emph{Laser CO2}

\begin{itemize}
\item
  \begin{quote}
  Lunghezza d'onda superiore (10,6 micrometri), quindi nell'ambito
  dell'infrarosso
  \end{quote}
\item
  \begin{quote}
  Hanno un raggio guida all'He-Ne per essere visibili
  \end{quote}
\item
  \begin{quote}
  Possono avere un'alimentazione con delle bombole di azoto e di elio
  per aumentare la potenza.
  \end{quote}
\item
  \begin{quote}
  Potenze che possono arrivare anche a 30KW
  \end{quote}
\item
  \begin{quote}
  Alcuni tipi di questi vengono utilizzati per la chirurgia con dei
  manopoli
  \end{quote}
\end{itemize}

\emph{Laser ad eccimeri}

\begin{itemize}
\item
  \emph{Laser prettamente chirurgico, usato in chirurgia oculare. }
\item
  \emph{Ad alta potenza. }
\item
  \emph{È usato in microchirurgia perché è un raggio focalizzato con
  dimensioni di pochi micron. Creando delle lesioni determina la
  correzione dei difetti visivi.}
\end{itemize}

\emph{Laser Neomidio Yag}

E' un laser solido molto potente. Esempio di utilizzo: dito a scatto.
Nelle fasi iniziali può dare dei risultati perché è molto potente mentre
nelle fasi avanzate o si interviene con infiltrazione cortisonica oppure
chirurgicamente. \emph{È costituito da un cristallo di yttrio alluminio
granato. Si possono usare anche altri materiali come il calcio o altri
cristalli perché si hanno delle potenze alte.} Ha un sistema di
pompaggio con lampada a tungsteno o a cripton, può raggiungere anche i
1000 watt di potenza con una lunghezza d'onda di 1064 nm quindi
nell'ambito dell'infrarosso (non è visibile) e usa un raggio guida
all'He-Ne perché si possa vedere. La frequenza è tra 10 e 40 Hz, la
potenza può arrivare come picco a 1000 watt e la durata dell'impulso è
abbastanza veloce (200 msec).

E' dotato di un manipolo con un distanziatore perché bisogna mantenere
una certa distanza in quanto più si avvicina il manipolo alla cute più
c'è il rischio di ustionare. Infatti questo tipo di laser funziona ad
assorbimento: man mano che si utilizza il raggio laser, la quantità di
laser viene assorbita e viene aggiunta, si ha un \emph{effetto
moltiplicatorio}.

Normalmente questi apparecchi vengono applicati per pochissimi secondi
(5,6,7 secondi) e per ogni trattamento perché viene definita con
precisione la quantità di energia da erogare in quella zona in J/cm².

Gli effetti di questo laser (che può essere anche di tipo chirurgico)
sono: effetti fotochimici, fototermici e fotomeccanici.

\begin{itemize}
\item
  \begin{quote}
  Effetti foto-chimici per via dell'assorbimento dei cromofori locali
  \end{quote}
\item
  \begin{quote}
  Effetti foto- termico si ha perché l'energia viene convertita in
  calore
  \end{quote}
\item
  \begin{quote}
  Effetti foto-meccanici perché c'è una vibrazione locale legata
  all'applicazione del raggio laser.
  \end{quote}
\end{itemize}

Esempio di trattamento: 660 J/cm², \textbf{livello 8} (significa che la
potenza deve essere tenuta per al massimo 7-8 secondi) e viene
utilizzato un \textbf{ciclo 3} (vuol dire che vengono utilizzate delle
pause ben definite).

\emph{Modalità d'uso}

Questi apparecchi possiedono un sistema che permette al paziente e
all'operatore di interrompere in qualsiasi momento il trattamento perché
quando il paziente inizia a sentire calore, l'impulso successivo è
dolore immediato. Bisogna quindi interrompere immediatamente il
trattamento e bisogna sapere esattamente come usarlo, tanto è vero che
la normativa prevede che questi apparecchi possano essere utilizzati
\emph{solo da personale medico}.

Nello specifico:

\begin{itemize}
\item
  I laser ad alta potenza possono essere utilizzati solo da medici
\item
  Quelli a media e bassa potenza, che vanno intorno a 70-80 watt,
  possono essere utilizzati anche dal personale non medico, anche se
  normalmente vengono usati gli apparecchi piccoli (quelli a diodi o
  quelli più piccoli vanno a 7/8/10 watt di potenza).
\end{itemize}

I laser CO2 viaggiano più o meno a 15-20 watt di potenza mentre gli
ultimi He-Ne e quelli a freddo, viaggiano a 100 watt e dovrebbero usarli
i medici anche se può succedere che lo usino anche i fisioterapisti.

Gli apparecchi che dovrebbero usare solo i medici sono quelli ad alta
potenza. Gli apparecchi devono essere tutti certificati CE e nella
normativa c'è scritto se quell'apparecchio può essere utilizzato solo da
personale medico, poi se questo non si fa è a rischio e pericolo del
terapista.

Inoltre gli apparecchi devono:

\begin{itemize}
\item
  Essere collocati in \emph{ambienti ben definiti}, identificati
\item
  Deve esserci sulla porta un \emph{segnale luminoso} che deve essere
  acceso quando il laser è in funzione perché ci sono delle radiazioni e
  nessuno deve entrare dentro
\item
  Deve essere segnalato con dei cartelli in maniera evidente
\item
  \emph{Deve essere presente in corrispondenza dell'uscita dalla stanza
  un sistema che permetta di interrompere l'emissione all'apertura della
  porta}
\item
  \emph{All'interno della stanza deve essere presente un sistema,
  generalmente a pedale o a pulsante, che permetta di interrompere in
  qualsiasi momento l'emissione del raggio} sia da parte dell'operatore
  sia da parte del paziente.
\end{itemize}

Poi c'è tutta una serie di indicazioni che sono legate alla
\emph{radioprotezione}:

\begin{itemize}
\item
  \begin{quote}
  Sia l'operatore sia il paziente devono indossare degli occhiali perché
  il raggio laser purtroppo si può riflettere nell'ambiente.
  \end{quote}
\item
  \begin{quote}
  La stanza deve avere delle pareti adatte che evitino la riflessione:
  se il raggio viene riflesso e va a colpire gli occhi ci sono delle
  lesioni importanti (da congiuntivite a cheratite a
  cheratocongiuntivite e lesioni alla cornea).
  \end{quote}
\end{itemize}

Esiste un manuale di applicazione del raggio laser in punti ben precisi:
i trigger point. In ogni caso durante la terapia laser normalmente non
vengono consultati perché molti apparecchi hanno nel loro software la
possibilità di scegliere la patologia e il laser in automatico dà il
tipo di applicazione (durata, tempo e modalità). Quindi l'operatore non
deve fare altro che scegliere il tipo di trattamento e poi utilizzare
l'apparecchio.

E' chiaro che la soluzione migliore è quella manuale, m l'operatore deve
conoscere bene l'apparecchio e lo deve saper modulare in maniera
adeguata (le ustioni da laser sono delle situazioni abbastanza noiose da
trattare).

\emph{Effetti}

Sono suddivisi in:

1. EFFETTI BIOLOGICI. Il raggio Laser produce i suoi effetti sulla
membrana cellulare e nei mitocondri, ovvero negli organuli che producono
l'energia della cellula e quindi del corpo umano.

• \emph{\textbf{Incremento dell'attività metabolica,} accelera la
formazione di ATP (carburante della cellula)}

\emph{dall'ADP e degli scambi elettrolitici tra gli ambienti intra ed
extra cellulari attraverso la membrana cellulare. Inoltre aumenta la
produzione di DNA, RNA, aminoacidi e proteine.}

• \emph{\textbf{La terapia del dolore} è causata dall'incremento della
soglia d'eccitazione sulle terminazioni nervose che portano il segnale
del dolore.}

• \emph{\textbf{Vasodilatazione,} per l'incremento del calore locale, di
conseguenza aumentano le attività metaboliche cellulari, stimolazione
neuro vegetativa e modifica della pressione idrostatica intra
capillare.}

• \emph{\textbf{Aumento del drenaggio linfatico} grazie
all'accelerazione della pompa sodio/potassio, provocazione del maggior
assorbimento dei liquidi interstiziali; riattivazione del microcircolo}.

• \emph{\textbf{La spiccata azione antiflogistica e stimolante} per il
tessuto cellulare è dato dallo stimolo a modificare le prostaglandine
che richiamano liquido infiammatorio in prostacicline}

\emph{• \textbf{Piccola Modificazione del PH} all'interno e all'esterno
delle cellule.}

2.EFFETTI TERAPEUTICI sono

• \emph{\textbf{Antalgico e rigenerativo del sistema nervoso periferico}
perché innalza la soglia di eccitazione delle terminazioni nervose del
dolore e attua una stimolazione metabolica della cellula}

• \emph{\textbf{Antinfiammatorio} dato dalla trasformazione delle
prostaglandine in prostacicline e dall'aumento del microcircolo}

• \emph{\textbf{Biostimolante} ottenuto con la stimolazione sul
metabolismo che accelera la cicatrizzazione di ulcere e piaghe, inoltre
provoca la chiusura delle lesioni muscolari che non hanno indicazione
chirurgica grazie alla formazione di tessuto fibroso cicatriziale}

• \emph{\textbf{Decontratturante} per via dell'effetto termico e
dell'aumento del metabolismo}

\emph{INDICAZIONI TERAPEUTICHE}

In passato la laser terapia veniva utilizzata nel soggetto con lesioni
nervose periferiche che presentava lesioni cutanee o delle ulcere anche
se l'indicazione generale non è sicuramente per le ulcere. Oggi ci sono
a disposizione dei sistemi più efficace e meno complessi da gestire per
il trattamento di ulcere e piaghe

Viene invece utilizzata soprattutto per:

\begin{itemize}
\item
  \emph{Tendiniti, borsiti, miositi}
\item
  \emph{Traumi, contusioni, distorsioni}
\item
  \emph{Lesioni muscolari}
\item
  \emph{Ematomi}
\item
  \emph{Versamenti}
\item
  \emph{Riacutizzazioni algiche di patologie artrosiche degenerative}
  come \emph{gonartrosi}, \emph{coxartrosi}, \emph{artrosi della
  tibiotarsica.}
\end{itemize}

Le controindicazioni sono quelle generali di tutti gli apparecchi per
elettroterapia. Non vanno utilizzati:

\begin{itemize}
\item
  \begin{quote}
  In fase acuta. \emph{Possono essere però trattati, non in fase acuta,
  riacutizzazioni i processi flogistici cronici e subcronici.}
  \end{quote}
\item
  \begin{quote}
  \emph{In caso di affezioni reumatologiche perché, essendo di natura
  flogistica come affezione, il calore determinato dai raggi laser può
  peggiorare la patologia.}
  \end{quote}
\item
  \begin{quote}
  In gravidanza o allattamento
  \end{quote}
\item
  \begin{quote}
  In presenza di pacemaker
  \end{quote}
\item
  \begin{quote}
  Non vanno orientati su zone sensibili come la tiroide o il glomo
  carotideo
  \end{quote}
\item
  \begin{quote}
  Vanno utilizzate con moltissima attenzione in soggetti con
  \emph{neuropatie periferiche}, siano esse metaboliche, come il
  diabete, (il soggetto è desensibilizzato quindi sente meno il calore)
  oppure altre patologie periferiche ben definite.
  \end{quote}
\item
  \begin{quote}
  Un'altra cosa che si deve tenere in considerazione è che dato che sono
  trattamenti che danno calore è bene evitare il trattamento laddove ci
  sono varici o condizioni di tromboflebiti.
  \end{quote}
\end{itemize}

\end{document}
