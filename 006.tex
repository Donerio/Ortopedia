\section{Sindrome del tunnel carpale}

Le sindromi canalicolari sono sindromi che coinvolgono sia l'arto superiore che l'arto inferiore e sono caratterizzate dalla \textbf{compressione di un nervo periferico all'interno di un canale inestensibile}.
Nello specifico la sindrome del tunnel carpale è una delle più frequenti dell'arto superiore ed è caratterizzata dalla compressione del \emph{nervo mediano} a livello del tunnel carpale stesso.
\\\\
L'innervazione motoria e sensitiva della mano è legata a tre nervi:

\begin{itemize}
\item
  \textbf{Nervo mediano:}
\begin{itemize}
\item
  INNERVAZIONE SENSITIVA: superficie palmare del primo, secondo, terzo e metà radiale del quarto dito.
\item
  INNERVAZIONE MOTORIA: muscoli dell'eminenza tenar.
\end{itemize}

\item
  \textbf{Nervo ulnare:}
\begin{itemize}
\item
  INNERVAZIONE SENSITIVA: superficie palmare del quinto dito e metà ulnare del quarto dito.
\item
  INNERVAZIONE MOTORIA: muscoli dell'eminenza ipotenar.
\end{itemize}

\item
  \textbf{Nervo radiale:}
\begin{itemize}
\item
  INNERVAZIONE SENSITIVA: dorso della mano.
\end{itemize}
\end{itemize}

\subsection{Definizione}

La sindrome del tunnel carpale è stata scoperta nel 1913 da Marie e Foix.

Il primo intervento di decompressione è stato svolto presso la Mayo Clinic da Galloway nel 1924.

Può essere definita come l'insieme dei sintomi e dei segni conseguenti alla compressione del nervo mediano nel passaggio all'interno del tunnel carpale.

Il \textbf{canale carpale} è un canale osteo-fibroso inestensibile, localizzato tra palmo della mano e polso, attraverso il quale decorrono il \emph{nervo mediano} e i \emph{9} (4 superficiali e 5 profondi) \emph{tendini flessori} per le dita della mano.

È costituito da:

\begin{itemize}
\item
  un PAVIMENTO concavo, composto dalle ossa carpali
\item
  un TETTO fibroso, formato dal \emph{legamento trasverso del carpo}.
\end{itemize}

\emph{Il \textbf{\emph{tunnel carpale}}, è definito come quel canale osteofibroso che ha come pavimento l'estremità distale di radio ed ulna, lo scafoide, il piramidale ed il pisiforme, come tetto il legamento trasverso del carpo, come parete laterale i tubercoli dello scafoide e del trapezio e come parete mediale il pisiforme ed l'uncino dell'uncinato.} \emph{All'interno del canale decorrono diverse strutture, nello specifico troviamo il \textbf{\emph{nervo mediano}}, in posizione alquanto superficiale, i \emph{tendini dei muscoli flessori
superficiale e profondo delle dita}, il \emph{tendine del flessore lungo del pollice} e diversi vasi sanguigni.}

\begin{figure}[!ht]
\centering
\includegraphics[width=0.4\textwidth]{006/image1.jpeg}
\end{figure}

La sintomatologia è dovuta ad un \textbf{aumento pressorio} all'interno di questo canale inestensibile (da alcuni studi si è visto che si può passare da 2,5 mmHg a più di \textbf{30 mmHg}).

Due posizioni (flessione palmare e dorsale) aumentano molto la pressione all'interno del canale carpale, soprattutto in questo tipo di patologia, e sono utilizzate come test specifici per fare diagnosi della patologia.

\subsection{Cause}

Le cause di aumento pressorio all'interno del tunnel carpale sono tutte quelle condizioni che portano a:

\begin{itemize}
\item
  ESPANSIONE del contenuto del canale
\item
  RESTRINGIMENTO di ciò che sta attorno al tunnel carpale.
\end{itemize}

Tra le varie cause troviamo:

\begin{itemize}
\item
  ispessimento del legamento trasverso del carpo
\item
  processo infiammatorio tendineo e peritendineo dei tendini flessori (una delle cause più frequenti)
\item
  neoformazioni, lipomi, cisti all'interno del canale carpale
\item
  frammento osseo di frattura che schiaccia il nervo
\item
  callo osseo esuberante dopo una frattura di polso
\item
  osteofiti derivanti da malattie degenerative
\item
  deposizione sostanze patologiche (come in gotta, amiloidosi, insufficienza renale,\ldots{})
\item
  tumori intraneurali (neurinomi del nervo mediano)
\end{itemize}

\subsubsection{Epidemiologia}

\begin{itemize}
\item
  E' più frequente nel sesso femminile, spesso in età perimenopausale (40-60 anni).
\item
  Il sesso maschile è meno coinvolto: riguarda perlopiù soggetti con attività lavorative che comportano prese di forza, movimenti ripetuti o uso di strumenti vibranti.
\item
  Nella maggior parte dei casi è colpito il lato dominante, ma ci può anche essere una bilateralità.
\item
  Può essere isolata o associata ad altre patologie della mano: \textbf{sindrome di De Quervain}, \textbf{dito a scatto}, \textbf{rizartrosi}.
\end{itemize}

\subsubsection{Eziopatogenesi}

\begin{itemize}
\item
  Idiopatica
\item
  Secondarie:
\begin{itemize}
\item
  Le condizioni già viste che causano aumento di pressione
\item
  Condizioni che aumentano la \emph{suscettibilità del nervo} (come il diabete o \emph{anomalie congenite} del nervo mediano stesso, ad esempio il nervo mediano bipartito, che aumentano la suscettibilità del nervo stesso).
\item
  Forme acute come quelle post-traumatiche.
\item
  La causa più frequente che riduce lo spazio è la \emph{TENOSINOVITE
  DEI FLESSORI} che vanno a schiacciare il nervo mediano.
\item
  \emph{Può poi essere dovuta o a una condizione di \textbf{\emph{aumento del volume del contenuto del canale}} (ad esempio per \emph{cisti sinoviali} o \emph{infiammazione delle guaine tendinee}, ma anche per \emph{tumori} o per altre \emph{patologie espansive}) oppure una \textbf{\emph{riduzione delle dimensioni del tunnel stesso}}, che può essere dovuta o ad \emph{esiti di fratture del polso} guarite con un vizio di consolidazione, o anche ad una \emph{tenovaginite cronica ipertrofica}, la quale è nella maggior parte dei casi idiopatica e legata probabilmente anche ad alcuni \textbf{fattori ormonali} (in particolare il calo degli estrogeni) e ai \textbf{microtraumatismi} (come le attività lavorative con prese di forza o movimenti ripetitivi del polso), mentre più raramente la causa è \textbf{\emph{reumatologica}}.}
\end{itemize}
\end{itemize}

Ci sono patologie che favoriscono l'insorgenza del tunnel carpale :

\begin{itemize}
\item
  diabete
\item
  patologie tiroidee
\item
  iperaldosteronismo
\item
  Insufficienza renale cronica
\item
  gravidanza e allattamento (che, anche se non sono condizioni patologiche, sono comunque predisponenti all'insorgenza della sindrome. A fine gravidanza e allattamento la sintomatologia recede)
\end{itemize}

In queste patologie (diabete, patologie tiroidee, iperaldosteronismo), l'esito dipende dal trattamento della patologia di base, perciò bisogna sempre dire al paziente che il recupero potrebbe non essere completo se il nervo è già rovinato dalla sottostante patologia.

\begin{itemize}
\item
  Bisogna prestare attenzione anche agli scoagulati.
\item
  Bisogna sempre aprire un gesso che dà fastidio, perché se troppo stretto potrebbe dare una compressione nervosa.
\item
  Lussazione delle ossa carpali possono dare compressione.
\item
  Trombosi dell'arteria (raro)
\end{itemize}

Ricapitolando, le cause più frequenti di compressione sono:

\begin{itemize}
\item
  ISPESSIMENTO LEGAMENTO TRASVERSO DEL CARPO
\item
  PROCESSO ESPANSIVO/INFIAMMATORIO DEI TENDINI FLESSORI CHE PASSANO ALL'INTERNO DEL TUNNEL CARPALE
\end{itemize}

\subsection{Diagnosi}

Nella maggior parte dei casi si presenta una signora (perché appunto le femmine sono maggiormente colpite rispetto ai maschi ), che di notte si sveglia alla stessa ora perché si sente la mano pesante e addormentata nella zona di innervazione del nervo mediano.
Quindi, dopo aver scrollato la mano per 1-2 minuti, questa sensazione passa e la paziente si riaddormenta.
Durante il giorno questa sintomatologia ritorna soprattutto in movimenti di \emph{presa statica} (quando si guida, si va in bici), con \textbf{formicolii} simili alla sintomatologia notturna, spesso associati a sensazione di \textbf{tumefazione}.
Ci può essere \textbf{dolore} fino alla radice delle dita o che risale fino al gomito ed alla spalla (irradiazione prossimale).
Al mattino c'è una maggiore rigidità dell'articolazione.
D'inverno i sintomi si verificano molto più frequentemente che d'estate.
\\\\
Negli stadi avanzati queste sensazioni di alterata sensibilità e di dolore diventano costanti e si può arrivare all'\textbf{ipotrofia dell'eminenza tenar} con difficoltà nello svolgimento di movimenti fini delle dita (in passato le donne riferivano l'impossibilità di cucire). Il dolore con il tempo tende a scomparire.
\\\\
Di solito i sintomi sensitivi anticipano quelli motori (è raro che avvenga il contrario).
Il decorso di solito non è particolarmente veloce, anche se in alcuni casi (ad esempio negli anziani e nei diabetici, dove ci possono essere altre neuropatie) l'insorgenza è acuta e si aggrava velocemente.

\subsubsection{Obiettività}

Bisogna innanzitutto osservare la mano e ricercare:

\begin{itemize}
\item
  Eventuali deformità
\item
  Ipotrofia dell'eminenza tenar
\item
  Se il paziente riesce ad effettuare il movimento di opposizione (controllato dal ramo motorio del n. mediano)
\end{itemize}

Esistono dei test neurologici come:

\begin{itemize}
\item
  Test di discriminazione di due punti statici (di Weber)
\item
  Test di soglia
\item
  Valutare se ci sono deficit motori, osservando l'abduttore breve e l'opponente del pollice e il flessore breve delle dita.
\end{itemize}

Si fanno poi dei test provocativi:

\begin{itemize}
\item
  \emph{Test di Tinel}: percussione a livello del passaggio del nervo mediano nel canale carpale. È positivo se c'è una sensazione di scossa nella zona di innervazione del nervo mediano
\item
  \emph{Test di Phalen normale}: si chiede al paziente di mettere le mani dorso contro dorso, tenendo i gomiti a 90 gradi. E' positivo se entro 60 secondi compare formicolio anche intenso se il nervo mediano è compresso.
\emph{\emph{Nb: è bene tenere a mente che questi due test sono sicuramente positivi nelle prime fasi, mentre in fase paralitica possono anche risultare negativi.}}
\item
  \emph{Test di Phalen inverso}: c'è una flessione dorsale della mano e, anche in questo caso, è positivo se il paziente avverte formicolio entro 60 secondi nella zona di innervazione del nervo mediano
\item
  \emph{Test di Durkan}: si schiaccia il nervo in corrispondenza del tunnel carpale. Di solito è positivo se compaiono i sintomi dopo 30 secondi dalla compressione.
\end{itemize}

\textbf{Diagnosi differenziale:}

\begin{itemize}
\item
  Una compressione C5-C6 del rachide, che presenta una sintomatologia simile.
\item
  Alcune patologie neurologiche possono dare sintomi simili, quindi bisogna fare attenti esami neurologici.
\item
  Considerare sempre il diabete, che può avere una neuropatia di base.
\end{itemize}

\subsubsection{Esami strumentali}

L'\emph{elettromiografia} è un esame elettrofisiologico in cui si applicano delle scosse prossimalmente al livello della compressione e si registra la velocità di propagazione dell'impulso -> se è bassa vuol dire che c'è la compressione del nervo.
Permette quindi di rilevare il \textbf{grado} di compressione del nervo e quanto è grave questa compromissione. Rileva anche la \textbf{localizzazione} della compressione del nervo: se la compressione è prossimale (a livello del gomito) o se è distale, a livello del polso.
\\\\
Questo è un aiuto importante dal punto di vista della scelta terapeutica e nel post-operatorio una rivalutazione elettromiografica permette di capire se l'intervento è riuscito.
\\\\
L'\emph{ecografia} è un altro esame che si può fare. È un esame di supporto, però molto meno usato rispetto all'elettromiografia. Usata in caso di sospette lesioni ai tessuti molli.
\\\\
L\emph{'Rx} è utile solo in esiti di patologie ossee.


\subsection{Classificazione Clinica}

A seconda del grado di compressione del nervo si distinguono:

\begin{itemize}
\item[1.] Fase \textbf{algico-irritativa:} in cui il nervo è neuroaprassico. Ci possono essere dei fastidi. \emph{Fase caratterizzata da parestesie prevalentemente notturne in corrispondenza della superficie palmare delle prime 3 dita della mano e della metà del quarto dito, cioè in quella che è l'area di innervazione del nervo mediano.}
\item[2.] Fase \textbf{parestesico-dolorosa:} in cui il nervo è più compresso, con alcune degenerazioni. \emph{Ai sintomi della fase precedente si associano anche \emph{ipovalidità} ed \emph{ipotrofia dei muscoli dell'eminenza tenar} (opponente del pollice, abduttore breve del pollice e capo superficiale del flessore breve del pollice) \emph{e dei primi due lombricali}, con anche ipoestesia nel territorio di innervazione e difficoltà nel controllo della presa degli oggetti}
\item[3.]
  Fase \textbf{atrofico-paralitica:} è la fase finale, la più grave, caratterizzata anche da disturbi motori. \emph{\emph{Marcata atrofia dei muscoli dell'eminenza tenar associata ad ipoanestesia delle prime 3 dita e della metà del quarto, con abolizione del movimento di opposizione del pollice.}}
\end{itemize}

\begin{figure}[!ht]
\centering
\includegraphics[width=0.4\textwidth]{006/image2.jpeg}
\end{figure}

Le fasi 1 e 2 sono reversibili con il trattamento.

Nella fase 3, anche se si va ad intervenire con un intervento chirurgico, sarà molto difficile avere una restitutio ad integrum: ci saranno dei miglioramenti dal punto di vista sintomatologico, ma non ci sarà miglioramento dal punto di vista della forza e dell'atrofia del muscolo.

\emph{Il recupero sarà tanto più veloce tanto più è stata breve la compressione e sarà tanto più completo quanto meno grave è stata la compressione.}

\subsection{Trattamento}

\begin{itemize}
\item
  \textbf{Conservativo} (Stadi 1 e 2): nello stadio 1 si possono fare delle \emph{infiltrazioni di corticosteroidi} (massimo due all'anno) all'interno del canale carpale, attorno al nervo e ai tendini che molto spesso sono infiammati.
Si possono fare anche \emph{terapie fisiche}.
Ma soprattutto ha un effetto positivo il trattamento con \emph{tutori} (solo di notte o giorno e notte) che evitano posizioni di iperestensione e di iperflessione che aumenterebbero la pressione nel canale. Essi immobilizzano il polso in \emph{posizione neutra} (2-9º in flessione
dorsale, 2-6º di deviazione ulnare), in cui il nervo è stressato al minimo.
\item
  Nello stadio 2 e 3 (o qualora il trattamento conservativo non abbia avuto effetto) si può arrivare al \textbf{trattamento chirurgico}: consiste nell'apertura del tunnel carpale attraverso la \emph{sezione del legamento trasverso del carpo} senza danneggiare il nervo mediano che sta sotto. E' effettuato in anestesia locale, quindi in regime ambulatoriale.
Si può fare:
\begin{itemize}
\item
  \emph{a cielo aperto} attraverso l'incisione classica: incisione del palmo della mano lungo l'asse del quarto dito, si scolla cute e sottocute e si seziona longitudinalmente il legamento trasverso del carpo. Una volta sezionato, sotto si trova il nervo e attorno i tendini: se il nervo è troppo compresso si può fare una \textbf{neurolisi}, cioè si libera il nervo dalle aderenze e a volte si toglie anche la sinovia che infiamma il tendine.
{[}Per differenziare \emph{nervi e tendini, basta piegare passivamente le dita: il nervo resta fisso mentre} i tendini scorrono{]}.
Si possono anche fare delle mini incisioni a livello del polso e del palmo della mano, si passa sotto con una sonda e si seziona per via sottocutanea.
\item
  per via endoscopica che offre minori tempi di recupero ma con un maggior rischio di complicanze intraoperatorie.
\end{itemize}
\end{itemize}

\emph{In ogni caso, il trattamento chirurgico garantisce un netto miglioramento del dolore ed un buon recupero sensitivo e motorio, purché si sia andati ad agire tempestivamente (infatti più a lungo il nervo rimane compresso e tanto minore sarà il recupero, fino ad essere del tutto assente).}

\subsubsection{Post-operatorio}

In genere si esegue un \textbf{bendaggio} o si utilizza un
\textbf{tutore}.
Si dice al paziente di muovere le dita e di tenere in alto la mano affinché non si provochino ematomi.
\\\\
Dopo 2 settimane si tolgono i punti ed inizia la cura della cicatrice, che può dar fastidio per 1-2 mesi. Si consigliano \emph{massaggi} (2-3 volte al giorno per una ventina di giorni) per far passare il fastidio legato alla cicatrice.
Si possono associare anche \emph{ultrasuoni}.
\\\\
È necessario avvisare il paziente che non ritornerà subito completamente alla manualità precedente perché i tempi di pieno recupero sono di circa 1-2 mesi.

\subsubsection{Complicanze}

Può sembrare un intervento banale ma il rischio più grave è il \textbf{taglio della branca motoria del nervo mediano.} Essa normalmente si sfiocca oltre il legamento trasverso mediano.

Ci possono però essere delle varianti anatomiche in cui si sfiocca al di sotto o dentro il legamento trasverso: per questo nel sezionare è meglio portarsi verso il lato ulnare piuttosto che direttamente sopra il nervo.

Ovviamente non bisogna stare troppo sul lato ulnare perché si rischia di lesionare il nervo ulnare.

Altra complicazione può essere la recidiva per cicatrizzazione ed ispessimento del legamento trasverso del carpo.

\begin{figure}[!ht]
\centering
\includegraphics[width=0.4\textwidth]{006/image3.jpeg}
\end{figure}

\subsection{Altri siti di compressione dei nervi nell'arto superiore }

\subsubsection{Nervo Ulnare}

Può avere 2 siti di compressione:

\begin{itemize}
\item
  a livello della \textbf{doccia epitrocleolecranica} del gomito. SINTOMI: intorpidimento lungo la zona di innervazione sensitiva del nervo ulnare e nei casi gravi ipotrofia dell'eminenza tenar.
\item
  \textbf{Sindrome del canale di Guyon} se è compresso a livello del polso.
\end{itemize}

\subsubsection{Nervo Mediano}

\begin{itemize}
\item
  A livello del \textbf{canale carpale}
\item
  \textbf{Sindrome del nervo INTEROSSEO ANTERIORE:} quando compresso nella sua branca motoria a livello del gomito.
SINTOMI: prevalentemente motori, a livello dei muscoli dell'avambraccio.
\end{itemize}

\subsubsection{Nervo Radiale}

\begin{itemize}
\item
  Può essere lesionato e stirato nelle fratture del terzo distale dell'omero
\item
  \textbf{Compressione del} \textbf{nervo INTEROSSEO POSTERIORE:} compressione della sua componente motoria, che innerva i muscoli estensori delle dita.
\item
  Compressioni più alte a livello sottoclavicolare (legate per esempio ad una sindrome dello sbocco toracico).
\end{itemize}

\subsection{Patologie compressive nell'arto inferiore}

\begin{itemize}
\item
  \textbf{Sindrome del tunnel tarsale:} è il corrispettivo, a livello del piede, della sindrome del tunnel carpale. Consiste nella compressione del \emph{nervo tibiale posteriore} a livello della doccia retromalleolare mediale.
SINTOMI: formicolii a livello della pianta del piede.
\item
  \textbf{Compressione del nervo sciatico popliteo} \textbf{esterno} a livello della testa del perone (particolarmente vero nelle forme acute delle fratture del piatto tibiale oppure nelle forme acute nel caso di pazienti allettati da molto tempo e che tendono ad avere la gamba extra ruotata, per cui il nervo può essere schiacciato, ma ci sono anche forme idiopatiche).
\item
  \textbf{Sindrome del piriforme:} è una sindrome per esclusione (nello sportivo soprattutto), da compressione del nervo sciatico dal bordo inferiore del muscolo piriforme. Il nervo sciatico può essere lesionato o compresso quando si ha una lussazione dell'anca.
\item
  Il \textbf{nervo femorale} a volte può essere stirato quando si fa un accesso laterale all'anca: uno stiramento da parte dei divaricatori può essere presente nella artrosi concentriche dell'anca.
\end{itemize}
