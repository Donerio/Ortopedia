\documentclass[]{article}
\usepackage{lmodern}
\usepackage{amssymb,amsmath}
\usepackage{ifxetex,ifluatex}
\usepackage{fixltx2e} % provides \textsubscript
\ifnum 0\ifxetex 1\fi\ifluatex 1\fi=0 % if pdftex
  \usepackage[T1]{fontenc}
  \usepackage[utf8]{inputenc}
\else % if luatex or xelatex
  \ifxetex
    \usepackage{mathspec}
  \else
    \usepackage{fontspec}
  \fi
  \defaultfontfeatures{Ligatures=TeX,Scale=MatchLowercase}
\fi
% use upquote if available, for straight quotes in verbatim environments
\IfFileExists{upquote.sty}{\usepackage{upquote}}{}
% use microtype if available
\IfFileExists{microtype.sty}{%
\usepackage{microtype}
\UseMicrotypeSet[protrusion]{basicmath} % disable protrusion for tt fonts
}{}
\usepackage[unicode=true]{hyperref}
\hypersetup{
            pdfborder={0 0 0},
            breaklinks=true}
\urlstyle{same}  % don't use monospace font for urls
\IfFileExists{parskip.sty}{%
\usepackage{parskip}
}{% else
\setlength{\parindent}{0pt}
\setlength{\parskip}{6pt plus 2pt minus 1pt}
}
\setlength{\emergencystretch}{3em}  % prevent overfull lines
\providecommand{\tightlist}{%
  \setlength{\itemsep}{0pt}\setlength{\parskip}{0pt}}
\setcounter{secnumdepth}{0}
% Redefines (sub)paragraphs to behave more like sections
\ifx\paragraph\undefined\else
\let\oldparagraph\paragraph
\renewcommand{\paragraph}[1]{\oldparagraph{#1}\mbox{}}
\fi
\ifx\subparagraph\undefined\else
\let\oldsubparagraph\subparagraph
\renewcommand{\subparagraph}[1]{\oldsubparagraph{#1}\mbox{}}
\fi

% set default figure placement to htbp
\makeatletter
\def\fps@figure{htbp}
\makeatother


\date{}

\begin{document}

\section{Principi di medicina fisica e
riabilitazione}\label{principi-di-medicina-fisica-e-riabilitazione}

\emph{Introduzione} \emph{alla medicina fisica e riabilitativa}

La medicina riabilitativa è una branca ampia che abbraccia tutte le
altre branche perché purtroppo non esiste una specialità, o ce ne sono
davvero poche, che non presupponga una trattamento rieducativo. Basti
pensare alla chirurgia toracica, alla chirurgia addominale, alla
chirurgia bariatrica, alla chirurgia vascolare, alla neurochirurgia dove
c'è una fetta importante di pazienti che occorre gestire con un
trattamento riabilitativo adeguato. Nella maggior parte dei casi si
tratta di \emph{pazienti con patologie neurologiche} o \emph{patologie
traumatiche con o senza esiti neurologici}. Ogni volta che siamo di
fronte ad un soggetto (sia per patologie del locomotore, sia per
patologie in ambito neurologico) per il fisiatra è fondamentale fare una
valutazione ossia un bilancio accurato perché occorre definire quanto
quella patologia ha contribuito a ridurre l'autonomia del soggetto e la
sua qualità della vita. Ci possono essere delle situazioni particolari
in cui entrano in gioco altre componenti che oltre ad essere
fondamentali per la patologia in sé, possono creare delle
sovrapposizioni. Infine dobbiamo partire dal presupposto che il soggetto
che ha una lesione di tipo traumatico potrebbe avere già delle patologie
di base (comorbilità) che possono condizionare l'approccio terapeutico.
\emph{Occorre quindi avere le idee chiare e conoscere la situazione di
quel soggetto, come si comporta, la sua situazione prima dell'evento
morboso, la situazione al momento e quali sono gli obiettivi da
raggiungere}. Per fare questo si seguono degli schemi:

\begin{itemize}
\item
  \textbf{Bilancio della lesione:} sovrapposizioni, co-morbilità per
  chiarire bene il quadro complessivo
\item
  \textbf{Bilancio della funzione} che quella lesione ha ridotto, di
  quanto l'ha ridotta e fino a che punto l'ha ridotta. Se un soggetto ha
  una riduzione delle capacità di eseguire delle attività quotidiane
  bisogna valutare se questa capacità potrà essere reintegrata con
  trattamento riabilitativo o eventualmente chirurgico. Se non sarà
  possibile, ci saranno degli esiti che dovranno essere compensati.
  Questo vale per l'ambito ortopedico, ma a maggior ragione per quello
  neurologico.
\item
  \textbf{Bilancio situazione.} È molto importante avere presente la
  situazione del soggetto \emph{perché rappresenta la prospettiva di
  recupero che ha un individuo in quel momento.} A tal proposito il prof
  ricorda che attualmente ci sono molte indicazioni per il trattamento
  di alcune patologie. In Emilia Romagna ad esempio ci sono indicazioni
  precise per trattare in maniera ottimale la patologia traumatica degli
  arti inferiori e in particolare nel paziente anziano. Per quest'ultimo
  ci sono delle indicazioni per il trattamento della frattura del
  femore. Questa è un'indicazione importante data dalla regione e
  presuppone un certo percorso e la messa in moto di numerose risorse.
  La frattura di femore può essere gestita in maniera chirurgica o meno
  e questo cambia il percorso e le conseguenze terapeutiche. Sono stati
  identificati dei settings riabilitativi per i diversi pazienti in
  relazione alle loro caratteristiche: un soggetto con date
  caratteristiche potrà fare un dato percorso, un altro ne può fare uno
  differente e per esempio un altro non potrà fare nessun percorso
  perché non risponde a determinati requisiti.
\end{itemize}

Bilancio della lesione

Bisogna prendere in esame vari parametri tra i quali:

\begin{itemize}
\item
  Il \textbf{\emph{Dolore}} il quale è dotato di un'ampia variabilità
  sia individuale sia legata alla patologia. \emph{Valutare il dolore
  correttamente e conoscere il tipo di dolore che affligge il paziente
  può permettere di orientarci da un punto di vista terapeutico.} I
  parametri presi in esame per l'analisi del dolore sono:
\end{itemize}

\begin{itemize}
\item
  \textbf{Intensità del dolore.} L'ideale sarebbe quindi poterlo
  quantificare e lo si fa usando delle scale per il dolore. La scala più
  semplice è la \textbf{scala visuo-analogica} che sfrutta una scala
  graduata da 1 a 10 o da 0 a 100 con la millimetrata e il paziente
  segna il livello di intensità percepita del suo dolore. \emph{Non deve
  stupire che ci si concentri sull'intensità percepita piuttosto che
  effettiva perché questo parametro può fare la differenza nella
  valutazione complessiva del paziente}. Ad esempio tantissimi pazienti
  che arrivano in ambulatorio con lombalgie croniche sono soggetti
  depressi e che hanno somatizzato questo aspetto con una conseguente
  soglia del dolore ridotta.
\item
  \textbf{Momento di insorgenza del dolore}
\item
  \textbf{Fattore/i scatenanti il dolore}: cosa ha provocato questa
  sintomatologia dolorosa?
\item
  \textbf{Il ritmo circadiano:} bisogna valutare se e come il dolore si
  presenta durante l'arco della giornata. A tal proposito il prof
  ricorda che i cortisonici di solito sono da utilizzare al mattina, ma
  adesso c'è un nuovo farmaco di questa classe simile al prednisolone
  però con dosaggio nettamente più basso, che agisce in maniera diversa
  e viene somministrato di sera e non la mattina come di solito. Questo
  perché si è compreso che molte patologie sono debilitanti soprattutto
  al mattino quindi è cambiato l'approccio anche in termini di
  somministrazione del farmaco.
\item
  \textbf{Sede e irradiazione}
\item
  \textbf{Tipo di dolore:} \emph{può essere un dolore di tipo meccanico
  oppure no, può accentuarsi con il movimento e attenuarsi con il
  riposo, può manifestarsi di giorno oppure di notte perciò è importante
  differenziare tra:}
\end{itemize}

\begin{itemize}
\item
  \textbf{Dolore meccanico} che in quanto tale insorge con il movimento
  e cessa con il riposo. In questo caso avremo certe indicazioni sul
  tipo di approccio farmacologico da mettere in atto: sarà necessaria
  una terapia farmacologica di tipo analgesica
\item
  \textbf{Dolore neuropatico} che spesso è notturno e insorge in maniera
  casuale e non cessa con il riposo, ma dà la sensazione che si attenui
  con riprese. In questo caso non si ricorrerà ad una terapia analgesica
  bensì anti-infiammatoria usando FANS. A tal proposito ci sono degli
  approcci moderni che arrivano dagli USA e che promettono di essere più
  efficaci dell'approccio standard. È prevista l'associazione
  analgesico-FANS: esistono delle formulazione al giorno d'oggi che
  combinano entrambi i principi attivi. Ricordare a tal proposito che
  uno degli obiettivi è importante mettere il proprio paziente in
  condizioni di avere meno fastidio possibile e più possibilità di
  approccio in termini di fisioterapia. Come esempio consideriamo il
  caso di una una signora anziana di ottant'anni magati con un po' di
  demenza e già in terapia per altre patologie. La signora in questione
  viene operata d'urgenza per una frattura del femore con inserimento di
  un chiodo. La paziente lamenta dolore e se non vado a valutare la
  componente dolorosa, non riuscirò a metterla in carico e farla
  camminare precocemente così da permetterle di recuperare le attività
  di ogni giorno. Quest'ultimo diventa un problema ed è fondamentale.
  Oltre alla terapia farmacologica \emph{ricordiamo che c'è anche quella
  strumentale} che in certi contesti è utile in termini di trattamento
  del dolore. È l'unica terapia ancora oggi riconosciuta a livello
  internazionale ed è stata inserita nel protocollo per esempio della
  lombalgia che colpisce l'85\% della popolazione. \emph{Il dolore va
  trattato con attenzione e saputo gestire anche con approcci diversi
  nelle varie situazioni.}
\end{itemize}

\begin{itemize}
\item
  Bisogna valutare la presenza di \textbf{flogosi}. Di solito la
  individuiamo grazie agli \emph{indici ematici} oppure a \emph{livello
  locale} attraverso un esame obiettivo: una cicatrice chirurgica può
  dare flogosi locale perché l'intervento ha creato delle zone di
  passaggio e si può quindi creare una zona locale di infiammazione. Si
  può avere anche un versamento o altre varie situazioni che vanno
  gestite. Se c'è flogosi locale la terapia rieducativa strumentale è
  controindicata: \emph{è una delle controindicazione assoluta in caso
  di flogosi acuta mentre quando è subacuta si può agire. Solo in rari
  casi si può agire in fase acuta. }
\end{itemize}

Il dolore è sicuramente il parametro fondamentale da valutare in ogni
situazione però se ho una patologia che interessa un organo o un
apparato sarà necessario approfondire ulteriori aspetti:

\begin{itemize}
\item
  L'\textbf{\emph{Articolarità}:} devo valutare il movimento residuo in
  quella situazione \emph{per avere un quadro dell'ampiezza dei
  movimenti e della possibilità di effettuare movimenti di fisioterapia
  su quell'articolazione}. Se interessa gli arti devo \emph{comparare}
  il movimento di un arto con il \emph{controlaterale} ed infine
  \emph{misurare i gradi di escursione articolare} con il goniometro che
  è uno strumento di uso quotidiano per l'ortopedico.
\item
  Lo \textbf{\emph{stato della muscolatura}:} alcune volte i pazienti
  sono mobilizzati a letto e questo vale soprattutto per gli anziani.
  Esiste una sindrome detta da immobilizzazione che è molto importante e
  va gestita, trattata, ma possibilmente evitata cercando di non
  allettare troppo a lungo un paziente. Lo stato della muscolatura si
  valuta con manovre contro resistenza per misurare la forza di quel
  muscolo e ci sono delle scale da 0 a 5 per stimare la \emph{forza}, la
  \emph{consistenza} e il \emph{tono muscolare} (se è per esempio
  ipotonico). Una possibile situazione è quella di un paziente allettato
  per politrauma che è stato in rianimazione ed è immobile: ha i muscoli
  rigidi, contratti, retratti. Questi dovrà essere mosso prima
  passivamente in rianimazione e in seguito, quando esce dalla
  rianimazione, bisognerà farlo muovere attivamente. Dopo dovrà
  eventualmente essere sottoposto ad intervento chirurgico.
\item
  \textbf{\emph{Deformità}:} un trauma può dare deformità locale che può
  ridurre alcune attività. Bisogna valutarle in maniera attenta in
  quanto quel soggetto in futuro potrà dover usare un supporto, un
  tutore.
\item
  \textbf{\emph{Stato della cute}:} deve essere integra, non avere
  soluzioni di continuità e non deve avere flogosi locale soprattutto se
  vado ad utilizzare delle terapie strumentali contro il dolore come la
  TENS. TENS sta per \emph{Neuro Stimolazione Elettrica Transcutanea} ed
  ha un buon effetto analgesico: in alcuni centri di politraumi sono
  stati introdotti lettini che conducono deboli correnti elettriche per
  dare effetto analgesico e per stimolare e aumentare la propriocezione
  del soggetto.
\end{itemize}

Bilancio Funzionale

\emph{Deve essere distinto tra funzione dell'arto superiore e funzione
dell'arto inferiore}

\begin{itemize}
\item
  Per l\textbf{'\emph{arto inferiore }}il progetto deve essere rivolto
  alla \textbf{ripresa della deambulazione spontanea}, naturale. A tal
  fine devo valutare se il soggetto riesce a controllare il tronco
  perché il presupposto per la verticalizzazione, ossia il passaggio
  dalla posizione seduta a quella in piedi, è controllare il tronco.
\end{itemize}

\begin{quote}
Per farlo devo mettere il paziente seduto. Se però lo metto su una
sedia, è difficile accertarsi che il controllo del tronco ci sia a meno
che non sia del tutto deafferentato; perciò lo metto seduto con le gambe
fuori dal letto e le mani di fianco. \emph{Se non ha oscillazioni,
antero pulsioni, latero pulsioni(destra\textbackslash{}sinistra) o
retropulsioni allora c'è un controllo sul tronco e una buona possibilità
che riesca a rimanere in piedi.}

Dopodiché per verticalizzare il soggetto devo tener conto di alcuni
parametri:
\end{quote}

\begin{itemize}
\item
  \emph{Pressione},
\item
  \emph{Emoglobina}: se un soggetto ha un trauma o ha subito un
  intervento e perde 2.5 di Hb ha buone possibilità di avere crisi
  ipotensive
\item
  La \emph{frequenza cardiaca}: se è anziano e ha delle problematiche,
  devo valutare la frequenza da seduto e nel passaggio in posizione
  verticale
\item
  \emph{La respirazione}
\item
  Gli \emph{atti respiratori}
\item
  Per alcuni pazienti si guarda anche \emph{la frazione di eiezione
  ventricolare}. Consideriamo il caso di un paziente al quale deve
  essere messa una protesi dopo amputazione e questa protesa aumenta la
  forza del tendine del 25\% e anche il consumo di ossigeno. Se il
  soggetto è dispnoico, quella protesi non la userà mai. Perché appena
  camminerà con la protesi andrà subito in dispnea severa. Vanno perciò
  considerati molti piccoli aspetti.
\end{itemize}

\begin{itemize}
\item
  Per l'arto \textbf{\emph{superiore}} consideriamo:
\end{itemize}

\begin{itemize}
\item
  La \emph{prensione} per quanto riguarda la mano
\item
  La \emph{forza di presa} o pina
\item
  Riconoscimento della \emph{posizione dell'articolazione nello spazio}
\item
  La \emph{capacità di eseguire movimenti}
\item
  \emph{Il \emph{senso cinestesico} ovvero il senso della posizione
  nello spazio}
\item
  \emph{\emph{Limiti di movimento sui piani articolari ad esempio per
  l'articolazione scapolo-omerale}}
\end{itemize}

Bilancio della situazione

\emph{Una volta trattato, il soggetto deve poter riprendere le normali
attività quotidiane e lavorative.} Dobbiamo perciò valutare l'handicap e
nello specifico i seguenti parametri

\textbf{-} \textbf{Cosa faceva prima dell'evento}

\begin{itemize}
\item
  \textbf{Contesto familiare:} è stato dimostrato che quando il soggetto
  è motivato e i familiari sono disponibili, c'è una gestione ottimale
  del paziente. \emph{Diversi studi hanno evidenziato che i famigliari
  che riescono a far eseguire al paziente movimenti illustrati
  precedentemente da un professionista, gli permettono un recupero
  migliore.} C'è attualmente una grande attenzione al care giver
\end{itemize}

\begin{itemize}
\item
  \textbf{Attività quotidiane}: se riusciva a fare tutti gli atti
  quotidiani
\end{itemize}

\begin{itemize}
\item
  \textbf{Attività nel tempo libero}
\end{itemize}

\begin{itemize}
\item
  \textbf{Attività professionali}
\end{itemize}

Tecniche di rieducazione

\emph{Dopo aver effettuato questi bilanci, possiamo mettere giù un
\textbf{\emph{programma riabilitativo}} ponendoci degli obiettivi da
raggiungere e rispettare, allo scopo di correggere la funzione e
raggiungere la condizione pre-patologia e laddove questo non sia
possibile, dobbiamo cercare di compensare quella funzione danneggiata.}
Bisogna dunque:

\begin{itemize}
\item
  \textbf{Recuperare la funzione} ovvero \textbf{\emph{restituzione
  completa}}
\end{itemize}

\begin{quote}
Oppure
\end{quote}

\begin{itemize}
\item
  \textbf{Compensare la funzione} quindi \textbf{\emph{non ho raggiunto
  l'obiettivo del recupero completo}}, ma ci sono degli esiti e scopo
  del fisioterapia è quello di mettere il paziente nelle condizione di
  \textbf{\emph{migliorare}} (perché non è possibile il recupero)
  \textbf{\emph{certe situazioni}}
\end{itemize}

Riguardo quest'ultimo punto c'è una branca gestita dai terapisti
precisamente dai \emph{terapisti} \emph{\emph{occupazionali}} che
seguono il paziente per far sì che possa attendere agli atti di vita
quotidiana: se il paziente a delle limitazioni per cui non riesce a
preparare il caffè con la macchinetta a mattina, allora cercheranno
insieme al paziente di mettere a punto delle strategie utili per farlo
in maniere ottimale e continuare a farlo

Una volta raggiunto l'obbiettivo devo \textbf{mantenere quel risultato a
lungo termine} ed evitare \textbf{una recidiva} laddove ci siano
situazioni che possano recidivare.

La rieducazione e riabilitazione comprendono varie tecniche: nel caso
della rieducazione respiratoria, si insegnano ai pazienti i movimenti
per migliorare l'afflusso venoso o fare in modo che siano accelerati i
processi riparativi.

Le tecniche di rieducazione più utilizzate sono:

\begin{itemize}
\item
  \textbf{Kinesiterapia;}
\end{itemize}

\begin{itemize}
\item
  \textbf{Ergoterapia;}
\end{itemize}

\begin{itemize}
\item
  \textbf{Mezzi fisic}i che sono le \textbf{terapie strumental}i. Prima
  non c'erano quasi, si usava per esempio un mattone refrattario
  scaldato avvolto in un panno e poi appoggiato per fare la
  termoterapia: fondamentalmente funzionava quanto le macchine usate
  oggi però non c'erano indicazioni su quanto, come e quando metterlo
\end{itemize}

Kinesiterapia

Può essere:

\begin{itemize}
\item
  \textbf{Passiva}: se eseguita dal tecnico
\end{itemize}

\begin{itemize}
\item
  \textbf{Attiva assistita}: il paziente esegue il movimento con
  supervisione del fisioterapista
\end{itemize}

\begin{itemize}
\item
  \textbf{Attiva:} cioè il fisioterapista dà un programma che il
  paziente esegue tranquillamente e dopo il fisioterapista farà delle
  valutazioni
\end{itemize}

Per ogni tipo di patologia riscontrata il fisioterapista è inoltre
tenuto a fare una \textbf{valutazione di base}. Ci sono diverse scale di
valutazione per le varie patologie e in ambito ortopedico sono
fondamentali anche se sono soggettive però servono in quanto danno
un'idea della situazione, Sono utilizzate per una \textbf{valutazione di
base}, \textbf{seguire nel tempo l'evoluzione del programma}, \textbf{se
questo sta effettivamente funzionando}.

\emph{Sono inclusi in questa categoria anche:}

\begin{itemize}
\item
  \textbf{La massoterapia (massaggio)}: terapia manuale eseguita dal
  massoterapista, è classificata come kinesiterapia passiva. È
  utilizzata anche per preparare alla kinesiterapia, in quanto il
  massaggio è molto efficace nel predisporre al meglio un'articolazione
  al movimento, dato il suo effetto vasodilatatorio.
\end{itemize}

\begin{itemize}
\item
  \textbf{Le rieducazione posturale globale} che è classificabile come
  una tecnica di kinesiterapia attiva e assistita. Il pilates rientra
  nella rieducazione posturale: sicuramente non eccelsa e con dei
  limiti, ma era nata con l'idea di prendere tecniche di riabilitazione
  e gestirle in maniera più semplice in palestra, usando il controllo di
  muscolatura e della respirazione
\end{itemize}

Ci sono inoltre \emph{\textbf{tecniche Kinesiterapiche particolari}:}

\begin{itemize}
\item
  \textbf{Tecniche KABAT}: tecnica di facilitazione neuromuscolare di
  tipo propriocettiva in cui il movimento è facilitato stimolando i
  propriocettori.
\item
  \textbf{Rieducazione propriocettiva}: è fondamentale nel caso di
  articolazioni e tendini e di alcune situazioni a livello dell'osso in
  cui ci sono propriocettori. Questi propriocettori sono importanti per
  definire la posizione nello spazio di quell'articolazione ovvero il
  senso cinestetico. Ci sono alcuni tipi di intervento che riducono la
  propriocezione (interventi di ricostruzione di legamenti della
  caviglia, di inserzione di protesi d' anca, di ginocchio, di spalla o
  anche interventi per la cuffia dei rotatori) e con tale tecnica si
  tenta di recuperare questa forma di sensibilità. \emph{Nella
  kinesiterapia di base si utilizzano delle pedane tonde o quadrate
  (tavolette di Freeman) per mettere il soggetto in una condizione di
  instabilità, allo scopo di determinare contrazioni muscolari riflesse
  in seguito a stimolazioni propriocettive.}
\end{itemize}

\begin{itemize}
\item
  \textbf{Idrokinesiterapia}: è una tecnica importante, utilizzata in
  molte situazioni soprattutto in quelle fasi in cui soggetto non deve
  caricare. Si sfrutta il galleggiamento per far eseguire dei movimenti
  che a secco il paziente non riuscirebbe a gestire e a fare. Può essere
  eseguita a partire dal galleggiamento fino all' immersione graduata
  perché in base al grado di immersione posso ridurre la forza peso di
  quel soggetto
\end{itemize}

Agenti fisici (terapie strumentali)

Questi sono utilizzati a seconda dell'effetto che voglio ottenere:

\begin{itemize}
\item
  \textbf{Agenti termic}i (crioterapia-termoterapia). La crioterapia e
  la termoterapia lavorano solo parzialmente in modo diverso! Bisogna
  considerare che con la crioterapia, l'effetto di vasocostrizione è
  solo temporaneo perché dopo interviene una vasodilatazione profonda
  per cui l'effetto è paragonabile all'applicazione di calore. Se
  volessi un effetto diverso, potrei usare il freddo in maniera
  intermittente, ma mai continua: il freddo usato in maniera prolungata
  produce desensibilizzazione cutanea e in certe situazioni viene
  proprio adoperato per tale ragione
\end{itemize}

\begin{itemize}
\item
  \textbf{Agenti vibratori} (ultrasuoni, onde d'urto);
\end{itemize}

\begin{itemize}
\item
  \textbf{Elettroterapia} (ionizzazione) che comprende:
\end{itemize}

\begin{itemize}
\item
  \emph{Correnti elettriche a bassa frequenza:} antalgico
\end{itemize}

\begin{itemize}
\item
  Correnti elettriche \emph{a media frequenza}: eccito motorie
\end{itemize}

\begin{itemize}
\item
  Correnti elettriche \emph{ad alta frequenza}: il radar che non si usa
  più mentre più usata è la diatermia o ipertermia endogena che dà la
  possibilità di ottenere un riscaldamento profondo invece con un
  apparecchio che scalda la cute, si tratta di termoterapia esogena. Ad
  oggi esistono elettroterapie di stimolazioni sia del muscolo normale
  sia del muscolo parzialmente o totalmente denervato
\end{itemize}

\begin{itemize}
\item
  \textbf{Laser} che può essere considerato come una forma di
  termoterapia
\end{itemize}

\begin{itemize}
\item
  \textbf{Magnetoterapia:} è un po' a metà strada perché può essere
  prodotta da magneti naturali o da magneti artificiali con l'uso di un
  campo elettrico che a sua volta crea un campo magnetico. Quelli
  naturali hanno potenze altissime e sono applicati con risultati
  importanti
\end{itemize}

Ergoterapia

É la \textbf{\emph{terapia occupazionale}} o \textbf{\emph{``del
movimento per il movimento'',}} si avvale di attività artigianali o
ludiche per:

\begin{itemize}
\item
  \textbf{Migliorare una funzione che si è persa nel quadro di una
  rieducazione globale }
\item
  \textbf{Favorire la compensazione di funzioni deficitarie in
  conseguenza ad una patologia o ad un trauma}
\end{itemize}

Alcuni includono anche la \emph{meccanoterapia} nello stesso contesto
dell'ergoterapia. La meccanoterapia è quella che si avvale dell'utilizzo
di macchine (che sono le stesse che si trovano nelle palestre) per
recuperare per esempio il tono muscolare o altro.

Tutori e Ortesi

Hanno una funzione di \textbf{prevenzione} o \textbf{correzione}. Le
tutorizzazioni possono essere:

\begin{itemize}
\item
  \textbf{Statiche}
\item
  \textbf{Dinamiche} che hanno il vantaggio di consentire un determinato
  movimento di un'articolazione in maniera corretta o addirittura
  consente di compiere un movimento che il paziente da solo non è in
  grado di realizzare.
\end{itemize}

\emph{Vi sono numerosissimi tipi di tutori o ortesi realizzati con
materiali termo modellati su uno stampo così da dare la possibilità di
seguire in modo adeguato l'articolazione. Anche i plantari sono ortesi,
utilizzati per dismetrie della colonna vertebrale, difetti di appoggio,
metatarsalgie.}

Prevenzione

Recuperata la funzione bisogna attuare un \textbf{\emph{programma di
prevenzione secondaria}} che \textbf{\emph{impedisca}} eventuali
recidive. Bisogna quindi consigliare di fare attività che possano
aiutare il paziente: un paziente con cervicalgia o lombalgia che ha
fatto varie terapie (farmacologiche e strumentali), deve aver acquisito
conoscenze su come muoversi (garantire un certo grado di attività che
però non sia iperattività), su cosa fare per mantenere il risultato
ottenuto ed evitare così la recidiva.

Quindi è necessario:

\begin{itemize}
\item
  \textbf{Evitare determinati movimenti se c'è una limitazione di
  movimento} che non è stato del tutto recuperato
\item
  \textbf{Operare nelle migliori condizioni biomeccaniche} cioè
  utilizzare la leva migliore che viene richiesta per quel movimento
\item
  \textbf{Lottare contro l'infiammazione}
\item
  \textbf{Lottare contro il dolore }
\item
  \textbf{Lottare contro le limitazioni articolari}
\end{itemize}

Per esempio quando recupero il gomito se non si continuano a fare
esercizi di mantenimento, si rischia poi di perdere irrimediabilmente
qualche grado. Si tratta di una situazione problematica nei bambini e
richiede molto lavoro infatti se non si esercitano, si possono trovare
con un gomito flesso che non si estende del tutto e perdono
irreversibilmente il movimento di prono supinazione. Ovviamente sarà un
fattore limitante per eventuali attività.

Oggi si lavora anche con la realtà virtuale: si usano degli schermi, il
soggetto indossa una cuffia ed è in una modalità detta immersiva in cui
viene simulato il movimento dell'articolazione e il paziente ha la
sensazione che il suo arto posso entrare in quel movimento.

Il reinserimento sociale del soggetto è fondamentale e deve essere
realizzato il prima possibile in base alle condizioni del paziente e
bisogna fare in modo che sia perfetto. Questo si può realizzare solo
facendo in modo che il paziente sia al centro di un percorso
riabilitativo che presupponga l'intervento di vari specialisti: medico,
internista, fisiatra, psicologo e talvolta l'assistenza sociale perché
alcuni soggetti sono soli e non hanno un'assistenza a domicilio

Per concludere bisogna ricordare \textbf{che la lesione iniziale non può
considerarsi guarita, se alla riparazione anatomica del danno non fa
seguito un ottimale ripristino della funzione compromessa dall'evento
morboso}. Il fisiatra deve:

\begin{itemize}
\item
  \emph{Stabilire i momenti e sequenze} (per esempio stabilendo un ciclo
  di sedute) \emph{della riabilitazione} cercando il miglior esito
  possibile.
\item
  \emph{Bisogna raggiungere la migliore evoluzione della cura
  rispettando l'anatomia, funzione e biomeccanica della struttura
  interessata. }
\end{itemize}

\end{document}
